%!LW recipe=latexmk (xelatex)
%!TEX program = xelatex

\documentclass[10pt]{article}

\input{../aircraftdata/style.tex}

\runningtitle{Ground Unit Data}
\setcounter{secnumdepth}{0}
\setcounter{tocdepth}{1}

\setlength{\tabcolsep}{0.3em}

% Add some space between footnotes.
\setlength{\footnotesep}{11pt}

% Don't break footnotes between columns so easily.
\interfootnotelinepenalty=1000

\iffalse
\newcommand{\note}[1]{%
    \footnote{%
        \setlength{\parindent}{0pt}%
        \setlength{\parskip}{\smallskipamount}%
        #1
    }%
}

\newcommand{\notesection}[1]{%
    \par{\bfseries #1}\par
}

\newcommand{\clearnotes}{\clearpage}
\else
\newcommand{\note}[1]{\relax}
\newcommand{\notesection}[1]{\relax}
\newcommand{\clearnotes}{\relax}
\fi

\twocolumn

\begin{document}

\title{Air Power:\\[1ex] Ground Unit and Ground Target Data\\[1ex]
1950 to 2010}
\author{Alan Watson Forster}
\date{1 January 2026}

\maketitle

\thispagestyle{empty}

\section{Introduction}

The document describes the properties of ground units and ground targets in the \emph{Air Power} game system, focusing on those relevant for the period from 1950 to 2010.

It builds on the information in the \emph{Air Strike}, \emph{Eagles of the Gulf}, and \emph{The Speed of Heat} games, the \emph{Air Power Journal}, and from Malcolm Pipe's extensive compendium of AAA and SAM units. However, the data have been revised to improve their accuracy and internal consistency, a wider variety of ground combat units are covered, and the descriptions of the units and their weapon systems have been expanded.

\section{Ground Units}

A ground \emph{unit} is a combat or logistics formation of infantry, armor, transports, AAA, SAM launchers, or radars. This section discusses them in general.%
\note{
    Much of this section should be included in the 3A rules, but it's here for the moment.
}

All ground units have these common properties which together determine their visibility to aircraft, their vulnerability to air-to-ground and ground-to-ground attacks, their capacity for ground-to-ground attacks, their ability to move through difficult terrain, and value in VPs:
\begin{itemize}
    \item basic type (e.g., infantry or tanks);
    \item size;
    \item damage resilience;
    \item defense strength and target class;
    \item mobility;
    \item sighting range in hexes;
    \item the VPs awarded for 3D/2D/D damage.
\end{itemize}
Many units also have properties related to their air-defense capabilities.

\subsection{Size}

A ground unit can be a squad, section, platoon, or battery according to its number of soldiers, vehicles, guns, or launchers. Typically:

\begin{itemize}
    \item A squad has five to ten soldiers and/or one vehicle, gun, or launcher;
    \item A section has ten to twenty soldiers and/or two vehicles, guns, or launchers; and
    \item A platoon or battery has thirty to fifty soldiers and/or three to six vehicles, guns, or launchers.
\end{itemize}

The term “platoon” is more commonly used for infantry and tank units and “battery” for artillery and air-defense units, but within the game they are equivalent.

Companies, battalions, and other larger units exist above platoons or batteries, but they are represented by multiple units. For example, an infantry company might be formed by three infantry platoons.

\subsection{Damage Resilience}

A ground unit has a numerical damage resilience, which is the amount of damage it can suffer before being killed. It is determined by the size of the unit:%
\note{
    Air Strike has a form of this, in that units whose counters have white text are only able to take 1D of damage before being killed (see the last paragraph of rule 28). Such units include infantry SAMs, armored CCUs, and many vehicle SAMs. TSOH also mentions units being able to take only 2D damage, but it doesn't have any relevant units since it attaches SAM teams to infantry platoons rather than having them as separate (see below) and doesn't have a separate FAC unit. In both cases, the idea seems to have been incompletely implemented in the rules. I'm generalizing and homogenizing the idea. This is important as we need sections for mobile AAA units with two vehicles and for the Oerlikon GDF (as Skyguard can only control two guns and not a full battery). To a large degree, a section is equivalent to a platoon that has taken 1D and a squad is equivalent to a platoon that has taken 2D, so the mechanics of the game are really not impacted.
}

\begin{itemize}
    \item A squad has a damage resilience of 1D;
    \item A section has a damage resilience of 2D; and
    \item A platoon or battery has a damage resilience of 3D.
\end{itemize}

\subsection{Defense Strength and Target Class}

A ground unit has a numerical defense strength and target class, which are used to calculate the odds of an air-to-ground attack and in ground-to-ground combat.

A ground unit can be a hard, open, or soft target according to the degree to which it provides armored protection to its personnel and weapons:
\begin{itemize}

    \item A hard target has armored protection both from side and top attacks and is denoted by an underlined defense strength. It is classified as heavy, medium, or light according to its degree of protection, but this classification is simply descriptive. Most tanks, IFVs, and APCs are hard targets.%
        \note{
            Most tracked SAM vehicles are classified as hard targets, despite many of them having completely exposed missiles. If an air-burst bomb explodes above a SAM-4 launcher, it is not going to shrug it off. I think many SAM vehicles are tracked mainly for \emph{mobility} and perhaps for crew protection, but in terms of a mission-kill on their weapon systems, they are soft targets. I've not changed anything yet, but we need to think about this.
        }

    \item An open target has armored protection from side attacks but not from top attacks and is denoted by an underlined defense strength followed by an exclamation point. It is normally treated as a hard target, but against napalm, fire, FAE, and air-burst HE bombs, it is considered to be a soft target. The BTR-50P APC and the M3 half-tracked APC are examples of open targets.%
        \note{
            Open targets are relevant to the 1956 and 1967 Arab-Israeli Wars where both sides used M3s. If an M3 is attacked by an air-burst or napalm bomb, the crew and passengers will suffer more or less as if they were in the open.
        }

    \item A soft target does not have armored protection and does not have an underlined defense strength. Examples include infantry, towed artillery, unarmored vehicles, and armored vehicles that do not fully protect their personnel and weapons such as the M-107 and M-110 tracked artillery vehicles.%
        \note{
            Nothing like the M-107/110 is included in Air Strike, but again I would assert that it is tracked mainly for mobility. Effectively, these are towed artillery installed in an exposed position on a tracked hull and have almost no protection for most of the crew.
        }

\end{itemize}

\subsection{Representation}
\begin{comment}
Figures \ref{figure:ground combat-units}, \ref{figure:aaa-units}, \ref{figure:radar-and-ccu-units}, \ref{figure:infantry-sam-units}, and \ref{figure:sam-launcher-units} show the graphical representation of ground combat units as counters.
\end{comment}

Figures \ref{figure:ground-combat-units}, \ref{figure:aaa-units}, \ref{figure:radar-and-ccu-units}, and \ref{figure:infantry-sam-units} show the graphical representation of ground units as counters.

The representation is based on NATO symbols, including those for infantry, armor, anti-armor, air defense, guns, missiles, tube artillery, and rocket artillery. One addition is that a rectangle denotes an unarmored vehicle such as a truck or tracked carrier. A truck or mobile unit has two wheel symbols in the lower position, whereas a tracked carrier has a tracked symbol.

All vehicular units have either an armored or unarmored vehicle symbol, all infantry units have an infantry symbol, and towed units have none of these.

The letter T in the central position indicates a transport capacity (e.g., APCs, IFVs, cargo trucks, and tracked cargo carriers). Words and letters in the lower position indicate variants on a basic type: the letters H, M, L, and O on an armored unit indicate heavy, medium, light, and open armored protection; the word WPN or FAC on an infantry unit indicates a weapons or FAC unit; and the letter H on an infantry SAM unit indicates a heavy version of the launcher (e.g., the LML version of Starstreak).

The size of a unit is indicated by one, two, or three dots (for squads, sections, or platoons) or one vertical bar (for batteries) in an upper position.

\subsection{Mobility}

A ground unit has a mobility that may restrict the hexes it may occupy according to the terrain in and around those hexes.
Infantry units, armored units (whether tracked and wheeled), and tracked units have unrestricted (U) mobility. Truck and mobile (i.e., truck-borne) units have good (G) or poor (P) mobility. All other units have towed (T) mobility.%
\note{
    The inspiration for this rule was \href{https://commons.wikimedia.org/wiki/File:Aerial_view_of_the_road_through_Mitla_Pass_dotted_with_wrecked_Egyptian_vehicles_and_armour_after_Israel_air_force_attacks._June_1967._D326-013.jpg}{this photograph} of the Mitla Pass from 1967.
    Also, think of the Falklands; the terrain was open, but impassable to everything except the CVR(T)s and Bv~202s. The basic rule is effectively: don't put trucks or towed weapons in forests unless there is a road.
    The advanced rule is effectively: if the terrain is difficult, keep your trucks and towed weapons near roads even in clear terrain.
    I note that this rule is not infallible: in the Falklands War the Panhard AMLs were restricted to roads and trails, and this corresponds to poor mobility but under this rule they would still have unrestricted mobility.
    A scenario could fix this problem by explicitly giving them poor mobility.
}

A scenario may modify the mobility classes of ground units and the associated restrictions.
One common modification is to explicitly indicate the hexes occupied by each ground unit; in this case this rule is largely superfluous.%
\note{
    Not entirely superfluous: if we allow ground units to move, then the restrictions are relevant.
}

\subsubsection{Basic Rule}

A ground unit may occupy hexes as follows:
\begin{itemize}
    \item A unit with unrestricted mobility may occupy: any urban hex; any clear hex; or any forest hex.

    \item A unit with good, poor, or towed mobility may occupy: any urban hex; any clear hex; or any forest hex with a road or trail.
\end{itemize}

The basic rules do not distinguish between good, poor, and towed mobility.

\subsubsection{Advanced Rule}

The following restrictions supersede those in the basic rules:
\begin{itemize}
    \item A unit with unrestricted mobility may occupy: any urban hex; any clear hex; or any forest hex.

    \item A unit with good mobility may occupy: any urban hex; any clear hex; or any forest hex with a road or trail.

    \item A unit with poor mobility may occupy: any urban hex; any clear or forest hex with a road, trail, or runway; or any clear hex adjacent to an urban hex or another clear hex with a road, trail, or runway.

    \item A unit with towed mobility may occupy hexes according to the mobility of its transport.
\end{itemize}

Figure~\ref{figure:mobility} illustrates these rules.

\begin{figure}
    \centering
    \includegraphics[width=1\linewidth]{figure-mobility.png}
    \caption{Mobility. {\normalfont A ground unit with poor mobility may only occupy the hexes marked with green dots; one with good mobility may occupy those marked with yellow or green dots; and one with unlimited mobility may occupy those marked with red, yellow, or green dots. No ground unit may occupy the hex marked with a black dot.}}
    \label{figure:mobility}
\end{figure}

When this advanced rule is used, a scenario must specify (or the players must agree) whether truck and mobile units have good or poor mobility, according to the terrain and the capabilities of the trucks represented, and explicitly or implicitly assign transport units to towed units.

For example, a scenario set in the open desert of southern Iraq might specify that trucks have good mobility, but one set in the Falkland Islands might specify that they have poor mobility. Similarly, a scenario might specify that cargo trucks have poor mobility, but mobile artillery units have good mobility. Finally, a scenario might not assign an explicit transport unit to a towed artillery battery, but might state that it has been transported by helicopter and so can be placed as if it had a transport unit with good mobility.

\subsection{Transportation}

Certain ground units and targets may be \emph{transported} by certain \emph{transport} units.%
\note{
    The idea of transporting units and cargo comes from the “Thunder and Fire” scenario we're currently playing. The rules here simply generalize it slightly, incorporate the idea of mobility, and add rules for mounting and dismounting as an advanced rule. They also generalize POL targets to POL and supplies targets.
}

An infantry or towed unit may be transported by a specific associated IFV, APC, truck, or tracked-carrier unit of at least the same size.
Furthermore, the transported unit may be \emph{mounted} or \emph{dismounted}.

A POL or supplies target may be transported by a truck or tracked-carrier platoon as \emph{cargo}.
Transported cargo must always be mounted.

A scenario may add further restrictions on transported units or cargos and their transports beyond those in this rule.
\subsubsection{Basic Rule}

\paragraph{Mounted Units.}

A mounted unit or mounted cargo is not represented by a separate counter, does not count for stacking, and may not be sighted, identified, or attacked directly.
Instead, it is considered to form part of its transport unit. A mounted unit may not attack other ground units or aircraft, may not launch missiles, and may not act as a FAC.

If a transport unit is sighted, no information is given on its mounted unit or cargo.

If a transport unit is identified, any towed mounted unit is also identified (e.g., ``the truck platoon is towing artillery'' or ``the truck platoon is towing a ZPU-4 battery''), otherwise no information is given on its transported unit or cargo.

If a transport unit suffers damage, then its mounted unit or cargo suffers the same damage.

\paragraph{Dismounted Units.}

At the start of a game, a dismounted unit may be placed according to either its mobility or that of its transport.
Otherwise, a dismounted unit and its transport are treated normally.
A dismounted unit and its transport need not occupy the same hex.%
\note{
    Being placed according to the mobility of its transport allows a tracked transport to place a towed unit in a forest hex.
}

\subsubsection{Advanced Rule}

\paragraph{Mounting and Dismounting Infantry.}

A dismounted infantry unit in the same hex as its specific associated transport may begin to mount by declaring its intention at the start of a Ground Unit Interaction Phase.
At the start of the Ground Unit Interaction Phase five game turns later, the unit is considered to be mounted, and its counter is removed from the map.%
\note{
    The rules for mounting and dismounting are completely new. I'm very open to suggestions about this.
}

A mounted infantry unit may begin to dismount by declaring its intention at the start of a Ground Unit Interaction Phase.
At the start of the Ground Unit Interaction Phase two game turns later, the unit is considered to be dismounted, and its counter is placed on the map in the same hex as its transport. If placing the counter would violate stacking requirements, the mounted unit must abort the dismounting process.

During these processes, the transported unit may not perform any other action and the transport may not move and may not carry out ground attacks, but may perform barrage fire if it has this capability.

The transported unit may abort the process at the end of any Ground Unit Interaction Phase.

\paragraph{Mounting and Dismounting Towed Units.}

A towed unit may not mount or dismount during the game.

\subsection{Composite and Attached Units}

\subsubsection{Advanced Rule}

\paragraph{Composite Platoons.}

At the start of a game, an infantry, infantry weapons, or infantry HQ platoon may \emph{attach} one or two infantry squads to form a \emph{composite} platoon.
The \emph{attached} squads are commonly infantry SAM or FAC squads.
A scenario may add further restrictions on forming composite platoons.%
\note{
    This rule comes from TSOH. It doesn't have counters for infantry SAMs, so in scenarios V-11 and V-22 it says treat infantry platoons as if they have an infantry SAM squad. This rule helps with stacking (since attached squads do not count for stacking) and with keeping your precious FACs and SAM teams anonymous. It also works nicely with the transport rule, since attached squads do not need their own transport but can ride with the platoon.
}

A composite platoon has the same damage resilience, defense strength, target class, mobility, and sighting range as its constituent platoon.

Once formed, a composite platoon may not be split into its constituent units.

\paragraph{Attached Squads.}

An attached squad is not represented by a separate counter, does not count for stacking, and may not be attacked directly.
Instead, it is considered to form part of its composite platoon.
An attached squad retains its normal capabilities (i.e., an attached SAM squad may engage aircraft and an attached FAC squad may sight and mark targets).

If a composite platoon is suppressed, each attached squad is suppressed too.
If a composite platoon suffers damage, then for each D of damage, roll one die for each attached squad and on a \minusafter{3} the attached squad is killed.
If a composite platoon is killed, each attached squad is killed too.
VPs are awarded for damage to any of the constituent units.%
\note{
    The roll of \minusafter{4} seems about right. The probability of killing an attached squad is 40\% for 1D and 64\% for 2D. Those numbers are fairly good approximations to 1/3 and 2/3. If the roll was \minusafter{3}, they would be 30\% and 51\%, and that seems a little low.
}

\clearpage
\section{Ground Combat Units}

Ground combat units represent a combat formations of infantry, tanks, IFVs, APCs, or artillery. Their primary role is combat with other ground units.

\begin{figure}[h]
    \includegraphics[width=\linewidth]{photos/centurion.jpg}
    %\caption*{A Dutch Centurion tank on maneuvers in 1970.}
    % https://commons.wikimedia.org/wiki/File:Nederlandse_leger_oefent_in_La_Courtine,_Frankrijk_oefening,_Bestanddeelnr_923-7334.jpg
\end{figure}

\subsection{Properties}

Table~\ref{table:ground-combat-units} summarizes the properties of ground combat units, and
Figure~\ref{figure:ground-combat-units} shows their representation as counters.
The AAA class is B2 for infantry platoons and B3 for AFV platoons, indicating that they are capable of barrage fire to altitudes of 2 and 3 levels, respectively.

\subsection{Infantry Units}

Infantry are soldiers trained to fight on foot. An infantry, infantry weapons, or infantry HQ platoon consists of thirty to fifty soldiers.
A mechanized infantry platoon will normally be transported by a platoon of APCs or IFVs.
A motorized infantry platoon will normally be transported by a platoon of trucks.
An infantry weapons platoon is equipped with machine guns, ATGMs, grenade launchers, or mortars.%
\note{
    An infantry weapons platoon is not strictly necessary, but it adds to the look-and-feel, and we might consider giving them longer range in ground-to-ground combat. Having a single type is probably more appropriate than having separate types for mortar, grenade launchers, recoilless rifles, ATGMs, and machine guns. One might think that the .50 cal infantry AAA unit could serve as a machine-gun weapons platoon, but I would imagine most machine-gun weapons platoons have low tripods for ground combat rather than high ones for AAA.
}

An infantry FAC squad has a specialized forward air controller with other soldiers providing communication and protection.
Infantry SAM and heavy SAM squads are described in more detail in a subsequent section. Their VP values depend on their type of SAM.
An infantry FAC and SAM squad will often be attached to an infantry, infantry weapons, or infantry HQ platoon or transported by a truck squad.

\subsection{Tank and Anti-Tank Units}

A tank is an armored vehicle that is specialized for direct fire with a gun.
Tanks are distinguished as having heavy, medium, light, or open armor.
An anti-tank vehicle is a lightly armored vehicle equipped with ATGMs.%
\note{
    Like infantry weapons platoons, anti-tank units are mainly chrome.
    However, in ground combat, we might give them more effectiveness against vehicular or towed targets and less against infantry targets.
}
All tank and anti-tank units are platoons.
Examples include:%
\note{
    These are based on the armor classes in the WRG 1950--1985 rules: open is class I; soft is class II to III; and hard is class IX and X.
    More modern tanks are guesses, and I'm willing to be corrected.
}

\begin{itemize}\raggedright
    \item Heavy Tanks: Ariete, Arjun, Chieftain, Challenger 1/2, Conqueror, IS-3, K1, Leclerc, Leopard 2, M1 Abrams, M-84, Merkava, PT-91, Stridsvagn 103, T-10, T-64/80/84, T-72/90, Type 90, and Type 96/99.
    \item Medium Tanks: AMX-30, ASU-85, Centurion,  IS-2, ISU-122, ISU-152, Kanonenjagdpanzer, Leopard 1, M4 Sherman, M26 Pershing, M46/47/48 Patton, M103, M60, Panzer 61, Panzer 68, SU-100, T-34, T-54/55, T-62, TAM, Type 59/69/79, Type 61, Type 74, Type 80/88/96, Vickers MBT, and Vijayanta.
    \item Light Tanks: AMX-10 RC, AMX-13, Ferret, Fox, M24 Chaffee, M41 Walker Bulldog, M551 Sheridan, M1128 Stryker MGS, Panhard AML, Panhard EBR, PT-76, Saladin, Scimitar, and Scorpion.
    \item Open Tanks: ASU-57 and SU-76.%
        \note{
            Open tanks are uncommon, but we need an open target class for M3 half-track APCs, so it is easy to apply it here too.
        }
    \item Light Anti-Tank Vehicles: AMX-10P HOT, BRDM-2/3 variants, FV438 Swingfire, Jaguar 1/2, M150, M901 ITV, M1134 Stryker ATGM, Raketenjagdpanzer 1/2, Striker, and VAB HOT.
\end{itemize}

\subsection{IFV and APC Units}

Both IFVs and APCs are armored vehicles that carry infantry.
However, an IFV provides fire support and typically has a 20~mm or larger cannon, and an APC does not.%
\note{
    And again we might distinguish APCs and IFVs in ground-to-ground combat.
    Bradley IFVs made mincemeat of Iraqi tank regiments; M113 would not have had the same impact.
}
All IFV and APC units are platoons.
They are further distinguished as having heavy, medium, light, or open protection.
Examples include:

\begin{itemize}\raggedright

    \item Heavy IFVs: none.

    \item Medium IFVs: CV90, M2A2 Bradley (only this version), and Marder.

    \item Light IFVs: all others including the AMX-10P, BMP-1/2/3, LAV-25, M2/3 Bradley (except the M2A2 version), M1296 Dragoon, VCBI, and Warrior.

    \item Heavy APCs: converted heavy tanks.
        \note{
            Heavy and medium APCs are not really relevant, but they exist and are easy to include once you've added medium IFVs.
        }

    \item Medium APCs: converted medium tanks.

    \item Open APCs: BTR-40, BTR-152, BTR-50P, BTR-60P, and M3 half-track.

    \item Light APCs: Bronco ATTC, BvS10/Viking, BTR-50PK, BTR-60/70/80, FV432, M113, M114, Saracen, Spartan, most Stryker variants, and VAB.

\end{itemize}

\subsection{Armored Cars and Half-Tracks}

An armored car or an armored half-track is considered to be a light tank, light ant-tank vehicle, light IFV, or a light or open APC according to its role and armament.

\subsection{Artillery Units}

Artillery is specialized in indirect fire. All artillery units are batteries. Artillery may be tube, rocket, or missile and armored, tracked, mobile, or towed. Examples include:

\begin{itemize}\raggedright

    \item Armored Artillery: 2S3 Akatsiya, 2S19 Msta-S, Abbot, AMX-30 AuF1, AS-90, M109, and Panzerhaubitze 2000.
        \note{
            I probably will include self-propelled mortars as armored artillery too. That said, it would only take me five minutes to make the counter with a mortar symbol and add them to the tables.
        }

    \item Tracked Artillery: 2S4 Tyulpan, 2S7 Pion, M7 Priest, M107, and M110.

    \item Mobile Artillery: Archer and CAESAR.

    \item Armored Rocket Artillery: M270 MLRS and TOS-1.

    \item Mobile Rocket Artillery: BM-21 Grad, BM-30 Smerch, and M142 HIMARS.

    \item Tracked Missile Artillery: MGM-52 Lance and Pluton.

    \item Mobile Missile Artillery: 9K52 Luna-M (Frog), OTR-21 Tochka (SS-21), and RSD-10 Pioneer (SS-20).%
        \note{
            Missile artillery is mainly included to serve as a target.
        }

\end{itemize}

A truck platoon may transport a towed artillery battery.

\subsection{HQ Units}

An armored or infantry HQ platoon represents the HQ unit of a battalion or regiment. There are no IFV, APC, or truck HQ units; the HQ platoon of a mechanized or motorized infantry battalion or regiment will consist of either an armored HQ platoon or an infantry HQ platoon associated with a normal IFV, APC, or truck platoon.%
\note{
    I wondered if HQ platoons were also used at the company level, but looking at the Air Strike scenarios they are included at a ratio of about one HQ platoons to nine normal platoons, so they seem to be battalion HQs.
}

\begin{figure*}[p]
    \centering
    \includegraphics[width=0.8\linewidth]{figure-ground-combat-units.png}
    \caption{Ground combat and Transport Units}
    \label{figure:ground-combat-units}
\end{figure*}

\begin{table*}[p]
    \caption{Ground Combat and Transport Units}
    \centering
    \footnotesize
    \begin{tabular}{lcccccc}
\toprule

            Type&
            \vertical{Size}&
            \vertical{Defense Strength}&
            \vertical{Sighting Range}&
            \vertical{Mobility}&
            \begin{tabular}[b]{@{}c@{}}\vertical{VPs}\\\midrule3D/2D/D\\\end{tabular}&
            \vertical{AAA Class}
            \\
        
\midrule
\addlinespace
Infantry &\wbox[l]{Section}{Platoon}&\wbox[l]{0}{4}&\wbox{00}{6}&U&\wbox{00/00/0}{5/3/2}&B2\\
Infantry Weapons &\wbox[l]{Section}{Platoon}&\wbox[l]{0}{4}&\wbox{00}{6}&U&\wbox{00/00/0}{6/4/2}&B2\\
Infantry HQ &\wbox[l]{Section}{Platoon}&\wbox[l]{0}{4}&\wbox{00}{6}&U&\wbox{00/00/0}{10/7/3}&B2\\
Infantry FAC &\wbox[l]{Section}{Squad}&\wbox[l]{0}{5}&\wbox{00}{6}&U&\wbox{00/00/0}{\wbox[c]{0}{--}/\wbox[c]{0}{--}/8}&---\\
Infantry SAM &\wbox[l]{Section}{Squad}&\wbox[l]{0}{5}&\wbox{00}{6}&U&\wbox{00/00/0}{}&---\\
Infantry Heavy SAM &\wbox[l]{Section}{Squad}&\wbox[l]{0}{5}&\wbox{00}{6}&U&\wbox{00/00/0}{}&---\\
\addlinespace
Heavy Tank &\wbox[l]{Section}{Platoon}&\wbox[l]{0}{\underline{8}}&\wbox{00}{12}&U&\wbox{00/00/0}{12/8/4}&B3\\
Medium Tank &\wbox[l]{Section}{Platoon}&\wbox[l]{0}{\underline{6}}&\wbox{00}{12}&U&\wbox{00/00/0}{10/7/3}&B3\\
Light Tank &\wbox[l]{Section}{Platoon}&\wbox[l]{0}{\underline{4}}&\wbox{00}{12}&U&\wbox{00/00/0}{6/4/2}&B3\\
Open Tank &\wbox[l]{Section}{Platoon}&\wbox[l]{0}{\underline{4}!}&\wbox{00}{12}&U&\wbox{00/00/0}{5/3/2}&B3\\
Light Anti-Tank &\wbox[l]{Section}{Platoon}&\wbox[l]{0}{\underline{4}}&\wbox{00}{12}&U&\wbox{00/00/0}{10/7/3}&B3\\
Armored HQ &\wbox[l]{Section}{Platoon}&\wbox[l]{0}{\underline{4}}&\wbox{00}{12}&U&\wbox{00/00/0}{12/8/4}&B3\\
\addlinespace
Medium IFV &\wbox[l]{Section}{Platoon}&\wbox[l]{0}{\underline{6}}&\wbox{00}{12}&U&\wbox{00/00/0}{10/7/3}&B3\\
Light IFV &\wbox[l]{Section}{Platoon}&\wbox[l]{0}{\underline{4}}&\wbox{00}{12}&U&\wbox{00/00/0}{6/4/2}&B3\\
Heavy APC &\wbox[l]{Section}{Platoon}&\wbox[l]{0}{\underline{8}}&\wbox{00}{12}&U&\wbox{00/00/0}{10/7/3}&B3\\
Medium APC &\wbox[l]{Section}{Platoon}&\wbox[l]{0}{\underline{6}}&\wbox{00}{12}&U&\wbox{00/00/0}{8/5/3}&B3\\
Light APC &\wbox[l]{Section}{Platoon}&\wbox[l]{0}{\underline{4}}&\wbox{00}{12}&U&\wbox{00/00/0}{6/4/2}&B3\\
Open APC &\wbox[l]{Section}{Platoon}&\wbox[l]{0}{\underline{4}!}&\wbox{00}{12}&U&\wbox{00/00/0}{5/3/2}&B3\\
\addlinespace
Truck &\wbox[l]{Section}{Platoon}&\wbox[l]{0}{2}&\wbox{00}{12}&G/P&\wbox{00/00/0}{4/3/1}&---\\
Truck &\wbox[l]{Section}{Section}&\wbox[l]{0}{2}&\wbox{00}{12}&G/P&\wbox{00/00/0}{\wbox[c]{0}{--}/3/1}&---\\
Truck &\wbox[l]{Section}{Squad}&\wbox[l]{0}{2}&\wbox{00}{12}&G/P&\wbox{00/00/0}{\wbox[c]{0}{--}/\wbox[c]{0}{--}/1}&---\\
Tracked Carrier &\wbox[l]{Section}{Platoon}&\wbox[l]{0}{2}&\wbox{00}{12}&U&\wbox{00/00/0}{4/3/1}&---\\
\addlinespace
Armored Artillery &\wbox[l]{Section}{Battery}&\wbox[l]{0}{\underline{4}}&\wbox{00}{12}&U&\wbox{00/00/0}{12/8/4}&B3\\
Tracked Artillery &\wbox[l]{Section}{Battery}&\wbox[l]{0}{2}&\wbox{00}{12}&U&\wbox{00/00/0}{10/7/3}&---\\
Mobile Artillery &\wbox[l]{Section}{Battery}&\wbox[l]{0}{2}&\wbox{00}{12}&G/P&\wbox{00/00/0}{9/6/3}&---\\
Towed Artillery &\wbox[l]{Section}{Battery}&\wbox[l]{0}{4}&\wbox{00}{18}&T&\wbox{00/00/0}{8/5/3}&---\\
Armored Rocket Artillery &\wbox[l]{Section}{Battery}&\wbox[l]{0}{\underline{4}}&\wbox{00}{12}&U&\wbox{00/00/0}{12/8/4}&B3\\
Mobile Rocket Artillery &\wbox[l]{Section}{Battery}&\wbox[l]{0}{2}&\wbox{00}{12}&G/P&\wbox{00/00/0}{9/6/3}&---\\
Tracked Missile Artillery &\wbox[l]{Section}{Battery}&\wbox[l]{0}{2}&\wbox{00}{12}&U&\wbox{00/00/0}{10/7/3}&---\\
Mobile Missile Artillery &\wbox[l]{Section}{Battery}&\wbox[l]{0}{2}&\wbox{00}{12}&G/P&\wbox{00/00/0}{9/6/3}&---\\
\addlinespace
\bottomrule
\end{tabular}

    \label{table:ground-combat-units}
\end{table*}

\clearpage
\section{Transport Units}

Transport units provide strategic mobility to combat units and are the life-blood of logistic networks.

\subsection{Properties}

Table~\ref{table:ground-combat-units} summarizes the properties of ground combat units, and
Figure~\ref{figure:ground-combat-units} shows their representation as counters.

\subsection{Trucks and Tracked Carrier Units}

Trucks and tracked carriers are unarmored or incompletely armored vehicles that transport infantry, towed weapons, and supplies.
A truck has good or poor mobility, whereas a tracked carrier has unrestricted mobility.

Trucks include light trucks such as Jeeps, Land Rovers, and Humvees.

Tracked carriers include the M548, Bv~202/206, and BvS10/Beowulf.

\clearpage
\section{AAA Units}

AAA units specialize in the use of guns to destroy attacking aircraft, but also have a secondary ground combat capability.

\subsection{Properties}

Table~\ref{table:aaa-units} summarizes the properties of AAA units, and Figure~\ref{figure:aaa-units} shows their representation as counters.
As well as the properties common to all ground units, Table~\ref{table:aaa-units} gives:%
\note{
    \notesection{AAA Values}

    This long note summarizes how I have determined the AAA values for AAA units. VP values are discussed separately.

    I compare AAA values for AAA units from \emph{Air Strike}, \emph{Eagles of the Gulf}, and \emph{The Speed of Heat} to the actual characteristics of the guns they represent. My aim is to empirically understand the underlying model sufficiently to identify anomalous values and to apply it to other guns. That said, I do comment on the model.

    \notesection{AAA Values for Batteries}

    I start by attempting to understand the AAA values for batteries.

    The upper part of the following table shows the original data for gun batteries from \emph{Air Strike} (AS), \emph{Eagles of the Gulf} (EOTG), and \emph{The Speed of Heat} (TSOH). I have added columns with the caliber, burst rate of fire per mount, and muzzle velocity.

    \begin{center}
        \medskip
        \scriptsize
        \begin{tabular}{lrlllllrrl}
            \toprule
            Type
            &\multicolumn{1}{c}{\vertical{Maximum Altitude}}
            &\multicolumn{1}{c}{\vertical{Ranges}}
            &\multicolumn{1}{c}{\vertical{Hit Rolls}}
            &\multicolumn{1}{c}{\vertical{Damage}}
            &\multicolumn{1}{c}{\vertical{FCR}}
            &\multicolumn{1}{c}{\vertical{Caliber (mm)}}
            &\multicolumn{1}{c}{\vertical{Rate of fire (RPM)}}
            &\multicolumn{1}{c}{\vertical{Muzzle Velocity (m/s)}}
            &\multicolumn{1}{c}{\vertical{Source}}
            \\
            \midrule
            \multicolumn{10}{c}{Original}\\
            \midrule
            M2/DShK-38&5&2/3/4&3/2/1&1&---&12.7&500&890&AS\\
            M55&5&2/3/4&5/4/3&2&---&12.7&2000&890&EOTG\\
            ZPU-1&6&2/3/5&3/2/1&1&---&14.5&600&1000&EOTG/TSOH\\
            ZPU-2&6&2/3/5&4/3/2&1&---&14.5&1200&1000&AS\\
            ZPU-4&6&2/3/5&5/4/3&2&---&14.5&2400&1000&AS\\
            ZPU-4&6&2/3/5&4/3/2&2&---&14.5&2400&1000&TSOH\\
            TCM-20&8&2/4/6&4/4/3&2&---&20&1450&860&EOTG\\
            Single Rh-202&8&2/4/6&3/2/1&2&---&20&1030&1044&AS\\
            Twin Rh-202&8&2/4/6&5/4/3&2&---&20&2060&1044&AS\\
            M167&8&2/4/6&6/5/4&3&R&20&3000&1030&AS\\
            ZU-23&9&2/5/8&4/3/2&2&---&23&2000&980&AS/TSOH\\
            Gemag&10&3/6/9&6/5/4&4&R&25&1800&1040&AS\\
            Oerlikon GDF&12&4/8/10&5/4/3&4&---&35&1100&1175&AS\\
            61-K (M-38)&12&4/8/11&3/2/1&4&---&37&165&880&AS/TSOH\\
            Bofors L/60&15&4/8/12&2/2/1&4&---&40&135&865&EOTG\\
            Bofors L/70&15&4/8/12&3/3/2&4&---&40&330&1000&AS\\
            BOFI&15&4/8/12&4/4/3&4&W&40&330&1000&AS\\
            S-60&18&5/10/15&2/2/1&5&---&57&110&1000&AS/TSOH\\
            KS-12&27&6/12/18&2/1/0&6&---&85&20&790&AS/TSOH\\
            KS-19&39&7/14/21&1/1/0&6&---&100&15&900&AS\\
            \midrule
            \multicolumn{10}{c}{Modified}\\
            \midrule
            ZPU-4&&&5/4/3&&\multicolumn{5}{l}{--- for rate of fire}\\
            Single Rh-202&&&4/3/2&&\multicolumn{5}{l}{--- for rate of fire}\\
            Twin Rh-202&& &3 & &\multicolumn{5}{l}{--- for rate of fire}\\
            ZU-23&&&5/4/3&3&\multicolumn{5}{l}{--- for rate of fire}\\
            Oerlikon GDF&&&4/3/2&&\multicolumn{5}{l}{--- for rate of fire}\\
            BOFI&&&5/5/3&&\multicolumn{5}{l}{--- for FCR-W}\\
            %Bofors L/60&14\\
            \bottomrule
        \end{tabular}
        \medskip
    \end{center}

    Qualitatively, altitude and range increase with caliber, hit rolls increase with rate of fire, and damage rating increases with both caliber and rate of fire.
    This is as expected.

    Quantitatively:
    \begin{itemize}
        \item Approximately doubling the rate of fire increases the hit rolls by one, with typical values of 3/2/1 around 500 rounds per minute, 4/3/2 around 1000, and 5/3/2 around 2000 and above.
        \item The base damage rating is 1 for 12.7--14.5~mm, 2 for 20--23~mm, 3 for 25--30~mm, 4 for 35--40~mm, 5 for 57~mm, and 6 for 85--100~mm.
        \item Damage rating by one is increased by one if the rate of fire is 1800 rounds per minute or more.
    \end{itemize}

    Most of the original values can be used as they are, but a few of the hit rolls need adjusting:
    \begin{itemize}
        \item ZPU-4 and ZU-23:

            There are six AAA batteries with rates of fire per mount of about 2000 rounds per minute or more: the M55, ZPU-4, twin Rh-202, M167, ZU-23, and Gemag. After accounting for the \plus{1} modifier for the FCR-R of the M167 and Gemag, all have hit rolls of 5/4/3 except for the ZPU-4 from \emph{The Speed of Heat} (although the one from \emph{Air Strike} does have 5/4/3) and the ZU-23.

            Furthermore, all of these except for the ZU-23 (both \emph{Air Strike} and \emph{The Speed of Heat} versions) have damage ratings increased by one above their baseline values.

            It seems that either there is some reason to reduce the effectiveness of the ZPU-4 and ZU-23 compared to these other guns, or perhaps they have inconsistent values rolls.

            The ZU-23 is undeniably basic, and one might think that would be reason to reduce its effectiveness. However, the TCM-20 is basic too and has both a lower rate of fire than the ZU-23 (1450 compared to 2000) and a smaller caliber (20~mm compared to 23~mm), yet its hit rolls are higher than the ZU-23 in \emph{The Speed of Heat} and its damage rating is the same.

            The best solution would seem to be to give these guns the values we would expect from comparing them to other guns with similar rate of fire and calibre. Therefore, I give both the ZPU-4 and ZU-23 hit rolls of 5/4/3 and the ZU-23 a damage rating of 3.

            (I also considered the idea that increasing the rate of fire increases \emph{either} the hit rolls \emph{or} the damage rating. However, while this might explain the ZPU-4 from \emph{The Speed of Heat} and the twin Rh-202, it does not correctly predict the values of the M167 or Gemag.)

        \item Single RH-202

            The single Rh-202 has unexpectedly low original hit rolls of 3/2/1 for its rate of fire. The values of 4/3/2 for the ZPU-2 and TCM-20, which have a similar rate of fire, are probably more appropriate.

        \item Twin Rh-202

            The twin has an unexpectedly low original damage rating of 2 for its rate of fire. A value of 3, increased by one from the baseline for the caliber, is probably more appropriate.

        \item Oerlikon GDF

            The Oerlikon GDF has unexpectedly high original hit rolls of 5/4/3 for its rate of fire. The values of 4/3/2 for the ZPU-2, which has a similar rate of fire, are probably more appropriate.

        \item BOFI

            The BOFI has unexpectedly low original hit rolls of 4/4/3. Values of 5/5/3, obtained by applying a \plus{2} modifier for FCR-W to the Bofors L/70, are probably more appropriate. (Note that the “BOFI” in \emph{Air Strike} corresponds to what I call the “BOFI-R” below.)

    \end{itemize}

    I adopt these modified values.

    Also, I note the following:

    \begin{itemize}

        \item FCR-R

            The M167 and Gemag have hit rolls of 6/5/4, and the twin Rh-202 has 5/4/3. Otherwise, they are similar, except that the M167 and Gemag have FCR-R, and the Rh-202 does not (and the M167 has a moderate advantage in rate of fire). I consider the FCR-R of the M167 and Gemag to be largely responsible for increasing their hit rolls by \plus{1} compared to the Rh-202.

        \item Bofors

            The hit rolls of 2/2/1 for the Bofors L/60, 3/2/1 for the 61-K (M-38), and 3/3/2 for the Bofors L/70 seem to reflect the increase in rate of fire.

            The L/60 and L/70 have the same maximum altitude of 16 and maximum range of 12, despite the L/60 having lower muzzle velocity. However, I am not especially motivated to attempt to change this, since the Bofors is primarily a weapon for point-defense against aircraft up to about level 10.

            The real BOFI and BOFI-R also have proximity fuzes, and I consider these will increase their hit rolls by \plus{1}.

    \end{itemize}

    \notesection{AAA Values for Sections}

    I now turn to sections. Again, my aim is to empirically understand the underlying model sufficiently to identify anomalous values and to apply it to other guns.

    The upper part of the following table shows the original data for gun sections from \emph{Air Strike} (AS), \emph{Eagles of the Gulf} (EOTG), and \emph{The Speed of Heat} (TSOH). Again, I have added columns with the caliber, burst rate of fire per mount, and muzzle velocity.

    \begin{center}
        \medskip
        \scriptsize
        \begin{tabular}{lrlllllrrl}
            \toprule
            Type
            &\multicolumn{1}{c}{\vertical{Maximum Altitude}}
            &\multicolumn{1}{c}{\vertical{Ranges}}
            &\multicolumn{1}{c}{\vertical{Hit Rolls}}
            &\multicolumn{1}{c}{\vertical{Damage}}
            &\multicolumn{1}{c}{\vertical{FCR}}
            &\multicolumn{1}{c}{\vertical{Caliber (mm)}}
            &\multicolumn{1}{c}{\vertical{Rate of fire (RPM)}}
            &\multicolumn{1}{c}{\vertical{Muzzle Velocity (m/s)}}
            &\multicolumn{1}{c}{\vertical{Source}}
            \\
            \midrule
            \multicolumn{10}{c}{Original}\\
            \midrule
            M16&5&2/3/4&4/3/2&2&---&12.7&2000&890&EOTG\\
            Mobile ZPU-4&6&2/3/5&3/3/2&2&---&14.5&2400&1000&AS\\
            M163&8&2/4/6&5/4/3&3&R&20&3000&1030&AS\\
            M3 TCM-20&8&2/4/6&3/3/2&2&---&20&1450&860&EOTG\\
            Panhard&8&2/4/6&2/2/1&2&---&20&1450&860&AS\\
            Mobile ZU-23&9&2/5/8&2/2/1&2&---&23&2000&980&AS\\
            Nile 23&9&2/5/8&3/3/2&2&---&23&2000&980&EOTG\\
            ZSU-23-4&9&2/5/8&5/4/3&3&W&23&4000&980&AS\\
            Blazer&10&3/6/9&5/4/3&4&W&25&1800&1040&AS\\
            M53/59&11&4/7/9&4/3/2&3&---&30&1000&1000&AS\\
            AMX-30&11&4/7/9&5/4/3&3&W&30&1200&970&AS\\
            ZSU-30-2&11&4/7/9&5/4/3&4&W&30&5000&960&AS\\
            Gepard&12&4/8/10&5/4/3&4&W&35&1100&1175&AS\\
            ZSU-57-2&18&5/10/15&2/1/1&5&---&57&240&1000&EOTG\\
            \midrule
            \multicolumn{10}{c}{Modified}\\
            \midrule
            Mobile ZPU-4&&&4/3/2&&\multicolumn{5}{l}{--- to match the ZPU-4}\\
            Panhard&&&3/3/2&&\multicolumn{5}{l}{--- to match the M3 TCM-20}\\
            Mobile ZU-23&&&4/3/2&3&\multicolumn{5}{l}{--- to match the ZU-23}\\
            Nile 23&&&4/3/2&3&\multicolumn{5}{l}{--- to match the ZU-23}\\
            M53/59&&&3/2/1&&\multicolumn{5}{l}{--- to match the AMX-30}\\
            \bottomrule
        \end{tabular}
        \medskip
    \end{center}

    Comparing similar weapons, batteries and sections have the same maximum altitudes, ranges, and damage ratings.
    However, batteries (with four to six mounts) typically have a \plus{1} advantage on their hit rolls compared to sections (with two mounts).
    In a very few cases, the difference is 0, and in a very few others, it is \plus{2}.
    A difference of 1 in the hit rolls is commensurate with the approximate doubling of the rate of fire from a section to a battery.

    To determine the appropriate AAA for sections, the following recipe seems to work in most cases:
    \begin{itemize}
        \item Take the value for an equivalent battery without FCR and apply an adjustment of \minus{1} to the hit rolls to account for the reduced rate of fire in the section. (Only do this if the rate of fire is reduced; see the comment on the ZSU-23-4 and ZSU-57-2 below.)
        \item Apply an adjustment of \plus{1} to the hit rolls for FCR-R or similar.
        \item Apply an adjustment of \plus{2} to the hit rolls for FCR-W.
        \item The preceding steps would then predict hit rolls of 6/5/4 for the Blazer and 7/6/5 for the ZSU-23-4 and ZSU-30-2. To conform to the original values, I limit the hit rolls to 5/4/3. It would appear that in the underlying model, beyond a certain rate of fire, other factors take over to limit the probability of a hit.
    \end{itemize}

    % One problem is that there are no 30~mm batteries, so data for 30~mm sections needs to be determined by comparison to smaller and larger calibers.

    Again, most of the original data can be used as they are, but a few of the values are worth commenting upon or need adjusting.
    \begin{itemize}
        \item Agreements

            The original hit rolls and the predicted hit rolls are in agreement for the M16 (from the M55 battery), M3 TCM-20 (from the TCM-20 battery), M163 (from the M167 battery), the Blazer (from the Gemag), and the Gepard (from the Oerlikon GDF).

        \item Mobile ZPU-4

            The original hit rolls for the mobile ZPU-4 section are 3/3/2, but 4/3/2 is predicted from the values of 5/4/3 for the battery and is in better agreement with the values for the M16.

        \item Panhard M3 DCA

            The Panhard M3 DCA has unexpectedly low original hit rolls of 2/2/1 compared to the TCM-20 battery and the M3 TCM-20 section, despite having very similar guns. Values of 3/3/2 are probably more appropriate.

        \item Mobile ZU-23 and Nile 23

            The original hit rolls for the mobile ZU-23 and Nile 23 section are 2/2/1 and 3/3/2, despite them being quite similar. The predicted values of 4/3/2 are probably more appropriate and are in better agreement with the values for the M16. Also, as with the ZU-23 battery, a damage rating of 3 is probably more appropriate given their high rate of fire.

        \item M53/59

            The original hit rolls for the M53/59 section are 4/3/2. The predicted values of 3/2/1, derived from the modified hit rolls of 4/3/2 for the Oerlikon GDF battery and also from the AMX-30SA after discounting the FCR-W, are probably more appropriate.

        \item ZSU-57-2

            The ZSU-57-2 section has the same hit rolls as the S-60 battery. This is probably appropriate, as the ZSU-57-2 is a twin mount and this largely compensates for the reduced rate of fire of the section. The same applies to the quad ZSU-23-4 compared to the twin ZU-23.
    \end{itemize}

    I adopt these modified values.

    Also, I note:

    \begin{itemize}

        \item Panhard M3 DCA

            The Panhard M3 DCA has an FCR-R capability, which is discussed below but not considered here.

        \item AMX-30 DCA

            The real AMX-30 DCA does not have an FCR. The AMX-30 DCA in the game seems to correspond to the similar AMX-30SA, which does.

    \end{itemize}

    \notesection{Summary}

    I have developed an empirical understanding of the model for AAA values.

    Consideration of outliers suggests:
    \begin{itemize}
        \item Increasing in the hit rolls of the mounted ZPU-4, single Rh-202, Panhard M3 VDA, ZU-23 (and derivatives), and BOFI.
        \item Decreasing the hit rolls of the Oerlikon GDF.
        \item Increasing the damage rating of the twin Rh-202 and the ZU-23 guns(and derivatives).
    \end{itemize}
    I have adopted these changes.

}
\clearnotes

\begin{itemize}
    \item the AAA class (light, medium, or heavy);
    \item the maximum altitude;
    \item the limits of short, medium, and long range in hexes;
    \item the hit rolls at short, medium, and long range;
    \item the damage rating;
    \item whether the unit has an all-weather or ranging-only fire-control radar and its frequency band;
    \item whether it has night infrared sights; and
    \item whether it is also equipped with a SAM system.
\end{itemize}

If the unit has a SAM, this will be described in detail in the section on SAM launcher units.

When comparing these properties, remember that towed batteries have six to eight gun mounts, and armored or mobile sections have only two gun mounts.

\subsection{FCRs}

Some AAA units have integrated FCRs that give them an all-weather and night capability.
Most medium and heavy can also use add-on FCRs, described below.
Units without an FCR (or with a range-only FCR) are limited to visually sighted targets.

\subsection{Light AAA Units}

\subsubsection{M2 Platoon}

The Browning M2 .50~cal (12.7~mm) heavy machine gun is normally found on a low tripod for ground combat, but here is used on a tall dual-purpose mount suitable for air defense. A platoon has six guns.%
\note{
    For the M2 platoon: The AAA values are from \emph{Air Strike}.
}

The M2 gun was introduced in 1933 and has been used by United States forces and their allies in numerous conflicts since then.

\subsubsection{DShK-38 Platoon}

The DShK-38 12.7~mm heavy machine gun is most commonly found on a low, wheeled mount for ground combat, but here is used on a tall tripod suitable for air defense. A platoon has six guns.%
\note{
    For the DShk-38 platoon: The AAA values are from \emph{Air Strike}.
}

The DShK-38 entered service with the Soviet Army in 1938 and saw action in World War II and numerous conflicts since then, including the Korean and Vietnam Wars, in which the Soviet Union participated or provided arms.

\subsubsection{M16 Section}

The M16 Multiple Gun Motor Carriage has an M45 turret with four M2 .50 cal (12.7~mm) heavy machine guns mounted on a modified M3 half-track. A section has two vehicles.%
\note{
    For the M16 section: The AAA values are from \emph{Eagles of the Gulf}.
}

The M16 was used by United States forces during World War II and (primarily as an anti-personnel weapon) in the Korean War for airfield defense. Additionally, Israeli forces employed it during the 1956 Arab-Israeli War.

\subsubsection{M55 Platoon}

The M55 has an M45 turret with four M2 .50 cal (12.7~mm) heavy machine guns mounted on a towed carriage. A platoon has four gun mounts.%
\note{
    For the M55 battery: The AAA values are from \emph{Eagles of the Gulf}.
}

The M55 was used by United States forces during the Korean War and by Pakistani forces during the 1965 India-Pakistan War.

%\paragraph{Mobile M55.} During the Vietnam War, the US forces mounted the M55 gun turret on a truck and used it in an anti-personnel role.

\subsubsection{ZPU-1/2/4 Battery}

The ZPU-1/2/4 have single, twin, or quad 14.5~mm heavy machine guns in open mounts on towed carriages. A battery has six gun mounts.%
\note{
    For the ZPU-1 battery: The AAA values are from \emph{Eagles of the Gulf} and \emph{The Speed of Heat}.
}%
\note{
    For the ZPU-2 battery: The AAA values are from \emph{Air Strike}.
}%
\note{
    For the ZPU-4 battery: The AAA values are from \emph{Air Strike} rather than \emph{The Speed of Heat}, for the reasons described in the extended note above.
}

The ZPU-1/2/4 was introduced in 1949 and saw extensive service with Soviet or Soviet-supported forces in many subsequent conflicts, including with communist forces in the Korean War and Vietnam War and with Arab forces in the various Arab-Israeli Wars from 1956 onwards.

\subsubsection{Mobile ZPU-1/2/4 Section}

This is a section of two light trucks mounting ZPU-1/2/4 guns.%
\note{
    For the mobile ZPU-4 section:
    AAA values are from \emph{Air Strike}, except that I have adopted hit rolls of 4/3/2 rather than 3/3/2 as described in the extended note above.
    %\emph{Air Strike} gives this 5 VPs, which is the same as the towed ZPU-4 battery despite the mobile section having worse hit rolls and defense strength. I think 4 VPs (i.e., one less than the battery) is more appropriate.
}%
\note{
    For the mobile ZPU-1 and ZPU-2 section: These do not appear in \emph{Air Strike}, \emph{Eagles of the Gulf}, or \emph{The Speed of Heat}, but they are very common. As described in the extended note above, I have produced AAA hit values for the sections from those of the ZPU-1 and ZPU-2 batteries.
}

Many users of the ZPU-1/2/4 mount it on a suitable truck for increased mobility. Like the towed versions, it has seen combat in many conflicts.

\subsubsection{M167 Platoon}

The M167 Vulcan Air Defense System (VADS) has a 20~mm six-barreled Vulcan gun in an open mount on a towed carriage. The fire-control system has optical sights and a range-only radar. A platoon has four gun mounts.%
\note{
    For the M167 platoon: The AAA values are from \emph{Air Strike}. I'm not sure when it gained FCR-R or if it gained night IR sights.
}

The M167 was introduced into service with the US Army in 1967, principally for airfield defense, and subsequently served with other US allies.

\subsubsection{M163 Section}

The M163 Vulcan Air Defense System (VADS) has a 20~mm six-barreled Vulcan gun in a lightly armored but open-topped turret on an adapted M113 APC. The fire-control system has optical sights and a range-only radar. A section has two vehicles.\note{
    For the M163 section: The AAA values are from \emph{Air Strike}. I have the same uncertainties as I do for the M167.
}

The M163 VADS was introduced into service with the US Army in 1968 and has since been adopted by other US allies. The M163 was used by US forces in the Vietnam War, the 1989 Invasion of Panama, and the 1991 Gulf War. It was also used by the Israel Army in the 1982 Lebanon War and in 2002 during the Second Intifada.

From 1988 in the US Army, each M163 was typically issued with a FIM-92 Stinger launcher with two rounds, so each M163 section effectively carried a mounted FIM-92 section.

\subsubsection{TCM-20 Platoon}

The TCM-20 is an Israeli adaptation of the M55, with two Hispano-Suiza 20mm guns from obsolete aircraft replacing the four .50 cal machine guns. As with the M55, the fire-control system is entirely optical. A platoon has four gun mounts.%
\note{
    For the TCM-20 platoon: The AAA values are from \emph{Eagles of the Gulf}.
}

The TCM-20 was employed by Israeli forces from 1970 and subsequently saw combat in the 1973 Arab-Israeli War and the 1982 Lebanon War.

\subsubsection{M3 TCM-20 Section}

The M3 TCM-20 has a TCM-20 turret mounted in an adapted M3 half-track, like the earlier M16. A section has two vehicles.%
\note{
    For the M3 TCM-20 section: The AAA values are from \emph{Eagles of the Gulf}.
}

The M3 TCM-20 was used by Israeli forces in the 1973 Arab-Israeli War and the 1982 Lebanon War.

\subsubsection{Rh-202 Battery}

The Rheinmetall Rh-202 has a single or twin 20mm gun in an open mount on a towed carriage. The fire-control system is entirely optical. The most common version is the twin mount. A battery has six gun mounts.%
\note{
    For the single and twin Rh-202 batteries: All AAA mounts seem to be twin. The AAA values are from \emph{Air Strike}, with the hit values for the single mount changed from 3/2/1 to 4/3/2 and the damage value for the twin mount changed from 2 to 3, as described in the extended note above.
}

The twin version entered service with West German forces in 1972 and was used for airfield defense. Subsequently, it was exported and notably used by the Argentine Air Force in the 1982 South Atlantic War in the defense of BAM Malvinas (Port Stanley Airport) and BAM Condor (Goose Green Airfield).

\subsubsection{Panhard M3 DCA Section}

The Panhard M3 DCA vehicles have an armored turret for twin 20~mm guns mounted on the Panhard M3 APC. The guns have optical sights, but typically one vehicle in each section or platoon is equipped with a search-and-ranging radar and can share this information automatically with the others. A section has two vehicles.%
\note{
    For the Panhard M3 DCA section:
    The AAA data are from \emph{Air Strike}, but modified as described in the extended note above to give hit rolls of 3/3/2 without FCR. However, the Panhard M3 DCA radar-equipped vehicle seems to be able to search and share speed and range information with the other vehicle in the section, which seems equivalent to FCR-R. Therefore, I have increased the hit rolls by one to give 4/4/3 with FCR-R.
}

The Panhard M3 DCA served with the armed forces of Côte d'Ivoire, Niger, and the UAE.

\subsubsection{ZU-23 Battery}

The ZU-23 has twin 23~mm cannons in an open mount on a towed carriage. It was designed as a more powerful replacement for the ZPU-1/2/4. The fire-control system is entirely optical. A battery has six gun mounts.%
\note{
    For the ZU-23 battery: The AAA values are from \emph{Air Strike}, but the hit rolls are modified as from 4/3/2 to 5/3/2 and the damage rating from 2 to 3 as described in the extended note above.
    %TSOH gives it 6 VPs and Air Strike gives it only 5. I think 6 is more appropriate, comparing to the ZPU-4 battery.
}

The ZU-23 was introduced into service by the Soviet Army in 1960. It later served with the armed forces of many Soviet allies, including with North Vietnam in the Vietnam War, with Egypt and Syria in the 1967 and 1973 Wars, with Soviet forces in the Soviet-Afghan War, with Iraq in the Iraq-Iran War, with Syria in the 1982 Lebanon War, and with both sides in the Afghan Civil War.

\subsubsection{Mobile ZU-23 Section}

This is a section of two light trucks mounting ZU-23 guns.%
\note{
    For the mobile ZU-23 section: The AAA values are from \emph{Air Strike} and \emph{The Speed of Heat}, but the hit rolls of 4/3/2 were obtained from those for the ZU-23 battery using the recipe described above, and the damage was modified from 2 to 3 as described in the extended note above.
    %The VPs seem about right, as the battery has 6 VPs. Air Strike gives 5 VPs.
}

As with the earlier ZPU-1/2/4, many users of the ZU-23 mount it on vehicles, including trucks, tracked carriers, and APCs, for better mobility.

\subsubsection{Sinai 23 Section}

The Sinai 23 has a Dassault TA20 turret with two 23~mm guns and six Ayn-al-Saqr (similar to the Strela-2M) or Stinger missiles on an M113 chassis. It has night IR sights. A section has two vehicles. Each battery operates with one vehicle equipped with an RA20S search radar.%
\note{
    For the Sinai 23 section: The turret seems to be similar to the LAV-AD turret, but with a pair of 23~mm guns instead of a GAU-12 25~mm gun. The AAA values were taken from the mobile ZU-23 battery. I don't know which model of Stinger was used; I would guess the FIM-92A.
}

The Sinai 23 entered service with the Egyptian Army in 1991.

\subsubsection{ZSU-23-4 Section}

The ZSU-23-4 has four radar-controlled 23~mm guns, similar to those in the ZU-23, mounted in an armored turret on a hull adapted from the PT-76 light tank. A section has two vehicles.%
\note{
    For the ZSU-23-4 section: The AAA values are from \emph{Air Strike}.
}

The ZSU-23-4 entered service in the Soviet Army in 1965, replacing the ZSU-57-2 and outclassing all NATO anti-aircraft guns at that time. A platoon of four was assigned to the air-defense battery of tank and motor-rifle regiments and later complemented by a platoon of four SA-9 vehicles. It was exported to Soviet allies and saw extensive combat with Arab forces in the 1973 Arab-Israeli War and with Iraqi forces in the Iran-Iraq War and the 1991 Gulf War.

\subsubsection{LAV-AD Section}

The LAV-AD has a 25~mm GAU-12 rotary cannon and eight FIM-92D Stinger missiles mounted in a turret on the LAV-25 vehicle. It does not have radar, but is equipped with a passive night IR sight. A section has two vehicles.%
\note{
    For the LAV-AD section: The system seems similar to the Blazer, but without radar and with night IR sights. I've adopted the range and damage rating for the Gemag battery in \emph{Air Strike} with the hit rolls reduced by one for being a section and one again for not having FCR-R giving 4/3/2.
}

The LAV-AD entered service with the USMC in 1997, but was retired after only a few years.

\subsubsection{M53/59 Section}

The M53/59 has a twin M53 30~mm gun mount on a partially armored truck. The fire-control system is entirely optical. A section has two vehicles.%
\note{
    For the M53/59 section: The AAA values are from \emph{Air Strike}, but the hit rolls were reduced from 4/3/2 to 3/2/1 as described in the extended note above.
    If the VADS is classed as armored, should this be armored too?
}

The M53/59 entered service in the Czech Army in 1959 and was used instead of the ZSU-57-2. It was also exported to Iraq and saw combat in the Iran-Iraq War and the 1991 Gulf War.

\subsubsection{AMX-30SA Section}

The AMX-30SA is equipped with a pair of radar-controlled 30~mm Hispano-Suiza guns mounted in an armored turret on the hull of an AMX-30 tank. It is an improved version of the AMX-30 DCA. A section has two vehicles.%
\note{
    For the AMX-30SA section: I adopt the Air Strike values given for the AMX-30 DCA. The real DCA does not have radar, but the Air Strike one does, matching the SA.
}

The AMX-30SA served only in the Saudi Army from 1979.

\subsubsection{Tunguska Section}

The Tunguska has twin radar-controlled 30mm guns and four SA-19 missiles in an armored turret on an armored and tracked hull. It was designed to counter the A-10 and AH-64, against which the earlier ZSU-23-4 had weaknesses. A section has two vehicles.%
\note{
    For the Tunguska section: The AAA values are from \emph{Air Strike} for the ZSU-30-2. %The VP values are tentative, and might be modified once other SAM launchers are considered.
}

The Tunguska entered service in the Soviet Army in 1984. It has been used by Russia, Ukraine, and Belarus after the break-up of the Soviet Union and has been exported.

\subsubsection{Tunguska-M Section}

The Tunguska-M is an improved version of the Tunguska with eight missiles instead of four. A section has two vehicles.%
\note{
    For the Tunguska-M section: The AAA values are from \emph{Air Strike} for the ZSU-30-2. %The VP values are tentative, and might be modified once other SAM launchers are considered.
}

The Tunguska-M entered service in the Soviet Army in 1990. It has been used by Russia, Ukraine, and Belarus after the break-up of the Soviet Union and has been exported.

%\subsubsection{Pantsir.} The Pantsir has twin radar-controlled 30~mm guns and twelve SA-22 missiles in an unarmored turret mounted on a truck. While the Tunguska was designed to accompany ground forces, the Pantsir was designed for point defense of mobile S-300/S-400 batteries and other high-value targets, and this led to the choice of the system being wheeled and unarmored. It entered service in 2012 with Russian forces and has since been exported.

\subsection{Medium AAA Units}

\subsubsection{Oerlikon GDF Battery and Section}

The Oerlikon GDF has two 35~mm guns in an open mount on a towed carriage.
A battery has six gun mounts and a section has two.
The basic fire-control system is optical, but a section is often coupled with the Super Fledermaus or Skyguard fire-control radars.
(These radars cannot control a whole battery, only a section.)%
\note{
    For the Oerlikon GDF battery and section: The AAA values for the battery are from \emph{Air Strike} with the hit rolls reduced from 5/4/3 to 4/3/2 as described in the extended note above.
    The AAA values for the section are determined using the recipe for sections described in the extended note above.
}

The Oerlikon GDF entered service in 1963 and has served widely. Notably, it saw combat in the South Atlantic War with the Argentine forces in the defense of BAM Malvinas (Port Stanley Airport) and BAM Condor (Goose Green Airfield).

\subsubsection{Gepard Section}

The Gepard has two radar-controlled 35~mm guns in an armored turret on an adapted Leopard 1 tank hull. A section has two vehicles.%
\note{
    For the Gepard section: The AAA values are from \emph{Air Strike}.
}

The Gepard entered service in 1976 with the West German Army and also served with the Belgian and Dutch armies. After the end of the Cold War, they were retired by these armies, and many were sold on to other countries.

\subsubsection{61-K Battery}

The 61-K (M1939) has a single 37~mm gun in an open mount on a towed carriage. The fire-control system is entirely optical, and the 61-K cannot be coupled to an external fire-control radar. A battery has six gun mounts.%
\note{
    For the 61-K battery: The AAA values are from \emph{Air Strike} and \emph{The Speed of Heat}.
    % but TSOH gives it 6 VPs and Air Strike only 5.
}

After entering service in 1939 with the Soviet Army, it served in WW2 and after until replaced by the S-60 57~mm gun. It also served with many Soviet allies, and in particular with communist forces in the Korean and Vietnam Wars.

The 61-K is referred to as the “M-38” in \emph{Air Strike}, \emph{Eagles of the Gulf}, and \emph{The Speed of Heat}.%
\note{
    I can find no sources other than \emph{Air Strike}, \emph{Eagles of the Gulf}, and \emph{The Speed of Heat} that refer to the 61-K 37~mm gun as the “M-38”. Another designation for the gun is “M-1939”, so I could understand “M-39“.
}

\subsubsection{Bofors L/60 Battery}

The Bofors L/60 has a single 40~mm gun in an open mount on a towed carriage. The fire-control system is entirely optical. A battery has six gun mounts.%
\note{
    For the Bofors L/60 battery: The AAA values are from \emph{Eagles of the Gulf}.
}

It entered service in the 1940s and was widely used by Allied forces during WW2, both on land and sea. After the war, many continued to see service, although it was increasingly inadequate against faster jet aircraft and in many cases was replaced by the new L/70 model.

\subsubsection{Bofors L/70 Battery}

The Bofors L/70 has a single 40~mm gun in an open mount on a towed carriage. Between the 1930s and the end of WW2, aircraft speed increased substantially, and so the engagement time of the Bofors L/60 became inadequate. The new L/70 fired a lighter shell at a higher velocity and achieved a significant improvement in range and engagement time. The basic fire-control system is optical, but various external fire-control radars can be used with it, including the Super Fledermaus and Flycatcher systems. A battery has six gun mounts.%
\note{
    For the Bofors L/70 battery: The AAA values are from \emph{Air Strike}.
}

NATO forces and others adopted it widely starting in 1952.

\subsubsection{Bofors L/70 BOFI and BOFI-R Battery}

The BOFI version of the Bofors L/70 adds modern sights, a laser rangefinder, and proximity-fuzed shells. The BOFI-R version also adds an integrated fire-control radar. A battery has six gun mounts.%
\note{
    For the BOFI and BOFI-R battery: The AAA values are adapted from \emph{Air Strike} values for the Bofors L/70, with the hit rolls increased by one for proximity fuzes, one for the BOFI laser rangefinder (effectively FCR-R), and another one for the BOFI-R radar, giving hit rolls of 5/5/4 for BOFI and 6/6/5 for BOFI-R whereas \emph{Air Strike} gives the BOFI-R 5/5/4.
}

The BOFI is used by the armed forces of Brazil.%
\note{
    I don't know when the BOFI and BOFI-R became available or who else uses them.
}

\subsubsection{S-60 Battery}

The S-60 is a Soviet single 57~mm gun on an open mount on a towed carriage. The basic fire-control system is optical, but the gun is often used with the SON-9 (Fire Can) radar. A battery has six gun mounts.%
\note{
    For the S-60 battery: The AAA values are from \emph{Air Strike} and \emph{The Speed of Heat}.
    % These are the same as in Air Strike, except that TSOH gives 8 VPs and Air Strike only 6.
}

It entered service in 1950 with Soviet forces as a replacement for the 61-K 37~mm gun. It was exported to many Soviet allies and saw combat with Arab forces in the 1967 and 1973 Arab-Israeli War, with communist forces in the Vietnam War, and with Iraqi forces in the Iran-Iraq War and the Gulf War. However, it was not used by communist forces in the Korean War. When paired with the SON-9 (Fire Can) radar, it was considered one of the more dangerous AAA systems faced by US aircraft in Vietnam.

\subsubsection{ZSU-57-2 Section.}

The ZSU-57-2 has two 57~mm cannons in a lightly armored but open-topped turret on an armored and tracked hull. The cannons are similar to the S-60 cannons. The fire-control system is optical. A section has two vehicles.%
\note{
    For the ZSU-57-2 section: The AAA values are from \emph{Eagles of the Gulf}.
}

The ZSU-57-2 entered service in the Soviet Army in 1955 and replaced the BTR-40A and BTR-152A self-propelled 14.5~mm AA guns in the AA batteries of tank regiments. It was in turn replaced by the ZSU-23-4 from 1965. The ZSU-57-2 also served widely with Soviet allies and saw combat with Egypt and Syria in the 1967 and 1973 Wars, with Syria in the 1982 Lebanon War, with North Vietnam in the 1972 Easter Offensive and the 1975 Spring Offensive, and with various factions in the Yugoslav Wars in the 1990s.

\subsection{Heavy AAA Units}

\subsubsection{KS-12 Battery}

The KS-12 (M1939 52-K) and the very similar KS-12A (M1944 KS-1) have a single 85~mm gun in an open mount on a towed carriage. The fire-control system is optical, but the guns are often paired with the SON-9 (Fire Can) radar. A battery has six gun mounts.%
\note{
    For the KS-12 battery: The AAA values are from \emph{Air Strike} and \emph{The Speed of Heat}.
}

It entered service with Soviet forces in 1939. After WW2, it began to be replaced in Soviet service by the KS-19 100~mm and KS-30 130~mm guns, but was widely used by Soviet allies and saw combat with communist forces in the Korean War and the Vietnam War.

\subsubsection{KS-19 Battery}

The KS-19 has a single 100~mm gun in an open mount on a towed carriage. The fire-control system is optical, but like other contemporary Soviet guns, it was often paired with the SON-9 (Fire Can) radar. A battery has six gun mounts.%
\note{
    For the KS-19 battery: The AAA values are from \emph{Air Strike}.
}

It entered service in 1948 with Soviet forces as a replacement for the KS-12 85~mm gun. It also served with many Soviet allies and saw combat with communist forces in the Korean War and Vietnam War and Iraqi forces in the Iran-Iraq and Gulf Wars.

\begin{figure*}[p]
    \centering
    \includegraphics[width=0.8\linewidth]{figure-aaa.png}
    \caption{Air-Defense Units: AAA Units}
    \label{figure:aaa-units}
\end{figure*}

\begin{table*}[p]
    \caption{Air-Defense Units: AAA Units}
    \centering
    \footnotesize
    \begin{tabular}{lcccccclccc@{~}c@{~}cc@{~}c@{~}ccccl}
\toprule

            Type&
            \vertical{Size}&
            \vertical{Year}&
            \vertical{Defense Strength}&
            \vertical{Sighting Range}&
            \vertical{Mobility}&
            \begin{tabular}[b]{@{}c@{}}\vertical{VPs}\\\midrule 3D/2D/D\\\end{tabular}&
            \wbox[r]{00}{\vertical{Gun}}&
            \vertical{AAA Class}&
            \multirow[b]{-3}{*}{\vertical{Altitude}}&
            \multicolumn{3}{c}{\begin{tabular}[b]{@{}ccc@{}}\multicolumn{3}{@{}c@{}}{\vertical{Range}}\\\midrule\wbox[c]{00}{S}&\wbox[c]{00}{M}&\wbox[c]{00}{L}\end{tabular}}&
            \multicolumn{3}{c}{\begin{tabular}[b]{@{}ccc@{}}\multicolumn{3}{@{}c@{}}{\vertical{Hit}}\\\midrule\wbox[c]{00}{S}&\wbox[c]{00}{M}&\wbox[c]{00}{L}\end{tabular}}&
            \vertical{Damage Rating}&
            \vertical{FCR}&
            \vertical{Night IR Sights}&
            \wbox[r]{00}{\vertical{SAM}}\\
        
\midrule
\addlinespace
M2 &\wbox[l]{Section}{Platoon}&1933&\wbox[l]{0}{3}&\wbox{00}{12}&U&\wbox{00/00/0}{3/2/1}&12.7 mm&L&\wbox{00}{5}&\wbox{00}{2}&\wbox{00}{3}&\wbox{00}{4}&\wbox{00}{3}&\wbox{00}{2}&\wbox{00}{1}&\wbox{0}{1}&---&---&---\\
DShK-38 &\wbox[l]{Section}{Platoon}&1938&\wbox[l]{0}{3}&\wbox{00}{12}&U&\wbox{00/00/0}{3/2/1}&12.7 mm&L&\wbox{00}{5}&\wbox{00}{2}&\wbox{00}{3}&\wbox{00}{4}&\wbox{00}{3}&\wbox{00}{2}&\wbox{00}{1}&\wbox{0}{1}&---&---&---\\
M16 &\wbox[l]{Section}{Section}&1944&\wbox[l]{0}{\underline{3}!}&\wbox{00}{12}&U&\wbox{00/00/0}{\wbox[c]{0}{--}/6/3}&\binarymultiply{12.7 mm}{4}&L&\wbox{00}{5}&\wbox{00}{2}&\wbox{00}{3}&\wbox{00}{4}&\wbox{00}{4}&\wbox{00}{3}&\wbox{00}{2}&\wbox{0}{2}&---&---&---\\
M55 &\wbox[l]{Section}{Platoon}&1945&\wbox[l]{0}{3}&\wbox{00}{12}&T&\wbox{00/00/0}{4/3/1}&\binarymultiply{12.7 mm}{4}&L&\wbox{00}{5}&\wbox{00}{2}&\wbox{00}{3}&\wbox{00}{4}&\wbox{00}{5}&\wbox{00}{4}&\wbox{00}{3}&\wbox{0}{2}&---&---&---\\
\addlinespace
ZPU-1 &\wbox[l]{Section}{Battery}&1949&\wbox[l]{0}{3}&\wbox{00}{12}&T&\wbox{00/00/0}{3/2/1}&14.5 mm&L&\wbox{00}{6}&\wbox{00}{2}&\wbox{00}{3}&\wbox{00}{5}&\wbox{00}{3}&\wbox{00}{2}&\wbox{00}{1}&\wbox{0}{1}&---&---&---\\
ZPU-2 &\wbox[l]{Section}{Battery}&1949&\wbox[l]{0}{3}&\wbox{00}{12}&T&\wbox{00/00/0}{4/3/1}&\binarymultiply{14.5 mm}{2}&L&\wbox{00}{6}&\wbox{00}{2}&\wbox{00}{3}&\wbox{00}{5}&\wbox{00}{4}&\wbox{00}{3}&\wbox{00}{2}&\wbox{0}{1}&---&---&---\\
ZPU-4 &\wbox[l]{Section}{Battery}&1949&\wbox[l]{0}{3}&\wbox{00}{12}&T&\wbox{00/00/0}{5/3/2}&\binarymultiply{14.5 mm}{4}&L&\wbox{00}{6}&\wbox{00}{2}&\wbox{00}{3}&\wbox{00}{5}&\wbox{00}{5}&\wbox{00}{4}&\wbox{00}{3}&\wbox{0}{2}&---&---&---\\
Mobile ZPU-1 &\wbox[l]{Section}{Section}&1949&\wbox[l]{0}{2}&\wbox{00}{12}&G/P&\wbox{00/00/0}{\wbox[c]{0}{--}/2/1}&14.5 mm&L&\wbox{00}{6}&\wbox{00}{2}&\wbox{00}{3}&\wbox{00}{5}&\wbox{00}{2}&\wbox{00}{1}&\wbox{00}{0}&\wbox{0}{1}&---&---&---\\
Mobile ZPU-2 &\wbox[l]{Section}{Section}&1949&\wbox[l]{0}{2}&\wbox{00}{12}&G/P&\wbox{00/00/0}{\wbox[c]{0}{--}/3/2}&\binarymultiply{14.5 mm}{2}&L&\wbox{00}{6}&\wbox{00}{2}&\wbox{00}{3}&\wbox{00}{5}&\wbox{00}{3}&\wbox{00}{2}&\wbox{00}{1}&\wbox{0}{1}&---&---&---\\
Mobile ZPU-4 &\wbox[l]{Section}{Section}&1949&\wbox[l]{0}{2}&\wbox{00}{12}&G/P&\wbox{00/00/0}{\wbox[c]{0}{--}/4/2}&\binarymultiply{14.5 mm}{4}&L&\wbox{00}{6}&\wbox{00}{2}&\wbox{00}{3}&\wbox{00}{5}&\wbox{00}{4}&\wbox{00}{3}&\wbox{00}{2}&\wbox{0}{2}&---&---&---\\
\addlinespace
M167 &\wbox[l]{Section}{Platoon}&1967&\wbox[l]{0}{3}&\wbox{00}{12}&T&\wbox{00/00/0}{7/5/2}&\binarymultiply{20 mm}{6}&L&\wbox{00}{8}&\wbox{00}{2}&\wbox{00}{4}&\wbox{00}{6}&\wbox{00}{6}&\wbox{00}{5}&\wbox{00}{4}&\wbox{0}{3}&R/HF&---&---\\
M163 &\wbox[l]{Section}{Section}&1968&\wbox[l]{0}{\underline{4}}&\wbox{00}{12}&U&\wbox{00/00/0}{\wbox[c]{0}{--}/8/4}&\binarymultiply{20 mm}{6}&L&\wbox{00}{8}&\wbox{00}{2}&\wbox{00}{4}&\wbox{00}{6}&\wbox{00}{5}&\wbox{00}{4}&\wbox{00}{3}&\wbox{0}{3}&R/HF&---&---\\
TCM-20 &\wbox[l]{Section}{Platoon}&1970&\wbox[l]{0}{3}&\wbox{00}{12}&T&\wbox{00/00/0}{5/3/2}&\binarymultiply{20 mm}{2}&L&\wbox{00}{8}&\wbox{00}{2}&\wbox{00}{4}&\wbox{00}{6}&\wbox{00}{4}&\wbox{00}{4}&\wbox{00}{3}&\wbox{0}{2}&---&---&---\\
M3 TCM-20 &\wbox[l]{Section}{Section}&1970&\wbox[l]{0}{\underline{3}!}&\wbox{00}{12}&U&\wbox{00/00/0}{\wbox[c]{0}{--}/6/3}&\binarymultiply{20 mm}{2}&L&\wbox{00}{8}&\wbox{00}{2}&\wbox{00}{4}&\wbox{00}{6}&\wbox{00}{3}&\wbox{00}{3}&\wbox{00}{2}&\wbox{0}{2}&---&---&---\\
Single Rh-202 &\wbox[l]{Section}{Battery}&1972&\wbox[l]{0}{3}&\wbox{00}{12}&T&\wbox{00/00/0}{4/3/1}&20 mm&L&\wbox{00}{8}&\wbox{00}{2}&\wbox{00}{4}&\wbox{00}{6}&\wbox{00}{4}&\wbox{00}{3}&\wbox{00}{2}&\wbox{0}{2}&---&---&---\\
Twin Rh-202 &\wbox[l]{Section}{Battery}&1972&\wbox[l]{0}{3}&\wbox{00}{12}&T&\wbox{00/00/0}{5/3/2}&\binarymultiply{20 mm}{2}&L&\wbox{00}{8}&\wbox{00}{2}&\wbox{00}{4}&\wbox{00}{6}&\wbox{00}{5}&\wbox{00}{4}&\wbox{00}{3}&\wbox{0}{3}&---&---&---\\
Panhard M3 DCA &\wbox[l]{Section}{Section}&1975&\wbox[l]{0}{\underline{4}}&\wbox{00}{12}&U&\wbox{00/00/0}{\wbox[c]{0}{--}/7/4}&\binarymultiply{20 mm}{2}&L&\wbox{00}{8}&\wbox{00}{2}&\wbox{00}{4}&\wbox{00}{6}&\wbox{00}{4}&\wbox{00}{4}&\wbox{00}{3}&\wbox{0}{2}&R/UF&---&---\\
\addlinespace
ZU-23 &\wbox[l]{Section}{Battery}&1960&\wbox[l]{0}{3}&\wbox{00}{12}&T&\wbox{00/00/0}{6/4/2}&\binarymultiply{23 mm}{2}&L&\wbox{00}{9}&\wbox{00}{2}&\wbox{00}{5}&\wbox{00}{8}&\wbox{00}{5}&\wbox{00}{4}&\wbox{00}{3}&\wbox{0}{3}&---&---&---\\
Mobile ZU-23 &\wbox[l]{Section}{Section}&1960&\wbox[l]{0}{2}&\wbox{00}{12}&G/P&\wbox{00/00/0}{\wbox[c]{0}{--}/5/3}&\binarymultiply{23 mm}{2}&L&\wbox{00}{9}&\wbox{00}{2}&\wbox{00}{5}&\wbox{00}{8}&\wbox{00}{4}&\wbox{00}{3}&\wbox{00}{2}&\wbox{0}{3}&---&---&---\\
Sinai 23 Ayn-al-Saqr &\wbox[l]{Section}{Section}&1992&\wbox[l]{0}{\underline{4}}&\wbox{00}{12}&U&\wbox{00/00/0}{\wbox[c]{0}{--}/10/5}&\binarymultiply{23 mm}{2}&L&\wbox{00}{9}&\wbox{00}{2}&\wbox{00}{5}&\wbox{00}{8}&\wbox{00}{4}&\wbox{00}{3}&\wbox{00}{2}&\wbox{0}{3}&---&Y&Ayn-al-Saqr\\
Sinai 23 Stinger &\wbox[l]{Section}{Section}&1992&\wbox[l]{0}{\underline{4}}&\wbox{00}{12}&U&\wbox{00/00/0}{\wbox[c]{0}{--}/12/6}&\binarymultiply{23 mm}{2}&L&\wbox{00}{9}&\wbox{00}{2}&\wbox{00}{5}&\wbox{00}{8}&\wbox{00}{4}&\wbox{00}{3}&\wbox{00}{2}&\wbox{0}{3}&---&Y&FIM-92\\
ZSU-23-4 &\wbox[l]{Section}{Section}&1965&\wbox[l]{0}{\underline{4}}&\wbox{00}{12}&U&\wbox{00/00/0}{\wbox[c]{0}{--}/9/5}&\binarymultiply{23 mm}{4}&L&\wbox{00}{9}&\wbox{00}{2}&\wbox{00}{5}&\wbox{00}{8}&\wbox{00}{5}&\wbox{00}{4}&\wbox{00}{3}&\wbox{0}{3}&W/VF&---&---\\
\addlinespace
LAV-AD &\wbox[l]{Section}{Section}&1997&\wbox[l]{0}{\underline{4}}&\wbox{00}{12}&U&\wbox{00/00/0}{\wbox[c]{0}{--}/14/7}&\binarymultiply{25 mm}{5}&L&\wbox{00}{10}&\wbox{00}{3}&\wbox{00}{6}&\wbox{00}{9}&\wbox{00}{4}&\wbox{00}{3}&\wbox{00}{2}&\wbox{0}{4}&---&Y&FIM-92D\\
\addlinespace
M53/59 &\wbox[l]{Section}{Section}&1959&\wbox[l]{0}{3}&\wbox{00}{12}&G/P&\wbox{00/00/0}{\wbox[c]{0}{--}/7/4}&\binarymultiply{30 mm}{2}&L&\wbox{00}{11}&\wbox{00}{4}&\wbox{00}{7}&\wbox{00}{9}&\wbox{00}{3}&\wbox{00}{2}&\wbox{00}{1}&\wbox{0}{3}&---&---&---\\
AMX-30SA &\wbox[l]{Section}{Section}&1979&\wbox[l]{0}{\underline{4}}&\wbox{00}{12}&U&\wbox{00/00/0}{\wbox[c]{0}{--}/10/5}&\binarymultiply{30 mm}{2}&L&\wbox{00}{11}&\wbox{00}{4}&\wbox{00}{7}&\wbox{00}{9}&\wbox{00}{5}&\wbox{00}{4}&\wbox{00}{3}&\wbox{0}{3}&W/VF&---&---\\
Tunguska &\wbox[l]{Section}{Section}&1984&\wbox[l]{0}{\underline{4}}&\wbox{00}{12}&U&\wbox{00/00/0}{\wbox[c]{0}{--}/20/10}&\binarymultiply{30 mm}{2}&L&\wbox{00}{11}&\wbox{00}{4}&\wbox{00}{7}&\wbox{00}{9}&\wbox{00}{5}&\wbox{00}{4}&\wbox{00}{3}&\wbox{0}{4}&W/VF&---&SA-19\\
Tunguska-M &\wbox[l]{Section}{Section}&1990&\wbox[l]{0}{\underline{4}}&\wbox{00}{12}&U&\wbox{00/00/0}{\wbox[c]{0}{--}/22/11}&\binarymultiply{30 mm}{2}&L&\wbox{00}{11}&\wbox{00}{4}&\wbox{00}{7}&\wbox{00}{9}&\wbox{00}{5}&\wbox{00}{4}&\wbox{00}{3}&\wbox{0}{4}&W/VF&---&SA-19\\
\addlinespace
Oerlikon GDF &\wbox[l]{Section}{Battery}&1963&\wbox[l]{0}{3}&\wbox{00}{12}&T&\wbox{00/00/0}{7/5/2}&\binarymultiply{35 mm}{2}&M&\wbox{00}{12}&\wbox{00}{4}&\wbox{00}{8}&\wbox{00}{10}&\wbox{00}{4}&\wbox{00}{3}&\wbox{00}{2}&\wbox{0}{4}&---&---&---\\
Oerlikon GDF &\wbox[l]{Section}{Section}&1963&\wbox[l]{0}{3}&\wbox{00}{12}&T&\wbox{00/00/0}{\wbox[c]{0}{--}/5/2}&\binarymultiply{35 mm}{2}&M&\wbox{00}{12}&\wbox{00}{4}&\wbox{00}{8}&\wbox{00}{10}&\wbox{00}{3}&\wbox{00}{2}&\wbox{00}{1}&\wbox{0}{4}&---&---&---\\
Gepard &\wbox[l]{Section}{Section}&1976&\wbox[l]{0}{\underline{4}}&\wbox{00}{12}&U&\wbox{00/00/0}{\wbox[c]{0}{--}/11/6}&\binarymultiply{35 mm}{2}&M&\wbox{00}{12}&\wbox{00}{4}&\wbox{00}{8}&\wbox{00}{10}&\wbox{00}{5}&\wbox{00}{4}&\wbox{00}{3}&\wbox{0}{4}&W/VF&---&---\\
\addlinespace
61-K &\wbox[l]{Section}{Battery}&1939&\wbox[l]{0}{3}&\wbox{00}{12}&T&\wbox{00/00/0}{5/3/2}&37 mm&M&\wbox{00}{12}&\wbox{00}{4}&\wbox{00}{8}&\wbox{00}{11}&\wbox{00}{3}&\wbox{00}{2}&\wbox{00}{1}&\wbox{0}{4}&---&---&---\\
\addlinespace
Bofors L/60 &\wbox[l]{Section}{Battery}&1935&\wbox[l]{0}{3}&\wbox{00}{12}&T&\wbox{00/00/0}{5/3/2}&40 mm&M&\wbox{00}{15}&\wbox{00}{4}&\wbox{00}{8}&\wbox{00}{12}&\wbox{00}{2}&\wbox{00}{2}&\wbox{00}{1}&\wbox{0}{4}&---&---&---\\
Bofors L/70 &\wbox[l]{Section}{Battery}&1952&\wbox[l]{0}{3}&\wbox{00}{12}&T&\wbox{00/00/0}{6/4/2}&40 mm&M&\wbox{00}{15}&\wbox{00}{4}&\wbox{00}{8}&\wbox{00}{12}&\wbox{00}{3}&\wbox{00}{3}&\wbox{00}{2}&\wbox{0}{4}&---&---&---\\
Bofors L/70 BOFI &\wbox[l]{Section}{Battery}&&\wbox[l]{0}{3}&\wbox{00}{12}&T&\wbox{00/00/0}{8/5/3}&40 mm&M&\wbox{00}{15}&\wbox{00}{4}&\wbox{00}{8}&\wbox{00}{12}&\wbox{00}{5}&\wbox{00}{5}&\wbox{00}{4}&\wbox{0}{4}&---&---&---\\
Bofors L/70 BOFI-R &\wbox[l]{Section}{Battery}&&\wbox[l]{0}{3}&\wbox{00}{12}&T&\wbox{00/00/0}{9/6/3}&40 mm&M&\wbox{00}{15}&\wbox{00}{4}&\wbox{00}{8}&\wbox{00}{12}&\wbox{00}{6}&\wbox{00}{6}&\wbox{00}{5}&\wbox{0}{4}&W/VF&---&---\\
\addlinespace
S-60 &\wbox[l]{Section}{Battery}&1950&\wbox[l]{0}{3}&\wbox{00}{12}&T&\wbox{00/00/0}{8/5/3}&57 mm&M&\wbox{00}{18}&\wbox{00}{5}&\wbox{00}{10}&\wbox{00}{15}&\wbox{00}{2}&\wbox{00}{2}&\wbox{00}{1}&\wbox{0}{5}&---&---&---\\
ZSU-57-2 &\wbox[l]{Section}{Section}&1955&\wbox[l]{0}{\underline{4}!}&\wbox{00}{12}&U&\wbox{00/00/0}{\wbox[c]{0}{--}/6/3}&\binarymultiply{57 mm}{2}&M&\wbox{00}{18}&\wbox{00}{5}&\wbox{00}{10}&\wbox{00}{15}&\wbox{00}{2}&\wbox{00}{1}&\wbox{00}{1}&\wbox{0}{5}&---&---&---\\
\addlinespace
KS-12 &\wbox[l]{Section}{Battery}&1939&\wbox[l]{0}{4}&\wbox{00}{18}&T&\wbox{00/00/0}{9/6/3}&85 mm&H&\wbox{00}{27}&\wbox{00}{6}&\wbox{00}{12}&\wbox{00}{18}&\wbox{00}{2}&\wbox{00}{1}&\wbox{00}{0}&\wbox{0}{6}&---&---&---\\
\addlinespace
KS-19 &\wbox[l]{Section}{Battery}&1948&\wbox[l]{0}{4}&\wbox{00}{18}&T&\wbox{00/00/0}{10/7/3}&100 mm&H&\wbox{00}{39}&\wbox{00}{7}&\wbox{00}{14}&\wbox{00}{21}&\wbox{00}{1}&\wbox{00}{1}&\wbox{00}{0}&\wbox{0}{6}&---&---&---\\
\addlinespace
\bottomrule
\end{tabular}

    \label{table:aaa-units}
\end{table*}

\clearpage
\section{FCR Units}

Most medium and heavy AAA batteries can be associated with an add-on fire-control radar (FCR), which improves their accuracy and allows them to engage unsighted targets at night and in adverse weather.

\subsection{Properties}

Table~\ref{table:fcr-units} summarizes the properties of FCR units, and Figure~\ref{figure:radar-and-ccu-units} shows their representation.
In addition to the properties common to other ground units, Table~\ref{table:fcr-units} gives:
\begin{itemize}
    \item the FCR class;
    \item the frequency band; and
    \item its modifier to the hit roll.
\end{itemize}

\subsection{Specific FCRs}

\subsubsection{SON-9}

The Soviet SON-9 (Fire Can) FCR-A was derived from the earlier SON-4, which in turn was derived from the US-supplied SCR-584.
It is commonly used with 57~mm S-60, 85~mm KS-12, 100~mm KS-19, and 130~mm KS-30 batteries.

The SON-9 entered service in 1955 with Soviet forces and was widely exported to Soviet allies.
It has seen combat, including with North Vietnam in the Vietnam War, Egypt and Syria in the 1967 War, the War of Attrition, and the 1973 War.

\subsubsection{Super Fledermaus}

The Contraves Super Fledermaus FCR-C can be used to control a section of two Oerlikon GDF 35~mm guns.

It entered service in 1965 with the Swiss Air Force and also was used by the Argentine Air Force, Brazil, Iran, Japan, and South Africa.
It saw combat in the 1982 South Atlantic War in the defense of BAM Malvinas (Port Stanley Airport).

\subsubsection{Flycatcher}

The Signaal Flycatcher FCR-C can control three to six Bofors L/60 or L/70 guns or SAM launchers.

Flycatcher entered service with the Royal Netherlands Air Force in 1979. It subsequently served with the Indian Armed Forces from 1985, the Irish Defense Forces from 2003, the Royal Netherlands Army from 1987,the Thai Armed Forces from 1981, Turkey, and Venezuela.

\subsubsection{Skyguard}

The Contraves Skyguard FCR-D can control two Oerlikon GDF 35~mm guns and additionally a SAM launcher with AIM-9 Sparrow, RIM-9 Sea Sparrow, or Aspide missiles.

Skyguard entered service in 1977 with the Swiss Air Force.
It was subsequently very widely exported and used by the Argentine Army, Austria, Bahrain, Bangladesh, Brazil, Canada, Chile, China, Cyprus, Egypt, Greece, Indonesia, Iran, Italy, South Korea, Kuwait, Malaysia, Oman, Pakistan, Saudi Arabia, Spain, Taiwan, and the UK.
It saw combat with the Argentine Army in the 1982 South Atlantic War in the defense of BAM Malvinas (Port Stanley Airport) and BAM Condor (Goose Green Airfield).

\begin{figure*}[p]
    \centering
    \includegraphics[width=0.8\linewidth]{figure-radar-and-ccu-units.png}
    \caption{Air-Defense Units: FCR, EWR, and CCU Units}
    \label{figure:radar-and-ccu-units}
\end{figure*}

\begin{table*}[p]
    \caption{Air-Defense Units: FCR Units}
    \centering
    \footnotesize
    \begin{tabular}{lccccccccccc}
\toprule

            Type&
            \vertical{Size}&
            \vertical{Defense Strength}&
            \vertical{Sighting Range}&
            \vertical{Mobility}&
            \begin{tabular}[b]{@{}c@{}}\vertical{VPs}\\\midrule3D/2D/D\\\end{tabular}&
            %\vertical{Class}&
            %\vertical{Frequency}&
            %\vertical{Modifier}
            \multicolumn{3}{c}{\begin{tabular}[b]{@{}ccc@{}}\multicolumn{3}{@{}c@{}}{\vertical{FCR}}\\\midrule\vertical{Class}&\wbox[c]{MW}{\vertical{Frequency}}&\wbox[c]{\minus{0}}{\vertical{Modifier}}\end{tabular}}
            \\
        
\midrule
\addlinespace
Towed FCR-A &\wbox[l]{Section}{Platoon}&\wbox[l]{0}{2}&\wbox{00}{12}&T&\wbox{00/00/00}{6/4/2}&\wbox[l]{D}{A}&\wbox[l]{MW}{LF}&\minus{1}\\
Towed FCR-B &\wbox[l]{Section}{Platoon}&\wbox[l]{0}{2}&\wbox{00}{12}&T&\wbox{00/00/00}{6/4/2}&\wbox[l]{D}{B}&\wbox[l]{MW}{MF}&\minus{2}\\
Towed FCR-C &\wbox[l]{Section}{Platoon}&\wbox[l]{0}{2}&\wbox{00}{12}&T&\wbox{00/00/00}{7/5/2}&\wbox[l]{D}{C}&\wbox[l]{MW}{VF}&\minus{2}\\
Towed FCR-D &\wbox[l]{Section}{Platoon}&\wbox[l]{0}{2}&\wbox{00}{12}&T&\wbox{00/00/00}{8/5/3}&\wbox[l]{D}{D}&\wbox[l]{MW}{MW}&\minus{3}\\
\addlinespace
\bottomrule
\end{tabular}

    \label{table:fcr-units}
\end{table*}

\clearpage
\section{SAM Units}

SAM launcher units are specialized in the destruction of attacking aircraft with surface-to-air missiles.

\subsection{Properties}

Table~\ref{table:infantry-sam-units} summarizes the properties of infantry SAM units, and
Figure~\ref{figure:infantry-sam-units} shows their representation as counters. In addition to the properties shared by all ground units, Table~\ref{table:infantry-sam-units} gives:
\begin{itemize}
    \item the number of ready, volley, and reload missiles;
    \item the multi-target capability;
    \item whether the unit has quick-reaction capability;
    \item whether the unit has IFF capability;
    \item whether the unit has night IR sights; and
    \item the optical lock-on number and range.
\end{itemize}

%the frequency band, range (in mega-hexes) of the EWR, if present, and whether it has a moving-target indicator (MTI) capability; the frequency band and range (in mega-hexes) of the target-tracking radar, if present; the multi-target capability of the unit; the number of ready missiles and the maximum number of missiles in a volley; whether the unit has a quick-reaction capability; its radar and optical lock-on rolls and (for IR SAMs) its lock-on range in hexes.

\subsection{Missile Properties}

Table~\ref{table:sams} summarizes the properties of the actual missiles:
\begin{itemize}
    \item guidance mode;
    \item launch roll;
    \item turn rate;
    \item flight time;
    \item visibility;
    \item ECCM, chaff, and flare ratings;
    \item whether the missile is active homing (AH) or has a home-on-jam (HOJ) capability;
    \item boost phase in FPs;
    \item base speed and sustainer duration in game turns;
    \item minimum altitude (and modifier for targets at T level);
    \item hit rolls for direct and proximity hits; and
    \item damage ratings for direct and proximity hits.
\end{itemize}

%Some missiles have more than one guidance mode (e.g., the SA-2F has an OG backup mode to its CG main mode).

\subsection{Infantry SAM Units}

Infantry SAM units are squads of infantry specialized in the destruction of attacking aircraft with portable SAMs.
Each squad has one launcher and, unless otherwise specified, one ready missile and two reload missiles.

Infantry SAM squads may be attached to infantry, infantry weapons, or infantry HQ platoons.

Most infantry SAMs do not have true proximity fuzes. Thus, proximity hits in the game correspond to glancing direct hits in reality.

\subsubsection{SA-7A/B Squad}

The 9K32 Strela-2 (SA-7A Grail) was the first Soviet portable SAM. It has a rear-aspect, uncooled IR seeker.

The 9K32M Strela-2M (SA-7B Grail) is similar, but uses a rear-aspect, electrically-cooled IR seeker.
In Warsaw Pact armies, the Strela-2M was sometimes used with the PRP portable RWR receiver that was capable of passively identifying enemy aircraft by their radar emissions.%
\note{
    For the SA-7A: The launcher and SAM values are from \emph{Air Strike} with significant modifications:
    \begin{itemize}
        \item Zaloga gives an in-service date of 1968.
        \item \emph{Air Strike} gives this a seeker type of I, but that does seem appropriate for an uncooled PbS seeker. Instead, I use a seeker type of E, by analogy with the uncooled seeker in the AIM-9B, and works at about 2 microns.
        \item \emph{Air Strike} gives a turn rate of HT, but the control capability seems to be the same as in the SA-7B and SA-15. I will adopt BT.
        \item \emph{Air Strike} gives this a speed of 8. Zaloga gives a lifetime of 8 seconds and an average speed of 1550 km/h. This corresponds to a range of 5.2 km or about 10 hexes.
    \end{itemize}
}%
\note{
    For the SA-7B: The launcher and SAM values are from \emph{Air Strike} with significant modifications:
    \begin{itemize}
        \item \emph{Air Strike} gives this a seeker type of I. That seems about right. The seeker was PbS and cooled electrically, like the AIM-9E which also has a type I seeker.
        \item \emph{Air Strike} gives a turn rate of BT, which I adopt.
        \item \emph{Air Strike} gives this a speed of 10. Zaloga states that it has a slightly more powerful rocket motor than the SA-7A, but the mean speed is the same at 1550 km/h. I adopt a speed of 10, like that of the SA-7A.
    \end{itemize}
}

The Strela-2 and -2M entered service with Soviet forces in 1968 and 1970.
They have seen combat in almost every conflict since 1970 in which the Soviet Union or its successor states have been participants or supporters.
In particular, the Strela-2M was used heavily by North Vietnamese forces in the 1973 Easter Offensive, by Arab forces in the 1973 Yom Kippur War, by Angolan forces in the South African Border War, by the Afghan Mujahedeen in the Soviet-Afghan War (having been supplied by the CIA), by Iran in the Iran-Iraq War in the 1980s, and by Serbian and Serbian-supported forces in the Yugoslav Wars in the 1990s.

\subsubsection{SA-14 Squad}

The 9K34 Strela-3 (SA-14 Gremlin) is a development of the earlier Strela-2M.
It has a rear-aspect, cryogen-cooled IR seeker.
In Warsaw Pact armies, the Strela-3 was sometimes used with the PRP portable RWR receiver that was capable of passively identifying enemy aircraft by their radar emissions.%
\note{
    For the SA-14: The launcher and SAM values are from \emph{Air Strike} with significant modifications:
    \begin{itemize}
        \item The seeker is cryogenically cooled at works at about 4 microns (Zaloga). \emph{Air Strike} gives this a type A seeker, but that does seem appropriate; the first type A seekers in Soviet AAMs did not appear until 1985 in the AA-8C. Also, type A seekers have all-aspect capability Zaloga says that the seeker has similar performance to the RIM-43C, but only limited all-aspect capability. A type M seeker seems to be more appropriate.
        \item \emph{Air Strike} gives a turn rate of ET, but the control capability seems to be the same as in the SA-7A and SA-7B. I will adopt BT.
        \item \emph{Air Strike} gives this a speed of 12. Zaloga gives a lifetime of 8 seconds and an average speed of 1440 km/h (as it is slightly heavier than the SA-7). This corresponds to a range of 4.8 km or about 9 hexes.
        \item The warhead is the same as on the SA-7A/7B.
    \end{itemize}
}

The Strela-3 entered service with the Soviet Army in 1974.
It has seen combat in many conflicts since then in which the Soviet Union or its successor states have been participants or supporters.
In particular, it was used by Iraqi forces in the 1991 Gulf War.

\subsubsection{SA-16/18 Squad}

9K310 Igla-1 (SA-16 Gimlet) and 9K38 Igla (SA-18 Grouse) use very similar missiles and launchers, but the Igla-1 has a single-channel IR seeker similar to the SA-14, and the Igla uses a dual-channel IR seeker to reduce its susceptibility to flares.
The warhead was similar to the SA-7/14, but also exploded any remaining rocket fuel.
In the game, this is simulated by increasing the damage rating by one if the missile attacks before expending more than half of its FPs (round down).
Both incorporate IFF interrogators, but these were only used in Warsaw Pact armies.%
\note{
    For the SA-16/18:  The launcher and SAM values are from Malcolm Pipes with significant modifications:
    \begin{itemize}
        \item The warhead of the SA-16/18 is 1.17 kg, identical to the 1.17 kg of the SA-7 and similar to the 1.1 kg of the SA-14. Only later Iglas had heavier warheads. Therefore, I have reduced the damage ratings to match the SA-7.
        \item Carl comments that the warhead explodes any remaining rocket fuel. I increase the damage rating by one if the target is reached in the first half of the FPs.
        \item I have reduced the lock-on range of the SA-16 to match that of the SA-14.
        \item I have increased the flare rating of the SA-16 to match that of the SA-14.
        \item I have given both missiles the same speed and turn rate. Zaloga states that the engagement range increased by about 1 km with respect to the SA-7/14, so I am increasing the speed by 2 to 12. I will give both a turn rate of BT/2, matching Stinger.
    \end{itemize}
}

The Igla-1 entered service with the Soviet Army in 1981 and the Igla in 1983.
Both have been widely exported and have seen combat in many conflicts since then in which the Soviet Union or its successor states have been participants or supporters.
The Igla was used by Iraqi forces in the 1991 Gulf War, by the Peruvian Army in the 1995 Cenepa War, and by Serbian forces in the 1999 Kosovo War.

\subsubsection{FIM-43C Squad}

The FIM-43C Redeye Block III was the first US infantry SAM to reach production and enter service.
It has a rear-aspect, cryogen-cooled IR seeker.%
\note{
    For the FIM-43: The launcher and SAM values are from \emph{Air Strike}.
}

The FIM-43C entered service with the US Army in 1968 and subsequently was used by Australia, Chad, Denmark, West Germany, Greece, Israel, Jordan, Saudi Arabia, Somalia, Sudan, Sweden, and Turkey. It was also supplied by the CIA to the Afghan Mujahedeen and the Nicaraguan Contras.

\subsubsection{FIM-92A/B/C/D Squad}

The FIM-92A Stinger has an all-angle, second-generation, cryogen-cooled IR seeker.
Its warhead weight 3 kg and was significantly more lethal than in earlier missiles.
The FIM-92B Stinger POST adds a UV seeker to distinguish flares from aircraft.
The FIM-92C/D Stinger RMP have software updates to further improve the performance against flares.
All have an IFF interrogator to help avoid fratricide.%
\note{
    For the FIM-92: The launcher and SAM values for the A model are from \emph{Air Strike}. Those for the B/C/D models are from Malcolm Pipes compendium.
    \begin{itemize}
        \item I have increased the flare susceptibility of the FIM-92A from 2 to 3. It did not seem to have significant protection.
        \item Wikipedia gives an effective range of 5 miles, which is equivalent to a speed of 15. I have left the speed at 14, as that is close enough.
        \item Wikipedia gives a peak speed of 1930 mp and a self-destruct time of 17 seconds.
        \item \emph{Air Strike} has the visibility as 2, but looking at photos I don't see much differ
    \end{itemize}
}

The FIM-92A entered service with the US Army in 1981, the FIM-92B in 1986, the FIM-92C in 1989, and the FIM-92D in 1992. They were exported to dozens of US allies including the Afghan Mujahedeen and the Angolan UNITA. They saw service in the 1982 South Atlantic War, the Soviet-Afghan War, the Angolan Civil War, the Libyan Invasion of Chad, the Tajik Civil War, the Chechen War, the Sri Lankan Civil War, the Syrian Civil War, and the Russian Invasion of Ukraine.

\subsubsection{Blowpipe Squad}

Blowpipe is an MCLOS optically-guided missile.
While in theory this gave it an all-aspect capability, in practice guiding the missile to the target in combat proved to be very difficult.%
\note{
    For the Blowpipe: The launcher and SAM values are from \emph{Air Strike}.
}

Blowpipe entered service with the British Army and Royal Marines in 1975.
It was also used by the Argentine Army, Canadian Army, Chilean Army and Navy, Ecuadorean Army, Guatemalan Army, Portuguese Army, Malawian Army, Malaysian Army, Nigerian Army, Omani Army, Royal Thai Air Force and Army, and the UAE Army.
It saw combat in the 1982 South Atlantic War with both sides and had a low success rate.
In 1986, it was trialed by the Afghan Mujahedeen, but again was not considered to be effective.
Finally, it was also used in the 1995 Cenepa War by Ecuador.

\subsubsection{Javelin/Starburst Squad}

Given the poor performance of Blowpipe in the 1982 South Atlantic War, Javelin was quickly developed as an upgraded version of Blowpipe with SACLOS guidance in place of MCLOS guidance. Starburst then replaced the radio-link with laser guidance. Javelin is also called “Javelin GL” and Starburst was originally called “Javelin S15”.%
\note{
    For the Javelin/Starburst: The launcher and SAM values for Javelin are from \emph{Air Strike}. The launcher and SAM values for Javelin are just those for Javelin but with the OG guidance replaced by LG guidance. I have seen mention of a Javelin LML; did anyone use it or was it just a prototype?
}

Javelin replaced Blowpipe in the British Army and Royal Marines in 1984.
It was also used by the Botswana Army, the Canadian Army, the Malaysian Army, the Peruvian Army, and the South Korean Army.

Starburst in turn replaced Javelin in the British Army and Royal Marines in 1989.
It was also used by the Canadian Army, the Malaysian Army, the Qatari Army, and the Royal Thai Army.
Starburst served with the British Army in the 1991 Gulf War, but was not used in combat.

\subsubsection{Starstreak Squad}

Starstreak is a high-speed laser-guided missile. As the missile approaches the target, it launches three independent explosive darts. The darts have contact fuzes, but not proximity fuzes.%
\note{
    For the Starstreak: The launcher and SAM values are from Malcolm Pipes. The real darks do not have proximity fuzes, but the missile in the game can obtain a proximity hit. Perhaps a proximity hit here means a hit with only one die?
}

Starstreak replaced Javelin in the British Army and Royal Marines in 2000 and is also used by the South African Army and the Armed Forces of Ukraine.
Starstreak served with the British Army in the 2003 Invasion of Iraq, but was not used in combat. It has seen combat in the Russian Invasion of Ukraine.

\subsubsection{Starstreak LML Squad}

The Starstreak Light Multiple Launcher (LML) has three ready-to-fire Starstreak missiles on a pedestal mount with a single sighting unit. The launcher and its associated equipment are heavy and will normally be transported by a vehicle unit.%
\note{
    For the Starstreak LML: The SAM values are from Malcolm Pipes. I have guessed the launcher values are the same as for shoulder-launched version, just that the launcher has three ready missiles. I don't know if the launcher has reloads. Is the launcher really portable or does it have to be transported by a vehicle?
}

Starstreak LML entered service with the British Army and Royal Marines in 2000.
It is also used by Indonesia, Malaysia, and South Africa.

\subsubsection{Mistral Squad}

The Mistral is a French IR-homing missile on a pedestal mount. The missile has a considerably larger warhead than either the Stinger or Igla missiles, but this comes at a cost in portability. The launcher and its associated equipment are heavy and will normally be transported by a vehicle unit.%
\note{
    For the Mistral: The launcher and SAM values are from \emph{Air Strike}. Is the launcher really portable or does it have to be transported by a vehicle?
}

The Mistral entered service with the French Army in 1990. It was subsequently exported to many other countries.

\subsubsection{RBS 70 Squad}

The RBS 70 is a Swedish laser-guided missile on a pedestal mount. The launcher and its associated equipment are heavy and will normally be transported by a vehicle unit.%
\note{
    For the RBS 70: The launcher and SAM values are from \emph{Air Strike}. Is the launcher really portable or does it have to be transported by a vehicle?
}

The RBS 70 entered service with the Swedish Army in 1977. It was subsequently used by Argentina, Australia, Bahrain, Brazil, the Czech Republic, Iran, Ireland, Lithuania, Norway, Pakistan, Singapore, Thailand, Tunisia, the UAE, Ukraine, and Venezuela.

\subsubsection{Ayn-al-Saqr Squad}

The Ayn-al-Saqr (”Eye of the Hawk”) missile is a  licensed version of the 9M32M Strela-2m (SA-7B Grail) manufactured in Egypt. It equips one version of the Sinai 23 air-defense vehicle.%
\note{
    For the Ayn-al-Saqr missile: The launcher and SAM values are taken to be equal to those of the SA-7B.
}

It entered service with the Egyptian Army in 1984.

\subsection{Vehicular and Towed SAM Units}

\begin{comment}
\begin{figure*}
    \centering
    \includegraphics[width=0.8\linewidth]{figure-sam-launcher.png}
    \caption{SAM Launcher Units}
    \label{figure:sam-launcher-units}
\end{figure*}

\paragraph{Echelon.}
The size of a SAM launcher unit is squad, section, or platoon according to its damage resilience (1D, 2D, or 3D, respectively).
This does not always coincide with the actual designations used historically. For example, the SA-2 battery in the game corresponds to a battalion in reality.

\paragraph{Nuclear Warheads.}
The SA-2E has the option of a nuclear warhead.

\subsection{Soviet and Russian SAM Launcher Units}

\subsubsection{SA-2B/C/E/F Battery}

This is an S-75 (SA-2 Guideline) battery of six single launchers for the long-range V-750 missile and their associated P-12 (Spoon Rest) EWR and SNR-75 (Fan Song) TTR. The variants are the S-75 Desna (SA-2B), S-75M Volkhov (SA-2C), S-75AK (SA-2E), and S-75SM (SA-2F).
All are radar-guided, but the SA-2F also has an optical backup.

The S-75 Dvina (SA-2A) system was deployed by the Soviet PVO in 1957. In addition to serving as a key part of the air-defense system of the Soviet Union (and its notable role in ending reconnaissance overflights by shooting down the U-2 of Gary Powers) and its allies, it saw combat in Cuba in 1962, with North Vietnamese forces from 1965, with Indian forces in the 1965 war with Pakistan, and with Egyptian and Syrian forces from 1967.

\subsubsection{SA-3B Battery}

This is an S-125 Neva (SA-3B Goa) battery with four quad launchers for the 5V24 missile and their associated P-15 (Flat Face) EWR and SNR-125 (Low Blow) TTR.

It saw combat with Egyptian and Syrian forces from 1970.

\subsubsection{Mobile SA-3B Battery}

This is a mobile S-125 Neva (SA-3B Goa) battery with four twin launchers for the 5V24 missile on trucks and their associated P-15 (Flat Face) EWR and SNR-125 (Low Blow) TTR.

It saw combat with Egyptian and Syrian forces from 1970 and with Iraqi forces in the Iran-Iraq War, the Gulf War, and the Invasion of Iraq.

\subsubsection{SA-4A/B Battery}

This is an armored 2K11 Krug (SA-4A/B Ganef) battery with three launcher vehicles each with two long-range 9M8M1/9M8M2 missiles and their associated P-40 (Long Track) EWR and 1S32 (Pat Hand) TTR.

It saw combat with Egyptian and Syrian forces from 1970 and with Iraqi forces in the Iran-Iraq War, the Gulf War, and the Invasion of Iraq.

\subsubsection{SA-6A/B Battery}

This is an armored 2K12 Kub (SA-6A/B Gainful) battery with four launcher vehicles, each with three medium-range 2K12 Kub (SA-6A) or 3M9M3 (SA-6B) missiles and their associated P-40 (Long Track) EWR and 1S32 (Pat Hand) TTR.

It saw combat with Egyptian and Syrian forces from 1970 and with Iraqi forces in the Iran-Iraq War, the Gulf War, and the Invasion of Iraq.

\subsubsection{SA-8A/B Squad}

This is a squad with a single 9K33 Osa or 9K33M Osa-AK (SA-8A/B Gecko) vehicle with integrated 1S51M3 (Land Roll) TTR.
The SA-8A carries four 9M33 missiles, and the SA-8B carries six 9M33M missiles.

The Osa has been widely exported and seen combat with Syrian forces in the 1982 Lebanon War, with Cuban forces in the South African Border War, with Iraqi forces in the 1990 Gulf War, and in more recent conflicts.

\subsubsection{SA-9A/B Squad}

This is a squad with a single 9K21 Strela-1 or 9K21M Strela-1M (SA-9A/B Gaskin) vehicle with four 9M21 or 9M21M missiles.
In the Soviet army, a platoon of four was assigned to the air-defense battery of tank and motor-rifle regiments, complementing the ZSU-23-4.

The Strela-1 has been widely exported and saw combat in the South African Border War, the Iran-Iraq War, the Lebanon War, the Gulf War, the Kosovo War, and the Invasion of Iraq.

\paragraph{SA-11 Squad}

This is a squad with a single 9K37 Buk (SA-11 Gadfly) vehicle with four 9M37 missiles and an integrated 9S35 (Fire Dome) TTR. Three squads make up a battery, and three batteries make up a battalion. Normally, one armored 9S18 (Tube Arm) or 9S18M1 (Snow Drift) EWR is allocated to each battalion.

The SA-11 has seen combat in the War in Abkhazia, the Russo-Ukrainian War, and during the Syrian Civil War.

\subsubsection{SA-13 Squad}

This is a single 9K35 Strela-10 (SA-13 Gopher) vehicle with four infra-red homing 9M37 missiles. Four squads form a platoon.
It began to replace the SA-9B in the air-defense platoons of tank and motor-rifle regiments in the Soviet Army starting in 1976 and was subsequently widely exported to Soviet allies.
Compared to the earlier system, the missile offers greater lethality, and the vehicle greater protection and mobility (although it is not amphibious).
Some are equipped with 9S16 (Flat Box-B) RWR.

The SA-13 saw combat in the South African Border War with Angolan forces, the Gulf War with Iraqi forces, and in the Kosovo War with Serbian forces.

\subsection{US SAM Launcher Units}

\subsubsection{M48 Squad}

This is a single M48 Chaparral vehicle with four adapted Sidewinder infrared-homing missiles. It was developed after the ambitious Mauler SAM program was abandoned and introduced in 1969. Night IR sights were added in 1984 to give it a limited night capability. The MIM-72A was the original missile based on the AIM-9D. The MIM-72C added an improved seeker and warhead, and the MIM-72E/F added a smokeless motor. The final MIM-72G added a seeker based on that of the FIM-92B Stinger, with much improved rejection of infrared countermeasures.

\subsubsection{I-HAWK Platoon}

This consists of three triple launchers for the MIM-23A/C/D missile, plus supporting radars and command elements.
The MIM-23B Improved HAWK and its associated improved radars were developed from the original HAWK MIM-23A to give much improved performance at low levels.
The MIM-23C has improved ECCM, and the MIM-23D adds a home-on-jam capability.
Two platoons with an EWR make up a battery.

The I-HAWK was deployed by the US Army, US Marines, and many US allies.

\subsubsection{MIM-104 Battery}

An MIM-104 Patriot battery consists of six launchers with associated radar and command elements.
The MIM-104A uses the “Standard” missile and was capable only of defense against aircraft.
Patriot is used by the US Army and numerous US allies.

\end{comment}

\begin{figure*}[p]
    \centering
    \includegraphics[width=0.8\linewidth]{figure-infantry-sam-units.png}
    \caption{Air-Defense Units: Infantry SAM Units}
    \label{figure:infantry-sam-units}
\end{figure*}

\begin{table*}[p]
    \caption{Air-Defense Units: Infantry SAM Units}
    \centering
    \footnotesize
    \begin{tabular}{lcccccclcccccccccl}
\toprule

            Name&
            \vertical{Size}      &
            \vertical{Year}      &
            \vertical{Defense Strength}&
            \vertical{Sighting Range}&
            \vertical{Mobility}&
            \begin{tabular}[b]{@{}c@{}}\vertical{VPs}\\\midrule3D/2D/D\\\end{tabular}&
            \wbox[r]{00}{\vertical{SAM}}          &
            \multicolumn{3}{c}{\begin{tabular}[b]{@{}ccc@{}}\multicolumn{3}{@{}c@{}}{\vertical{Missiles}}\\\midrule\wbox[c]{00}{\vertical{\wbox[l]{Optical}{Ready}}}&\wbox[c]{00}{\vertical{Volley}}&\wbox[c]{00}{\vertical{Reload}}\end{tabular}}&
            \vertical{Multi-Target}&
            \vertical{Quick Reaction}&
            \vertical{IFF}&
            \vertical{Night IR Sights}&
            %\vertical{Optical}      &
            %\vertical{Range}      &
            \multicolumn{2}{c}{\begin{tabular}[b]{@{}cc@{}}\multicolumn{2}{@{}c@{}}{\vertical{Lock-On}}\\\midrule\wbox[c]{00}{\vertical{Optical}}&\wbox[c]{00}{\vertical{Range}}\end{tabular}}&
            \wbox[r]{00}{\vertical{Other Names}}\\
            
\midrule
\addlinespace
SA-7A &\wbox[l]{Section}{Squad}&1968&\wbox[c]{00}{5}&\wbox[c]{00}{6}&\wbox[c]{00}{U}&\wbox{---/---/00}{\wbox[c]{0}{--}/\wbox[c]{0}{--}/3}&SA-7A&\wbox[c]{00}{1}&\wbox[c]{00}{1}&\wbox[c]{00}{2}&\wbox[c]{00}{---}&\wbox[c]{00}{Y}&\wbox[c]{00}{---}&\wbox[c]{00}{---}&\wbox[c]{00}{7}&\wbox[r]{00}{9}&9K32 Strela-2 (Grail)\\
SA-7B &\wbox[l]{Section}{Squad}&1972&\wbox[c]{00}{5}&\wbox[c]{00}{6}&\wbox[c]{00}{U}&\wbox{---/---/00}{\wbox[c]{0}{--}/\wbox[c]{0}{--}/4}&SA-7B&\wbox[c]{00}{1}&\wbox[c]{00}{1}&\wbox[c]{00}{2}&\wbox[c]{00}{---}&\wbox[c]{00}{Y}&\wbox[c]{00}{---}&\wbox[c]{00}{---}&\wbox[c]{00}{7}&\wbox[r]{00}{9}&9K32M Strela-2M (Grail)\\
SA-14 &\wbox[l]{Section}{Squad}&1974&\wbox[c]{00}{5}&\wbox[c]{00}{6}&\wbox[c]{00}{U}&\wbox{---/---/00}{\wbox[c]{0}{--}/\wbox[c]{0}{--}/6}&SA-14&\wbox[c]{00}{1}&\wbox[c]{00}{1}&\wbox[c]{00}{2}&\wbox[c]{00}{---}&\wbox[c]{00}{Y}&\wbox[c]{00}{---}&\wbox[c]{00}{---}&\wbox[c]{00}{7}&\wbox[r]{00}{12}&9K34 Strela-3 (Gremlin)\\
\addlinespace
SA-16 &\wbox[l]{Section}{Squad}&1981&\wbox[c]{00}{5}&\wbox[c]{00}{6}&\wbox[c]{00}{U}&\wbox{---/---/00}{\wbox[c]{0}{--}/\wbox[c]{0}{--}/8}&SA-16&\wbox[c]{00}{1}&\wbox[c]{00}{1}&\wbox[c]{00}{2}&\wbox[c]{00}{---}&\wbox[c]{00}{Y}&\wbox[c]{00}{Y}&\wbox[c]{00}{---}&\wbox[c]{00}{7}&\wbox[r]{00}{12}&9K310 Igla-1 (Gimlet)\\
SA-18 &\wbox[l]{Section}{Squad}&1983&\wbox[c]{00}{5}&\wbox[c]{00}{6}&\wbox[c]{00}{U}&\wbox{---/---/00}{\wbox[c]{0}{--}/\wbox[c]{0}{--}/10}&SA-18&\wbox[c]{00}{1}&\wbox[c]{00}{1}&\wbox[c]{00}{2}&\wbox[c]{00}{---}&\wbox[c]{00}{Y}&\wbox[c]{00}{Y}&\wbox[c]{00}{---}&\wbox[c]{00}{8}&\wbox[r]{00}{16}&9K38 Igla (Grouse)\\
\addlinespace
FIM-43C &\wbox[l]{Section}{Squad}&1968&\wbox[c]{00}{5}&\wbox[c]{00}{6}&\wbox[c]{00}{U}&\wbox{---/---/00}{\wbox[c]{0}{--}/\wbox[c]{0}{--}/4}&FIM-43C&\wbox[c]{00}{1}&\wbox[c]{00}{1}&\wbox[c]{00}{2}&\wbox[c]{00}{---}&\wbox[c]{00}{Y}&\wbox[c]{00}{---}&\wbox[c]{00}{---}&\wbox[c]{00}{7}&\wbox[r]{00}{9}&Redeye Block III\\
\addlinespace
FIM-92A &\wbox[l]{Section}{Squad}&1981&\wbox[c]{00}{5}&\wbox[c]{00}{6}&\wbox[c]{00}{U}&\wbox{---/---/00}{\wbox[c]{0}{--}/\wbox[c]{0}{--}/9}&FIM-92A&\wbox[c]{00}{1}&\wbox[c]{00}{1}&\wbox[c]{00}{2}&\wbox[c]{00}{---}&\wbox[c]{00}{Y}&\wbox[c]{00}{Y}&\wbox[c]{00}{---}&\wbox[c]{00}{8}&\wbox[r]{00}{12}&Stinger\\
FIM-92B &\wbox[l]{Section}{Squad}&1986&\wbox[c]{00}{5}&\wbox[c]{00}{6}&\wbox[c]{00}{U}&\wbox{---/---/00}{\wbox[c]{0}{--}/\wbox[c]{0}{--}/10}&FIM-92B&\wbox[c]{00}{1}&\wbox[c]{00}{1}&\wbox[c]{00}{2}&\wbox[c]{00}{---}&\wbox[c]{00}{Y}&\wbox[c]{00}{Y}&\wbox[c]{00}{---}&\wbox[c]{00}{8}&\wbox[r]{00}{12}&Stinger POST\\
FIM-92C &\wbox[l]{Section}{Squad}&1989&\wbox[c]{00}{5}&\wbox[c]{00}{6}&\wbox[c]{00}{U}&\wbox{---/---/00}{\wbox[c]{0}{--}/\wbox[c]{0}{--}/10}&FIM-92C&\wbox[c]{00}{1}&\wbox[c]{00}{1}&\wbox[c]{00}{2}&\wbox[c]{00}{---}&\wbox[c]{00}{Y}&\wbox[c]{00}{Y}&\wbox[c]{00}{---}&\wbox[c]{00}{8}&\wbox[r]{00}{12}&Stinger RMP\\
FIM-92D &\wbox[l]{Section}{Squad}&1992&\wbox[c]{00}{5}&\wbox[c]{00}{6}&\wbox[c]{00}{U}&\wbox{---/---/00}{\wbox[c]{0}{--}/\wbox[c]{0}{--}/10}&FIM-92D&\wbox[c]{00}{1}&\wbox[c]{00}{1}&\wbox[c]{00}{2}&\wbox[c]{00}{---}&\wbox[c]{00}{Y}&\wbox[c]{00}{Y}&\wbox[c]{00}{---}&\wbox[c]{00}{8}&\wbox[r]{00}{12}&Stinger RMP\\
\addlinespace
\midrule
\addlinespace
Blowpipe &\wbox[l]{Section}{Squad}&1975&\wbox[c]{00}{5}&\wbox[c]{00}{6}&\wbox[c]{00}{U}&\wbox{---/---/00}{\wbox[c]{0}{--}/\wbox[c]{0}{--}/6}&Blowpipe&\wbox[c]{00}{1}&\wbox[c]{00}{1}&\wbox[c]{00}{2}&\wbox[c]{00}{---}&\wbox[c]{00}{Y}&\wbox[c]{00}{Y}&\wbox[c]{00}{---}&\wbox[c]{00}{7}&\wbox[r]{00}{9}&\\
Javelin &\wbox[l]{Section}{Squad}&1984&\wbox[c]{00}{5}&\wbox[c]{00}{6}&\wbox[c]{00}{U}&\wbox{---/---/00}{\wbox[c]{0}{--}/\wbox[c]{0}{--}/8}&Javelin&\wbox[c]{00}{1}&\wbox[c]{00}{1}&\wbox[c]{00}{2}&\wbox[c]{00}{---}&\wbox[c]{00}{Y}&\wbox[c]{00}{Y}&\wbox[c]{00}{---}&\wbox[c]{00}{7}&\wbox[r]{00}{9}&Javelin GL\\
Starburst &\wbox[l]{Section}{Squad}&1989&\wbox[c]{00}{5}&\wbox[c]{00}{6}&\wbox[c]{00}{U}&\wbox{---/---/00}{\wbox[c]{0}{--}/\wbox[c]{0}{--}/8}&Starburst&\wbox[c]{00}{1}&\wbox[c]{00}{1}&\wbox[c]{00}{2}&\wbox[c]{00}{---}&\wbox[c]{00}{Y}&\wbox[c]{00}{Y}&\wbox[c]{00}{---}&\wbox[c]{00}{7}&\wbox[r]{00}{9}&Javelin S15\\
\addlinespace
Starstreak &\wbox[l]{Section}{Squad}&2000&\wbox[c]{00}{5}&\wbox[c]{00}{6}&\wbox[c]{00}{U}&\wbox{---/---/00}{\wbox[c]{0}{--}/\wbox[c]{0}{--}/12}&Starstreak&\wbox[c]{00}{1}&\wbox[c]{00}{1}&\wbox[c]{00}{2}&\wbox[c]{00}{---}&\wbox[c]{00}{Y}&\wbox[c]{00}{Y}&\wbox[c]{00}{---}&\wbox[c]{00}{7}&\wbox[r]{00}{12}&\\
Starstreak LML &\wbox[l]{Section}{Squad}&2000&\wbox[c]{00}{4}&\wbox[c]{00}{6}&\wbox[c]{00}{U}&\wbox{---/---/00}{\wbox[c]{0}{--}/\wbox[c]{0}{--}/14}&Starstreak&\wbox[c]{00}{3}&\wbox[c]{00}{1}&\wbox[c]{00}{0}&\wbox[c]{00}{---}&\wbox[c]{00}{Y}&\wbox[c]{00}{Y}&\wbox[c]{00}{---}&\wbox[c]{00}{7}&\wbox[r]{00}{12}&\\
\addlinespace
Mistral &\wbox[l]{Section}{Squad}&1990&\wbox[c]{00}{4}&\wbox[c]{00}{6}&\wbox[c]{00}{U}&\wbox{---/---/00}{\wbox[c]{0}{--}/\wbox[c]{0}{--}/11}&Mistral&\wbox[c]{00}{1}&\wbox[c]{00}{1}&\wbox[c]{00}{2}&\wbox[c]{00}{---}&\wbox[c]{00}{Y}&\wbox[c]{00}{---}&\wbox[c]{00}{---}&\wbox[c]{00}{7}&\wbox[r]{00}{12}&\\
\addlinespace
RBS 70 &\wbox[l]{Section}{Squad}&1977&\wbox[c]{00}{4}&\wbox[c]{00}{6}&\wbox[c]{00}{U}&\wbox{---/---/00}{\wbox[c]{0}{--}/\wbox[c]{0}{--}/8}&RBS 70&\wbox[c]{00}{1}&\wbox[c]{00}{1}&\wbox[c]{00}{2}&\wbox[c]{00}{---}&\wbox[c]{00}{Y}&\wbox[c]{00}{Y}&\wbox[c]{00}{---}&\wbox[c]{00}{7}&\wbox[r]{00}{10}&\\
\addlinespace
Ayn-al-Saqr &\wbox[l]{Section}{Squad}&1984&\wbox[c]{00}{5}&\wbox[c]{00}{6}&\wbox[c]{00}{U}&\wbox{---/---/00}{\wbox[c]{0}{--}/\wbox[c]{0}{--}/4}&Ayn-al-Saqr&\wbox[c]{00}{1}&\wbox[c]{00}{1}&\wbox[c]{00}{2}&\wbox[c]{00}{---}&\wbox[c]{00}{Y}&\wbox[c]{00}{---}&\wbox[c]{00}{---}&\wbox[c]{00}{7}&\wbox[r]{00}{12}&\\
\addlinespace
\bottomrule
\end{tabular}

    \label{table:infantry-sam-units}
\end{table*}

\begin{comment}
\begin{table*}[p]
    \caption{Air-Defense Units: SAM Launcher Units}
    \centering
    \footnotesize
    \begin{tabular}{lcccccclccccccccccccccl}
\toprule

            Name&
            \vertical{Size}      &
            \vertical{Year}      &
            \vertical{Defense Strength}&
            \vertical{Sighting Range}&
            \vertical{Mobility}&
            \begin{tabular}[b]{@{}c@{}}\vertical{VPs}\\\midrule3D/2D/D\\\end{tabular}&
            \wbox[r]{00}{\vertical{SAM}}&
            %\vertical{Frequency}    &
            %\vertical{Range}        &
            %\vertical{MTI}          &
            \multicolumn{3}{c}{\begin{tabular}[b]{@{}ccc@{}}\multicolumn{3}{@{}c@{}}{\vertical{EWR}}\\\midrule\wbox[c]{MW}{\vertical{Frequency}}&\wbox[c]{00}{\vertical{Range}}&\wbox[c]{---}{\vertical{MTI}}\end{tabular}}&
            %\vertical{Frequency}    &
            %\vertical{Range}        &
            \multicolumn{2}{c}{\begin{tabular}[b]{@{}cc@{}}\multicolumn{2}{@{}c@{}}{\vertical{TTR}}\\\midrule\wbox[c]{MW}{\vertical{Frequency}}&\wbox[c]{00}{\vertical{Range}}\end{tabular}}&
            \multicolumn{3}{c}{\begin{tabular}[b]{@{}ccc@{}}\multicolumn{3}{@{}c@{}}{\vertical{Missiles}}\\\midrule\wbox[c]{00}{\vertical{\wbox[l]{Frequency}{Ready}}}&\vertical{Volley}&\vertical{Reload}\end{tabular}}&
            \vertical{Multi-Target}&
            \vertical{Quick Reaction}&
            \vertical{Night IR Sights}&
            %\vertical{Radar}        &
            %\vertical{Optical}      &
            %\vertical{Range}        &
            \multicolumn{3}{c}{\begin{tabular}[b]{@{}ccc@{}}\multicolumn{3}{@{}c@{}}{\vertical{Lock-On}}\\\midrule\vertical{\wbox[l]{Frequency}{Radar}}&\vertical{Optical}&\wbox[c]{00}{\vertical{Range}}\end{tabular}}&
            \wbox[r]{00}{\vertical{Other Names}}\\
            
\midrule
\addlinespace
SA-2B &\wbox[l]{Section}{Battery}&1957&\wbox[l]{0}{4}&\wbox{00}{18}&G/P&\wbox{00/00/00}{8/5/3}&SA-2B&\wbox[l]{MW}{LF}&\wbox{00}{30}&---&\wbox[l]{MW}{LF}&\wbox{00}{20}&\wbox{00}{6}&\wbox{0}{2}&\wbox{0}{0}&---&---&---&6&---&\wbox{00}{---}&S-75 Desna (Guideline)\\
SA-2C &\wbox[l]{Section}{Battery}&1960&\wbox[l]{0}{4}&\wbox{00}{18}&G/P&\wbox{00/00/00}{8/5/3}&SA-2C&\wbox[l]{MW}{LF}&\wbox{00}{30}&---&\wbox[l]{MW}{MF}&\wbox{00}{20}&\wbox{00}{6}&\wbox{0}{2}&\wbox{0}{0}&---&---&---&6&---&\wbox{00}{---}&S-75M Volkhov (Guideline)\\
SA-2E &\wbox[l]{Section}{Battery}&1962&\wbox[l]{0}{4}&\wbox{00}{18}&G/P&\wbox{00/00/00}{8/5/3}&SA-2E&\wbox[l]{MW}{LF}&\wbox{00}{30}&---&\wbox[l]{MW}{MF}&\wbox{00}{20}&\wbox{00}{6}&\wbox{0}{2}&\wbox{0}{0}&---&---&---&6&---&\wbox{00}{---}&S-75AK (Guideline)\\
SA-2F &\wbox[l]{Section}{Battery}&1967&\wbox[l]{0}{4}&\wbox{00}{18}&G/P&\wbox{00/00/00}{8/5/3}&SA-2F&\wbox[l]{MW}{LF}&\wbox{00}{30}&---&\wbox[l]{MW}{LF}&\wbox{00}{20}&\wbox{00}{6}&\wbox{0}{2}&\wbox{0}{0}&---&---&---&6&6&\wbox{00}{---}&S-75SM (Guideline)\\
\addlinespace
SA-3B &\wbox[l]{Section}{Battery}&1964&\wbox[l]{0}{4}&\wbox{00}{18}&G/P&\wbox{00/00/00}{10/7/3}&SA-3B&\wbox[l]{MW}{MF}&\wbox{00}{35}&---&\wbox[l]{MW}{VF}&\wbox{00}{10}&\wbox{00}{16}&\wbox{0}{2}&\wbox{0}{0}&---&---&---&7&6&\wbox{00}{---}&S-125M Neva-M (Goa)\\
Mobile SA-3B &\wbox[l]{Section}{Battery}&1964&\wbox[l]{0}{4}&\wbox{00}{18}&G/P&\wbox{00/00/00}{12/8/4}&SA-3B&\wbox[l]{MW}{MF}&\wbox{00}{35}&---&\wbox[l]{MW}{VF}&\wbox{00}{10}&\wbox{00}{16}&\wbox{0}{2}&\wbox{0}{0}&---&---&---&7&6&\wbox{00}{---}&S-125M Neva-M (Goa)\\
\addlinespace
SA-4A &\wbox[l]{Section}{Battery}&1963&\wbox[l]{0}{\underline{3}}&\wbox{00}{12}&U&\wbox{00/00/00}{10/7/3}&SA-4A&\wbox[l]{MW}{VF}&\wbox{00}{50}&---&\wbox[l]{MW}{HF}&\wbox{00}{30}&\wbox{00}{6}&\wbox{0}{2}&\wbox{0}{0}&---&---&---&7&---&\wbox{00}{---}&2K11 Krug (Ganef)\\
SA-4B &\wbox[l]{Section}{Battery}&1973&\wbox[l]{0}{\underline{3}}&\wbox{00}{12}&U&\wbox{00/00/00}{10/7/3}&SA-4B&\wbox[l]{MW}{VF}&\wbox{00}{50}&---&\wbox[l]{MW}{HF}&\wbox{00}{30}&\wbox{00}{6}&\wbox{0}{2}&\wbox{0}{0}&---&---&---&7&---&\wbox{00}{---}&2K11M Krug-M (Ganef)\\
\addlinespace
SA-6A &\wbox[l]{Section}{Battery}&1966&\wbox[l]{0}{\underline{3}}&\wbox{00}{12}&U&\wbox{00/00/00}{11/7/4}&SA-6A&\wbox[l]{MW}{MF}&\wbox{00}{24}&---&\wbox[l]{MW}{HF}&\wbox{00}{18}&\wbox{00}{12}&\wbox{0}{2}&\wbox{0}{0}&---&---&---&7&6&\wbox{00}{---}&2K12 Kub (Gainful)\\
SA-6B &\wbox[l]{Section}{Battery}&1977&\wbox[l]{0}{\underline{3}}&\wbox{00}{12}&U&\wbox{00/00/00}{12/8/4}&SA-6B&\wbox[l]{MW}{MF}&\wbox{00}{24}&---&\wbox[l]{MW}{HF}&\wbox{00}{18}&\wbox{00}{12}&\wbox{0}{2}&\wbox{0}{0}&---&---&---&7&6&\wbox{00}{---}&2K12M3 Kub-M3 (Gainful)\\
\addlinespace
SA-8A &\wbox[l]{Section}{Squad}&1967&\wbox[l]{0}{\underline{3}}&\wbox{00}{12}&U&\wbox{00/00/00}{\wbox[c]{0}{--}/\wbox[c]{0}{--}/10}&SA-8A&\wbox[l]{MW}{HF}&\wbox{00}{10}&---&\wbox[l]{MW}{VF}&\wbox{00}{6}&\wbox{00}{4}&\wbox{0}{2}&\wbox{0}{0}&2&Y&---&8&---&\wbox{00}{---}&9K33 Osa (Gecko)\\
SA-8B &\wbox[l]{Section}{Squad}&1980&\wbox[l]{0}{\underline{3}}&\wbox{00}{12}&U&\wbox{00/00/00}{\wbox[c]{0}{--}/\wbox[c]{0}{--}/10}&SA-8B&\wbox[l]{MW}{HF}&\wbox{00}{10}&---&\wbox[l]{MW}{VF}&\wbox{00}{6}&\wbox{00}{6}&\wbox{0}{2}&\wbox{0}{0}&2&Y&---&8&7&\wbox{00}{---}&9K33M2 Osa-AK (Gecko)\\
\addlinespace
SA-9A &\wbox[l]{Section}{Squad}&1968&\wbox[l]{0}{\underline{4}}&\wbox{00}{12}&U&\wbox{00/00/00}{\wbox[c]{0}{--}/\wbox[c]{0}{--}/7}&SA-9A&\wbox[l]{MW}{---}&\wbox{00}{---}&---&\wbox[l]{MW}{---}&\wbox{00}{---}&\wbox{00}{4}&\wbox{0}{2}&\wbox{0}{0}&2&Y&---&---&7&\wbox{00}{12}&9K31 Strela-1 (Gaskin)\\
SA-9B &\wbox[l]{Section}{Squad}&1970&\wbox[l]{0}{\underline{4}}&\wbox{00}{12}&U&\wbox{00/00/00}{\wbox[c]{0}{--}/\wbox[c]{0}{--}/7}&SA-9B&\wbox[l]{MW}{---}&\wbox{00}{---}&---&\wbox[l]{MW}{---}&\wbox{00}{---}&\wbox{00}{4}&\wbox{0}{2}&\wbox{0}{0}&2&Y&---&---&7&\wbox{00}{12}&9K31M Strela-1M (Gaskin)\\
\addlinespace
SA-11 &\wbox[l]{Section}{Squad}&1980&\wbox[l]{0}{\underline{4}}&\wbox{00}{12}&U&\wbox{00/00/00}{\wbox[c]{0}{--}/\wbox[c]{0}{--}/12}&SA-11&\wbox[l]{MW}{---}&\wbox{00}{---}&---&\wbox[l]{MW}{HF}&\wbox{00}{18}&\wbox{00}{4}&\wbox{0}{2}&\wbox{0}{0}&2&---&---&8&7&\wbox{00}{---}&9K37 Buk (Gadfly)\\
\addlinespace
SA-13 &\wbox[l]{Section}{Squad}&1976&\wbox[l]{0}{\underline{4}}&\wbox{00}{12}&U&\wbox{00/00/00}{\wbox[c]{0}{--}/\wbox[c]{0}{--}/10}&SA-13&\wbox[l]{MW}{---}&\wbox{00}{---}&---&\wbox[l]{MW}{---}&\wbox{00}{---}&\wbox{00}{4}&\wbox{0}{2}&\wbox{0}{0}&2&Y&---&---&8&\wbox{00}{12}&9K35 Strela-10 (Gopher)\\
\addlinespace
M48 &\wbox[l]{Section}{Squad}&1969&\wbox[l]{0}{4}&\wbox{00}{12}&U&\wbox{00/00/00}{\wbox[c]{0}{--}/\wbox[c]{0}{--}/9}&MIM-72&\wbox[l]{MW}{---}&\wbox{00}{---}&---&\wbox[l]{MW}{---}&\wbox{00}{---}&\wbox{00}{4}&\wbox{0}{2}&\wbox{0}{0}&2&Y&Y&---&7&\wbox{00}{12}&Chaparral\\
\addlinespace
I-Hawk &\wbox[l]{Section}{Platoon}&1971&\wbox[l]{0}{3}&\wbox{00}{12}&G/P&\wbox{00/00/00}{12/8/4}&MIM-23B&\wbox[l]{MW}{---}&\wbox{00}{---}&---&\wbox[l]{MW}{VF}&\wbox{00}{20}&\wbox{00}{9}&\wbox{0}{2}&\wbox{0}{0}&---&Y&---&7&7&\wbox{00}{---}&\\
\addlinespace
MIM-104A &\wbox[l]{Section}{Battery}&1984&\wbox[l]{0}{3}&\wbox{00}{12}&G/P&\wbox{00/00/00}{45/30/15}&MIM-104A&\wbox[l]{MW}{LF}&\wbox{00}{60}&Y&\wbox[l]{MW}{LF}&\wbox{00}{40}&\wbox{00}{24}&\wbox{0}{8}&\wbox{0}{0}&8&Y&---&9&---&\wbox{00}{---}&Patriot (Standard)\\
\addlinespace
\addlinespace
\bottomrule
\end{tabular}

    \label{table:sam-launcher-units}
\end{table*}
\end{comment}

\begin{table*}[p]
    \caption{SAMs}
    \centering
    \footnotesize
    \begin{tabular}{lccccccccccccccccc@{~}cc@{~}cl}
\toprule

            Name&
            \vertical{Year}&
            \vertical{Guidance Mode}&
            \vertical{Launch Roll}&
            \vertical{Turn Rate}&
            \vertical{Flight Time}&
            \vertical{Visibility}&
            \vertical{ECCM}&
            \vertical{Chaff}&
            \vertical{Flare}&
            \vertical{AH}&
            \vertical{HOJ}&
            \vertical{Instant Arming}&
            \vertical{Boost Phase}&
            \vertical{Base Speed}&
            \vertical{Sustainer}&
            \vertical{Minimum Altitude}&
            \multicolumn{2}{c}{\begin{tabular}[b]{@{}cc@{}}\multicolumn{2}{@{}c@{}}{\vertical{Hit}}\\\midrule\wbox[c]{00}{\vertical{Direct}}&\wbox[c]{00}{\vertical{Proximity}}\end{tabular}}&
            \multicolumn{2}{c}{\begin{tabular}[b]{@{}cc@{}}\multicolumn{2}{@{}c@{}}{\vertical{Damage}}\\\midrule\wbox[c]{00}{\vertical{Direct}}&\wbox[c]{00}{\vertical{Proximity}}\end{tabular}}&
            \wbox[r]{00}{\vertical{Other Names}}
            \\
            
\midrule
\addlinespace
SA-7A&1968&\wbox[c]{TVM}{I}&6&BT&1&\wbox{00}{6}&---&---&5&---&---&Y&\wbox{00}{---}&\wbox{00}{10}&0&T&\wbox{00}{5}&\wbox{00}{6}&\wbox{00*}{4\phantom{*}}&\wbox{00*}{2\phantom{*}}&9M32 Strela-2 (Grail)\\
SA-7B&1972&\wbox[c]{TVM}{I}&7&BT&1&\wbox{00}{6}&---&---&5&---&---&Y&\wbox{00}{---}&\wbox{00}{10}&0&T&\wbox{00}{5}&\wbox{00}{6}&\wbox{00*}{4\phantom{*}}&\wbox{00*}{2\phantom{*}}&9M32M Strela-2M (Grail)\\
SA-14&1974&\wbox[c]{TVM}{M}&7&BT&1&\wbox{00}{6}&---&---&4&---&---&Y&\wbox{00}{---}&\wbox{00}{9}&0&T&\wbox{00}{6}&\wbox{00}{7}&\wbox{00*}{4\phantom{*}}&\wbox{00*}{2\phantom{*}}&9M34 Strela-3 (Gremlin)\\
\addlinespace
SA-16&1981&\wbox[c]{TVM}{M}&8&BT/2&1&\wbox{00}{6}&---&---&4&---&---&Y&\wbox{00}{---}&\wbox{00}{12}&0&T&\wbox{00}{6}&\wbox{00}{7}&\wbox{00*}{4*}&\wbox{00*}{2*}&9K310 Igla-1 (Gimlet)\\
SA-18&1983&\wbox[c]{TVM}{A}&8&BT/2&1&\wbox{00}{6}&---&---&2&---&---&Y&\wbox{00}{---}&\wbox{00}{12}&0&T&\wbox{00}{7}&\wbox{00}{8}&\wbox{00*}{4*}&\wbox{00*}{2*}&9K38 Igla (Grouse)\\
\addlinespace
FIM-43C&1968&\wbox[c]{TVM}{M}&7&BT&1&\wbox{00}{6}&---&---&5&---&---&Y&\wbox{00}{---}&\wbox{00}{10}&0&T&\wbox{00}{6}&\wbox{00}{7}&\wbox{00*}{4\phantom{*}}&\wbox{00*}{2\phantom{*}}&Redeye Block III\\
\addlinespace
FIM-92A&1981&\wbox[c]{TVM}{A}&8&BT/2&1&\wbox{00}{2}&---&---&3&---&---&Y&\wbox{00}{---}&\wbox{00}{14}&0&T&\wbox{00}{7}&\wbox{00}{8}&\wbox{00*}{5\phantom{*}}&\wbox{00*}{3\phantom{*}}&Stinger\\
FIM-92B&1986&\wbox[c]{TVM}{A}&8&BT/2&1&\wbox{00}{2}&---&---&1&---&---&Y&\wbox{00}{---}&\wbox{00}{14}&0&T&\wbox{00}{7}&\wbox{00}{8}&\wbox{00*}{5\phantom{*}}&\wbox{00*}{3\phantom{*}}&Stinger POST\\
FIM-92C&1989&\wbox[c]{TVM}{A}&8&BT/2&1&\wbox{00}{2}&---&---&1&---&---&Y&\wbox{00}{---}&\wbox{00}{14}&0&T&\wbox{00}{7}&\wbox{00}{8}&\wbox{00*}{5\phantom{*}}&\wbox{00*}{3\phantom{*}}&Stinger RMP\\
FIM-92D&1992&\wbox[c]{TVM}{A}&8&BT/2&1&\wbox{00}{2}&---&---&1&---&---&Y&\wbox{00}{---}&\wbox{00}{14}&0&T&\wbox{00}{7}&\wbox{00}{8}&\wbox{00*}{5\phantom{*}}&\wbox{00*}{3\phantom{*}}&Stinger RMP\\
\addlinespace
\midrule
\addlinespace
\addlinespace
Blowpipe&1975&\wbox[c]{TVM}{OG}&8&HT&1&\wbox{00}{6}&---&3&0&---&---&---&\wbox{00}{---}&\wbox{00}{8}&0&T&\wbox{00}{3}&\wbox{00}{7}&\wbox{00*}{6\phantom{*}}&\wbox{00*}{5\phantom{*}}&\\
Javelin&1984&\wbox[c]{TVM}{OG}&8&ET&1&\wbox{00}{5}&---&2&0&---&---&Y&\wbox{00}{---}&\wbox{00}{10}&0&T&\wbox{00}{5}&\wbox{00}{8}&\wbox{00*}{6\phantom{*}}&\wbox{00*}{5\phantom{*}}&Javelin GL\\
Starburst&1989&\wbox[c]{TVM}{LG}&8&ET&1&\wbox{00}{5}&---&1&0&---&---&Y&\wbox{00}{---}&\wbox{00}{10}&0&T&\wbox{00}{5}&\wbox{00}{8}&\wbox{00*}{6\phantom{*}}&\wbox{00*}{5\phantom{*}}&Javelin S15\\
\addlinespace
Starstreak&2000&\wbox[c]{TVM}{LG}&8&ET&1&\wbox{00}{5}&---&1&0&---&---&Y&\wbox{00}{---}&\wbox{00}{24}&0&T&\wbox{00}{7}&\wbox{00}{9}&\wbox{00*}{6\phantom{*}}&\wbox{00*}{4\phantom{*}}&\\
\addlinespace
Mistral&1990&\wbox[c]{TVM}{A}&8&ET&1&\wbox{00}{5}&---&---&2&---&---&Y&\wbox{00}{---}&\wbox{00}{18}&0&T&\wbox{00}{7}&\wbox{00}{9}&\wbox{00*}{6\phantom{*}}&\wbox{00*}{3\phantom{*}}&\\
\addlinespace
RBS 70&1977&\wbox[c]{TVM}{LG}&8&ET&1&\wbox{00}{6}&---&2&0&---&---&---&\wbox{00}{---}&\wbox{00}{12}&0&T&\wbox{00}{5}&\wbox{00}{8}&\wbox{00*}{6\phantom{*}}&\wbox{00*}{3\phantom{*}}&\\
\addlinespace
Ayn-al-Saqr&1984&\wbox[c]{TVM}{I}&7&BT&1&\wbox{00}{6}&---&---&5&---&---&Y&\wbox{00}{---}&\wbox{00}{10}&0&T&\wbox{00}{5}&\wbox{00}{7}&\wbox{00*}{4\phantom{*}}&\wbox{00*}{3\phantom{*}}&\\
\addlinespace
\bottomrule
\end{tabular}

    \label{table:sams}
\end{table*}

\clearpage
\section{EWR and CCU Units}

\begin{comment}

An early-warning radar (EWR) can detect aircraft at longer ranges than the target acquisition radars typically associated with SAM units. It can then pass this information down to SAMs in their integrated air-defense system, sometimes with the intermediation of a CCU, to improve the chance of a target acquisition and lock-on.

A command-and-control unit (CCU) acts to coordinate multiple independent SAM firing units in an integrated air-defense system. For early SAMs, where each battery can engage one target, CCUs are typically at the regimental level (i.e., the formation above the battery). For later SAMs, where each launcher can engage one or more targets, they are typically at the battery level.

Figure~\ref{figure:radar-and-ccu-units} shows the representation of EWR and CCU units as counters, and Table~\ref{table:ccu-and-ewr-units} summarizes their properties.

\paragraph{Properties.} In addition to the properties common to other ground units, Table~\ref{table:ccu-and-ewr-units} gives: the EWR class, its frequency band, and its range in mega-hexes.

\paragraph{EWR-A.}
\paragraph{EWR-B.}

\paragraph{Mobile CCU.}

\paragraph{Armored CCU.}

\paragraph{Crotale CCU/EWR.} This is the HQ and EWR vehicle of a Crotale SAM battery. It is equipped with a Mirador IV radar.

\begin{table*}[p]
    \caption{Air-Defense Units: EWR and CCU Units}
    \centering
    \footnotesize
    \begin{tabular}{lccccccccccc}
\toprule

            Type&
            \vertical{Size}&
            \vertical{Defense Strength}&
            \vertical{Sighting Range}&
            \vertical{Mobility}&
            \begin{tabular}[b]{@{}c@{}}\vertical{VPs}\\\midrule3D/2D/D\\\end{tabular}&
            \multicolumn{3}{c}{\begin{tabular}[b]{@{}ccc@{}}\multicolumn{3}{@{}c@{}}{\vertical{EWR}}\\\midrule\wbox[c]{---}{\vertical{Class}}&\wbox[c]{LF}{\vertical{Frequency}}&\wbox[c]{\minus{---}}{\vertical{Range}}\end{tabular}}
            \\
        
\midrule
\addlinespace
Mobile EWR-A &\wbox[l]{Section}{Platoon}&\wbox[l]{0}{2}&\wbox{00}{12}&G/P&\wbox{00/00/00}{12/8/4}&\wbox[l]{D}{A}&\wbox[l]{MW}{LF}&\wbox[r]{}{---}\\
Mobile EWR-B &\wbox[l]{Section}{Platoon}&\wbox[l]{0}{2}&\wbox{00}{12}&G/P&\wbox{00/00/00}{12/8/4}&\wbox[l]{D}{B}&\wbox[l]{MW}{LF}&\wbox[r]{}{---}\\
Mobile CCU &\wbox[l]{Section}{Platoon}&\wbox[l]{0}{2}&\wbox{00}{12}&G/P&\wbox{00/00/00}{8/5/3}&\wbox[l]{D}{---}&\wbox[l]{MW}{---}&\wbox[r]{}{---}\\
Armored CCU &\wbox[l]{Section}{Squad}&\wbox[l]{0}{\underline{4}}&\wbox{00}{12}&U&\wbox{00/00/00}{\wbox[c]{0}{--}/\wbox[c]{0}{--}/8}&\wbox[l]{D}{---}&\wbox[l]{MW}{---}&\wbox[r]{}{---}\\
Crotale CCU/EWR &\wbox[l]{Section}{Squad}&\wbox[l]{0}{\underline{4}}&\wbox{00}{12}&U&\wbox{00/00/00}{\wbox[c]{0}{--}/\wbox[c]{0}{--}/10}&\wbox[l]{D}{B}&\wbox[l]{MW}{LF}&\wbox[r]{}{8}\\
\addlinespace
\bottomrule
\end{tabular}

    \label{table:ccu-and-ewr-units}
\end{table*}

\end{comment}

\clearpage
\section{Ground Targets}

A ground \emph{target} is a unit of infrastructure such as a bridge and a building.

\subsection{Properties and Representation}

Ground targets are listed with their properties in Table~\ref{table:ground-targets}.

All ground targets have these properties:
\begin{itemize}
    \item defense strength;
    \item defense strength and target class;
    \item sighting range in hexes; and
    \item the VPs awarded for 3D/2D/D damage.
\end{itemize}

All ground targets have a damage resilience of 3D. That is, they are killed when they suffer 3D or more damage.

Some ground targets are represented by counters, and others are represented by features on the maps. Figure~\ref{figure:ground-targets} shows the graphical representation of the counters, which mainly uses adapted NATO symbols.

\subsection{Notes on Specific Targets}

\subsubsection{Aircraft}

No VP values are given for transport or fighter airplanes or helicopters, since these can vary significantly with the capacity of the aircraft. A scenario must specify the VP values. As a guideline, they will often be similar to the VP values of the attacking aircraft.

\subsubsection{Bridges}

Unless otherwise indicated in a scenario, a major bridge is one that extends over two or more hexes, a minor bridge is one that is confined to a single hex, and a small bridge is where a road crosses a river without a bridge being marked on the map.

\subsubsection{Penetrating Bombs}

When attacking a bunker entrance, a tunnel entrance hex, or a shelter, the attack strength of a penetrating bomb is doubled.

\begin{figure*}[p]
    \centering
    \includegraphics[width=0.8\linewidth]{figure-ground-targets.png}
    \caption{Ground Targets}
    \label{figure:ground-targets}
\end{figure*}

\onecolumn

\begin{table*}[p]
    \caption{Ground Targets}
    \centering
    \footnotesize
    \begin{tabular}{lcccc}
\toprule

            Type&
            \vertical{Counter?}&
            \vertical{Defense Strength}&
            \vertical{Sighting Range}&
            \begin{tabular}[b]{@{}c@{}}\vertical{VPs}\\\midrule3D/2D/D\\\end{tabular}
            \\
        
\midrule
\addlinespace
Locomotive&Y&\wbox{00}{2}&\wbox{00}{18}&\wbox{00/00/00}{12/8/4}\\
Railcar&Y&\wbox{00}{2}&\wbox{00}{18}&\wbox{00/00/00}{5/3/2}\\
\addlinespace
POL&Y&\wbox{00}{4}&\wbox{00}{24}&\wbox{00/00/00}{15/10/5}\\
Supplies&Y&\wbox{00}{4}&\wbox{00}{24}&\wbox{00/00/00}{15/10/5}\\
\addlinespace
Small Building&Y&\wbox{00}{3}&\wbox{00}{18}&\wbox{00/00/00}{4/3/1}\\
Large Building&Y&\wbox{00}{6}&\wbox{00}{24}&\wbox{00/00/00}{7/5/2}\\
Factory&Y&\wbox{00}{8}&\wbox{00}{18}&\wbox{00/00/00}{20/13/7}\\
Power Station&Y&\wbox{00}{5}&\wbox{00}{18}&\wbox{00/00/00}{12/8/4}\\
Bunker Entrance&Y&\wbox{00}{\underline{20}}&\wbox{00}{12}&\wbox{00/00/00}{30/20/10}\\
\addlinespace
Barge&Y&\wbox{00}{2}&\wbox{00}{12}&\wbox{00/00/00}{4/3/1}\\
Junk&Y&\wbox{00}{2}&\wbox{00}{12}&\wbox{00/00/00}{4/3/1}\\
\addlinespace
Hanger&Y&\wbox{00}{5}&\wbox{00}{24}&\wbox{00/00/00}{8/5/3}\\
Shelter&Y&\wbox{00}{\underline{12}}&\wbox{00}{18}&\wbox{00/00/00}{12/8/4}\\
Tower&Y&\wbox{00}{3}&\wbox{00}{18}&\wbox{00/00/00}{5/3/2}\\
Transport Airplane&Y&\wbox{00}{3}&\wbox{00}{18}\\
Fighter Airplane&Y&\wbox{00}{3}&\wbox{00}{12}\\
Helicopter&Y&\wbox{00}{3}&\wbox{00}{12}\\
\addlinespace
Railroad Hex&N&\wbox{00}{6}&\wbox{00}{24}&\wbox{00/00/00}{5/3/2}\\
Railyard Hex&N&\wbox{00}{10}&\wbox{00}{24}&\wbox{00/00/00}{12/8/4}\\
Docks Hex&N&\wbox{00}{\underline{10}}&\wbox{00}{36}&\wbox{00/00/00}{8/5/3}\\
Piers Hex&N&\wbox{00}{\underline{10}}&\wbox{00}{36}&\wbox{00/00/00}{8/5/3}\\
Road Hex&N&\wbox{00}{8}&\wbox{00}{24}&\wbox{00/00/00}{2/1/0}\\
Trail Hex&N&\wbox{00}{6}&\wbox{00}{12}\\
Tunnel Entrance Hex&N&\wbox{00}{\underline{12}}&\wbox{00}{0}&\wbox{00/00/00}{10/7/3}\\
Runway Hex&N&\wbox{00}{\underline{10}}&\wbox{00}{36}&\wbox{00/00/00}{8/5/3}\\
City Hex&N&\wbox{00}{5}&\wbox{00}{48}&\wbox{00/00/00}{8/5/3}\\
Town Hex&N&\wbox{00}{3}&\wbox{00}{36}&\wbox{00/00/00}{2/1/0}\\
Village Hex&N&\wbox{00}{3}&\wbox{00}{36}&\wbox{00/00/00}{2/1/0}\\
\addlinespace
Major Bridge&N&\wbox{00}{\underline{18}}&\wbox{00}{48}&\wbox{00/00/00}{30/20/10}\\
Minor Bridge&N&\wbox{00}{12}&\wbox{00}{36}&\wbox{00/00/00}{18/12/6}\\
Small Bridge&N&\wbox{00}{6}&\wbox{00}{24}&\wbox{00/00/00}{6/4/2}\\
Pontoon Bridge&Y&\wbox{00}{4}&\wbox{00}{24}&\wbox{00/00/00}{6/4/2}\\
Dam&N&\wbox{00}{\underline{20}}&\wbox{00}{24}&\wbox{00/00/00}{15/10/5}\\
\addlinespace
\bottomrule
\end{tabular}

    \label{table:ground-targets}
\end{table*}

\clearpage
\twocolumn
\appendix

\section{Example Orders of Battle}

This appendix contains a number of example orders of battle and their correspondence in game units.

\begin{itemize}
    \item Table~\ref{table:ob-two-para}: 2 Para Battalion at the 1982 Battle of Goose Green. At this time, the battalion had attached half a battery of 105~mm Light Guns from Alma Battery of 29 Commando Regiment, RA, a section of Blowpipes from 43rd Air Defence Battery, RA, and a Recce Troop of 59 Independent Commando Squadron, RE.

    \item Table~\ref{table:ob-twelth-infantry}: 12th Infantry Regiment at the 1982 Battle of Goose Green, with attachments from the 25th Infantry Regiment, 4th Airborne Artillery Regiment, 601st Anti-Aircraft Artillery Group.
\end{itemize}

We see that we need the following additional ground units:
\begin{itemize}
    \item towed artillery \emph{section}
\end{itemize}

\begin{table*}[p]
    \centering
    \caption{2 Para at Goose Green}
    \label{table:ob-two-para}

    \begin{tabularx}{0.9\linewidth}{LLL}
        \toprule
        Element&Game Ground Unit&Notes\\
        \midrule

        Battalion HQ
        &\binarymultiply{1}{infantry HQ platoon}\\

        A Company
        &\binarymultiply{3}{infantry platoon}\\

        B Company
        &\binarymultiply{3}{infantry platoon}\\

        C Company
        &\binarymultiply{2}{infantry platoon} &Patrol Company\\

        D Company
        &\binarymultiply{3}{infantry platoon}\\

        Support Company
        &\binarymultiply{1}{infantry weapons platoon}&Mortars\\
        &\binarymultiply{1}{infantry weapons platoon}&GPMG\\
        &\binarymultiply{1}{infantry weapons platoon}&MILAN\\
        &\binarymultiply{1}{infantry platoon}&Engineers\\

        Alma Battery
        &\binarymultiply{1}{towed artillery section}&Three 105-mm Light Guns\\

        Blowpipe Section
        &\binarymultiply{2}{infantry SAM squad -- Blowpipe}\\

        Engineers Troop
        &\binarymultiply{1}{infantry platoon}\\

        \bottomrule
    \end{tabularx}
\end{table*}

\begin{table*}[p]
    \centering
    \caption{12IR at Goose Green}
    \label{table:ob-twelth-infantry}

    \begin{tabularx}{0.9\linewidth}{LLL}
        \toprule
        Element&Game Ground Unit&Notes\\
        \midrule
        Regimental HQ
        &\binarymultiply{1}{infantry HQ platoon}\\

        SAM Section
        &\binarymultiply{2}{infantry SAM squad -- SA-7B}\\

        A Company 12IR
        &\binarymultiply{5}{infantry platoon}\\

        B Company 12IR
        &\binarymultiply{3}{infantry platoon}\\

        C Company 25IR
        &\binarymultiply{2}{infantry platoon}\\
        &\binarymultiply{1}{infantry weapons platoon}\\

        C Company 12IR
        &\binarymultiply{2}{infantry platoon}\\
        &\binarymultiply{1}{infantry weapons platoon}\\

        A Battery, 4AAR
        &\binarymultiply{1}{towed artillery section}&Three 105-mm Pack Howitzers\\

        B Battery, 601 GADA
        &\binarymultiply{1}{Oerlikon 35~mm GDF section}&Two guns\\
        &\binarymultiply{1}{towed FCR-D}&Skyguard\\

        9th Engineering Company
        &\binarymultiply{1}{infantry platoon}\\

        \midrule

        3 Section B Battery 601AAAG
        &\binarymultiply{1}{Rh-202 20~mm battery}&Six guns\\
        &\binarymultiply{1}{EWR}&Elta\\

        Security Company
        &\binarymultiply{1}{infantry platoon}&\\

        \bottomrule
    \end{tabularx}
\end{table*}

\end{document}
