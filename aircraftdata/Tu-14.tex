\section{Tupolev Tu-14}

\includegraphics[width=\linewidth]{Tu-14.jpg}

The Tupolev Tu-14 was a conventional medium bomber and torpedo bomber. Its NATO reporting name is Bosun.

\subsection{Versions}

\subsubsection{Tu-14}

The Tu-14 was a conventional tactical bomber. It was developed in competition with the Ilyushin Il-28 and shares many features with that aircraft, including unswept wings, a swept tail, two Klimov VK-1 engines in pods under the wings, and a gun armament of two fixed 23~mm NR-23 guns and two more in a tail turret. It could carry 3,000~kg (6,600~lb) in its internal bomb bay.

However, the Tu-14 was not accepted for service by the VVS, which preferred the Il-28.

\subsubsection{Tu-14T}

The Tu-14T was a torpedo bomber, developed for Soviet AM VF. In contrast to the Tu-14 bomber version, the pilot and weapons officer are provided with ejection seats.

The Tu-14T served from 1952 to 1959 in Naval Aviation but did not serve in other branches and was not exported.

\subsection{Armament and Stores}

The Tu-14T could carry torpedoes, mines, or bombs in its internal bomb bay.

\subsection{Combat}

The Tu-14 did not see combat.

\subsection{ADCs}
\begin{adclist}
    \adcitem{Tu-14}
    \adcitem{Tu-14T}
\end{adclist}

\subsection{Photo Credit}
\begin{itemize}\raggedright
    \item \href{https://commons.wikimedia.org/wiki/File:Tu-14_01_00089082.jpg}{Tu-14}: SDASM Archives (Public Domain)
\end{itemize}
