\section{Ilyushin Il-10}

\includegraphics[width=\linewidth]{Il-10.jpg}

The Ilyushin Il-10 was a propeller-driven ground-attack aircraft developed during WWII as a replacement for the famous Il-2 Sturmovik.

\subsection{Versions}

Like its predecessor, the Il-10 featured extensive armor for the crew and engine and a dorsal defensive gun to protect from attacks from the rear.

The initial Il-10 had two 23~mm VYa-23 cannons and two 7.62~mm ShKAS machine guns in the wings, with 150 and 750 rounds each, respectively, plus a 12.7~mm UBT machine gun with 150 rounds protecting the tail. Later versions had four 23~mm NR-23 cannons in the wings and a 20~mm B-20 cannon in the dorsal mount.

The Il-10 was also manufactured under license by Avia in Czechoslovakia.

The Il-10 entered service with the Soviet VVS in 1944 and was retired in 1956. After the war, it served with the Bulgarian Air Force, the Chinese PLAAF (from 1950 to 1972), the Czechoslovak Air Force, the Hungarian Air Force, the Indonesian Air Force (former Polish aircraft), the North Korean KPAF, the Polish Air Force, the Romanian Air Force, and the Yemen Air Force.

\subsection{Armament and Stores}

In addition to its gun armament, a typical weapon load would be four 100~kg bombs, two in the bomb bays and two on the wing stations, or two 250~kg bombs on the wings. Alternatively, the wing stations could carry a total of eight RS-82 or four RS-132 rockets (although apparently these rockets were not used in the Korean War).

\subsection{Combat}

The Il-10 was used by the North Korean KPAF and saw extensive service in the first weeks of the Korean War, before suffering large losses against USAF fighters. It was also used in combat by the PLAAF in 1955 during the First Taiwan Strait Crisis.

\subsection{ADCs}
\begin{adclist}
    \adcitem{Il-10}
    \adcitem{Il-10 (Late)}
\end{adclist}

\subsection{Photo Credit}
\begin{itemize}\raggedright
    \item \href{https://commons.wikimedia.org/wiki/File:Ilyushin_Il-10_(China_Aviation_Museum).jpg}{Il-10}: Flavio Mucia (CC BY 2.0)
\end{itemize}
