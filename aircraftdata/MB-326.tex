\section{Aermacchi MB-326}

\includegraphics[width=\linewidth]{Impala II.jpg}

The Aermacchi MB-326 was designed as a trainer but then developed into a light attack aircraft. It featured tandem seating, a single Viper 11 turbojet engine, a low wing with wing-tip fuel tanks, and excellent performance for an aircraft of its class and era.

\subsection{Versions}

\subsubsection{MB-326}

The initial MB-326 trainer (with no letter suffix) entered service with the Italian AMI in 1962. It had no provision for armament.

\subsubsection{MB-326B/E/F/H}

The MB-326B/E/F/H were the first combined trainer and light attack versions. They were largely similar and all featured the 2,500~lbf Viper 11 engine. They were equipped with six under-wing weapon stations for up to 2000~lb of stores, including gun pods, bombs, and rocket pods. Versions after the B could also carry under-wing fuel tanks.

The B was delivered to the Tunisian air force in 1965, the E to the Italian AMI for weapons training in 1968, the F to the Ghanaian air force in 1965, and the H to the Australian RAAF and RAN starting in 1967.

The H was also manufactured in Australia by the Commonwealth Aircraft Corporation (CAC) as the CA-30. It served in the Australian RAAF and RAN from 1967 to 2001 and was known informally as the “Macchi”. It replaced the Vampire T.33A, T.34A, and T.35 trainers.

\subsubsection{MB-326G}

The later MB-326G light attack version featured a more powerful 3,410~lbf Viper 20-540 engine and a strengthened wing, which gave better performance in general and allowed up to 4000~lb of stores to be carried.

The GB version was delivered to the Argentine COAN starting in 1969 and served until sometime after the 1982 South Atlantic War, when it became impossible to maintain because of the UK embargo of spares for the Viper engine.

The GB also was delivered to the Zairean air force in 1969 and the Zambian air force in 1971.

The GC version was delivered to the Brazilian air force starting in 1971. It was also manufactured under license by Embraer as the EMB-326. In Brazilian FAB service, the GC was known as the AT-26 or RT-26 Xavante and served from 1971 to 2010. A number of Brazilian GCs were transferred to the Argentine Navy after 1982 to replace losses in the South Atlantic War.

\subsubsection{MB-326K and Impala Mk II}

The MB-326K was a dedicated light attack aircraft and dispensed with the second crew member. It had two internal 30 mm DEFA cannon mounted in the fuselage, an armored, single-seat cockpit, an uprated 4,000~lbf Viper 632-43 engine, six stations for up to 4000~lb of stores, including gun pods, bombs, rocket pods, fuel tanks, a four-camera photo-reconnaissance pod, and R.550 Magic infrared-homing missiles.

The K was used by the South African SAAF starting sometime after 1971. It was also manufactured under license in South Africa from 1974 by the Atlas Aircraft Corporation, albeit with the earlier 3,140~lbf Viper 20-540 motor. Both variants of the K served in the SAAF as the Impala Mk.II. They later served in the Brazilian FAB from 2004 to 2009 as the AT-26A Xavante.

\subsubsection{MB-326L}

The MB-326L was a two-seater version of the K without the integrated guns.

It was used by the air forces of Ghana, Tunisia, the UAE, and Zaire from 1975. It was further developed into the MB-336.

\subsubsection{MB-326M and Impala Mk I}

Despite M coming after K and L in the alphabet, the M was delivered before both.
It was largely similar to the G.

It was manufactured both by Aermacchi and under license in South Africa by the Atlas Aircraft Corporation.

Both versions served with the South African SAAF from 1966 as the Impala Mk I.

\subsection{Armament and Stores}

The E/F/H could use 300L under-wing FTs, whereas the G/K/L/M could use slightly larger 330L FTs.

All versions from the B onwards could carry a variety of bombs, rocket pods, and gun pods.

In Argentine COAN service, the MB-326GB would typically mount two single .50~cal gun pods plus four LAU-10 RPs (each with four Zuni rockets), four LAU-32 RPs (each with seven 70~mm rockets), four Matra-122 RPs (each with seven 68~mm), four Mk~81 250~lb bombs, or two Mk~82 500~lb bombs.

In RAAF service, the MB-326H could mount two SUU-11A/A 7.62~mm gun pods with a rate of fire of 2000~rpm per pod and 200 rounds per pod (3 ammo). It could also carry eight 25~kg practice bombs or, presumably in an emergency situation, 250~lb or 500~lb bombs. To increase its range or endurance, it could use 300L fuel tanks.

In Brazilian FAB service, the air-to-ground options for the AT-26 Xavante (G and K) included 250~lb and 500~lb bombs, 7.62~mm gun pods, and rocket pods with seven, nineteen, or thirty-seven 70~mm rockets.

In the SAAF service, the Impala Mk I and II (M and K) were typically armed with up to six Matra F2 rocket pods (each with six 68 mm rockets) or with 120 or 250 kg (250 and 500~lb) bombs and sometimes also equipped with a photographic reconnaissance pod or 330L fuel tanks. Although the outer pylons were apparently wired for IRMs, there is no evidence that they were carried.

\subsection{Combat}

In the SAAF service, the Impala Mk I and II saw combat in the South African Border War. They flew both daylight and nighttime {\itshape Maanskyn} (moonshine) missions.

Argentine CANA MB-326Gs were based on the mainland during the South Atlantic War and did not see combat.

\subsection{ADCs}

\begin{adclist}
    \adcitem{MB-326B}
    \adcitem{MB-326E}
    \adcitem{MB-326F}
    \adcitem{MB-326G}
    \adcitem{MB-326H}
    \adcitem{MB-326K}
    \adcitem{MB-326K (Viper 20-540)}
    \adcitem{MB-326L}
    \adcitem{MB-326M}
\end{adclist}

\subsection{See Also}
\begin{typelist}
    \typeitem{Aermacchi MB-339}
\end{typelist}

\subsection{Photo Credit}
\begin{itemize}\raggedright
    \item \href{https://commons.wikimedia.org/wiki/File:SAAF_Impala_MkII_1075_(6902824683).jpg}{Atlas Impala II}: Bob Adams (CC BY-SA 2.0)
\end{itemize}
