%!TEX root = ./helicopterdata.tex
%LTeX: enabled=true

\subsection{Hughes AH-64 Apache}

The Hughes AH-64 Apache is an attack helicopter designed specifically for the anti-tank role. Hughes Helicopters was later taken over by McDonnell Douglas which in turn was taken over by Boeing, so the Apache is now produced and developed by Boeing.

The initial AH-64A version is armed with a 30 mm M230 chain gun in a swivel mount under the nose and 1200 rounds. Its stub wings have four weapon stations, each of which can carry four AGM-114 Hellfire RGs, a M261 \binarymultiply{19}{70 mm} RPs, or a dual rack for AIM-92 Stinger IRMs. It entered service in the US Army in 1984 and was declared operational in 1986. It served in the First Gulf War and other US operations in the 1990s, and was finally retired by the US Army in 2012.

The AH-64D/E versions have the Longbow mast-mounted radar that allows targets to be identified and engaged from behind defilade. The radar can be removed in low-threat environments. The AH-64D entered service with the US Army in 1997 and saw combat in the Second Gulf War, the Occupation of Iraq, and Afghanistan.
