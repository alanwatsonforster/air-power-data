\section{Dassault Ouragan}

\includegraphics[width=\linewidth]{Ouragan.jpg}

The Dassault Ouragan was a fighter-bomber and the first French-designed jet-powered combat aircraft enter service. Like many of its contemporaries, it had a single engine, an air-intake in the nose, swept wings and tail, and wing-tip fuel tanks. Its name means “hurricane” in French.

\subsection{Versions}

\subsubsection{MD 450A and MD 450B}

The initial 450A version used a non-afterburning Rolls-Royce Nene 102 engine. The subsequent 450A version used a similar Rolls-Royce Nene 104 engine license-built by Hispano-Suiza and had a revised undercarriage. However, the MD 450B was otherwise almost identical to the MD 450A.

The Ouragan entered service with the Armée de l'Air in 1952, replacing Vampires and other older aircraft. Its service life was short, and in 1955 it was replaced by the Mystère IV.

The Ouragan also served with the Indian Air Force from 1953 and was known as the Toofani (“hurricane” in Hindi). It was withdrawn from front-line services in 1965.

It also served with the Israeli Air Force from 1953 and served until the 1970s. In 1975, several were sold on to El Salvador, where they served until the late 1980s.

\subsection{Armament and Stores}

The principal armament was four 20~mm Hispano-Suiza cannons under the nose. Two 450L fuel tanks would often be carried for extended range. For air-to-ground missions, the Ouragan could also carry bombs, napalm tanks, 105~mm Brand T-10 rockets, and SNEB 68~mm rocket pods.

\subsection{Combat}

The Ouragan saw combat with the Israeli Air Force in the Suez Crisis and the 1967 War. In Indian service, it saw combat in the 1962 China-India War and the 1965 India-Pakistan War.

\subsection{ADC}
\begin{adclist}
    \adcitem{Ouragan}
\end{adclist}

\subsection{Photo Credit}
\begin{itemize}\raggedright
    \item \href{https://commons.wikimedia.org/wiki/File:RMM_Brussel_Dassault_MD450_Ouragan_3.JPG}{Ouragan}: Ad Meskens
\end{itemize}