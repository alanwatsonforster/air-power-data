\section{Dassault Mystère IV}

\includegraphics[width=\linewidth]{Mystère IVB.jpg}

The Mystère IV was a fighter-bomber. It resembled the earlier Mystère aircraft, but was a new design adapted for supersonic flight. It featured an air-intake in the nose, a single engine, and sharply swept wings and tail.

\subsection{Versions}

\subsection{MD 454A}

The MD 454A or Mystère IVA model was the only version to reach production. It featured a non-afterburning Rolls-Royce Tay engine (supplied by Rolls-Royce themselves or built under license by Hispano-Suiza) and two 30~mm DEFA cannons under the nose. It could reach supersonic speeds only in a dive.

The Mystère IVA entered service with the Armée de l'aire in 1955 and served first as an interceptor and then as a fighter-bomber. It also served with the Israeli Air Force from 1956 to 1971 and with Indian Air Force from 1957 to 1973.

\subsection{Armament and Stores}

The principal armament of the Mystère IVA was two 30~mm DEFA cannons under the nose. Two 450L fuel tanks would often be carried for extended range. For air-to-ground missions, the Mystère IVA could also carry bombs, napalm tanks, 105~mm Brand T-10 rockets, and SNEB 68~mm rocket pods.

\subsection{Combat}

Israeli Mystère IVAs saw combat in the 1956 and 1967 Arab-Israeli Wars. French aircraft also fought in the 1956 Suez Crisis. Indian aircraft fought in the 1965 and 1971 India-Pakistan Wars.

\subsection{ADC}
\begin{adclist}
    \adcitem{Mystère IVA}
\end{adclist}

\subsection{Photo Credit}
\begin{itemize}\raggedright
    \item \href{https://commons.wikimedia.org/wiki/File:Dassault_Mystère_IV_landing_at_Sion_1986_01.jpg}{Mystère IV}: Anidaat (CC BY-SA 4.0)
\end{itemize}