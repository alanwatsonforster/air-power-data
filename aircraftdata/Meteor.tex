\section{Gloster Meteor}

\includegraphics[width=\linewidth]{Meteor.jpg}

The Gloster Meteor was the first British jet fighter and the only Allied jet to see combat in WW2. It entered service as a day fighter with the RAF in 1944 and continued to serve in various roles after the war. It was powered by two wing-mounted centrifugal-flow jet engines, had unswept wings, and was armed with four 20 mm guns.

\subsection{Versions}

\subsubsection{Meteor F.8}

The F.8 was a post-WW2 development of the late-WW2 F.4. It had more powerful Rolls-Royce Derwent 8 engines, a lengthened fuselage, a new tail to solve center-of-gravity problems, improved visibility from the cockpit, and provided the capability to carry air-to-ground weapons.

In common with many early jet fighters, it had short endurance and range compared to contemporary propeller-engined aircraft, and this was partially addressed by equipping it with a jettisonable, conformal ventral fuel tank.

The unsophisticated aerodynamics of the Meteor led it to be outclassed by newer swept-wing fighters like the MiG-15 and F-86, and so the F.8 was the last fighter version produced. Subsequent versions were reconnaissance and night fighters.

It was introduced into RAF service in 1949 and served until it was replaced by the Canadair Sabre 4 and Hawker Hunter in the 1950s. It was also used by No.~77 Squadron RAAF in the Korean War, replacing their F-51Ds in April 1951 and serving until replaced by the CAC Sabre. Also, it served in the air forces of Belgium, Brazil, Denmark, Ecuador, and the Netherlands.

\subsubsection{Meteor FR.9}

The FR.9 was a photoreconnaissance version of the F.8. It had an extended nose for a single camera that could be configured on the ground to be either overhead or oblique and retained the full combat capability of the F.8.

It served in the RAF from 1950, the Ecuadorian Air Force, the Israeli IAF, and the Syrian Air Force.

% Sabre 4 in RAF?
% Other users?
% Add PR.10?

\subsection{Armament and Stores}

The internal armament of the Meteor was four 20 mm Hispano V cannons.

A typical air-to-air load was the ventral 175-gallon fuel tank, perhaps with two 100-gallon fuel tanks under the wings to increase patrol time.

A typical air-to-ground load was two 1000 lb bombs or eight or sixteen RP-3 rockets in addition to the ventral tank.

\subsection{Combat}

The F.8 saw combat with the RAAF in the Korean War, mainly in the ground-attack role. In the Suez Crisis, the F.8 was used by the Egyptian, Syrian, and Israeli air forces, and the FR.9 by the RAF.

\subsection{ADCs}

ADCs are provided for:
\begin{adclist}
    \adcitem{Meteor F.8}
    \adcitem{Meteor FR.9}
\end{adclist}

\subsection{Photo Credit}
\begin{itemize}\raggedright
    \item \href{https://commons.wikimedia.org/wiki/File:Gloster_Meteor_Centenary_of_Military_Aviation_2014_(cropped).jpg}{Gloster Meteor}: Chris Phutully (CC BY 2.0)
\end{itemize}
