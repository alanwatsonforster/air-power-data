\section{de Havilland Vampire}

\includegraphics[width=\linewidth]{Vampire.jpg}

The de Havilland Vampire was a first-generation British jet day fighter, fighter-bomber, night fighter, and trainer. It first flew in 1943 but did not enter service until after the end of WW2. It had a single engine, straight wings, and an innovative twin-boom tail. The wings, rear fuselage, and tail were aluminum for strength, but the forward fuselage was fabricated from molded plywood, spruce, and balsa, building on de Havilland’s extensive experience with this technique, most notably in the Mosquito. Initially, the pilot was not provided with an ejector seat, although one was refitted to some export versions.

Its armament was four Hispano 20 mm cannons, the standard for British fighters at that time, mounted under the nose.

The Vampire entered service with the RAF in 1946. Its contemporary, the twin-engined Meteor, was more complex, expensive, and had less endurance on internal fuel. However, the Meteor crucially had a higher rate of climb, and this led to it being selected for the interceptor role, and the Vampire served mainly as a day fighter, fighter-bomber, and trainer.

Nevertheless, the Vampire had considerable success on the export market, perhaps because of its lower price and simplicity compared to the Meteor, and for many air forces, it was their first jet aircraft.

\subsection{Versions}

\subsubsection{Vampire F.1}

The initial F.1 version was a day fighter. The initial batch of F.1 aircraft had the less powerful Goblin 1 engine and lacked cockpit pressurization, but later F.1 aircraft had the more powerful Goblin 2 engine and pressurization. It could carry 100 imperial gallon (450L) fuel tanks under the outer wings, but had no provision for air-to-ground weapons other than their guns.

It served in the RAF, French AA, Swedish Flygvapnet, and the Dominican AMD/FARD.

\subsubsection{Vampire F.3}

The F.3 was similar to the F.1, but had significantly more internal fuel and a much improved pressurization system. Like the F.1, it could carry 100 imperial gallon (450L) fuel tanks, but had no provision for air-to-ground stores.

It served in the RAF, RCAF, and Mexican FAM.

\subsubsection{Vampire FB.5}

The FB.5 was the first fighter-bomber version. It was developed from the F.3 and featured clipped wings for better performance at low level, armor to protect the engine, provision for bombs in place of the fuel tanks, and rails under the inner wings for up to eight RP-3 or T-10 rockets.

It served in the RAF, the French AA, the Italian AMI, the Lebanese LAF, the RNZAF, and the SAAF. The aircraft for the French AA were license-built by SNCASE.

\subsubsection{Vampire FB.6}

The FB.6 was derived from the FB.5 but had the more powerful Goblin 3 engine. The “Late” version reflects the 1960 upgrade to install an ejection seat.

It was manufacturer by both de Havilland and Eidgenössische Konstruktionswerkstätte (F+W).

It served in the Swiss Flugwaffe from 1950 to 1990 (alongside de Havilland Vampire FB.6s), although from 1968 only as a trainer.

\subsubsection{Vampire FB.9}

The FB.9 was also derived from the FB.5 and had air conditioning for use in tropical zones. Most FB.9s retained the Goblin 2 engine of the FB.5, but Rhodesian FB.9s were fitted with the Goblin 3 for better performance.

It served in the RAF, RAAF, Jordanian RJAF, Lebanese LAF, RNZAF, SRAF/RRAF, and SAAF.

\subsubsection{Vampire NF.10}

The Vampire NF.10 was a night fighter version, derived from the FB.5 but with a new forward fuselage for the AI Mk X (SCR-720B) radar, side-by-side seating for the pilot and radar operator, and the Goblin 3 engine to compensate for the weight of the radar equipment. It was developed initially for export, but was taken up by the RAF to bridge the gap between the Mosquito NF.36 and the Meteor NF.11.

It served in the RAF.

\subsubsection{Vampire T.11}

The Vampire T.11 was a trainer version, derived from the FB.5 but with a new forward fuselage similar to that of the NF.11, with side-by-side seating for the instructor and pupil and dual controls. It retained full combat capability. RAF aircraft were refitted with ejection seats between 1954 and 1957.

It served in the RAF, Austrian Luftstreitkräfte, Chilean FACh, Indian IAF, Jordanian RJAF, Mexican FAM, and Swiss Air Force.

\subsubsection{Vampire F.20}

The Sea Vampire F.20 was an adaptation of the FB.5 for RN carrier operations. It had strengthened undercarriage, an arrestor hook, and more effective speed brakes, but no capacity to fold its wings.

It served in the RN FAA, but for trials and familiarization and not as a front-line fighter.

\subsubsection{Vampire T.22}

The Sea Vampire T.22 trainer was essentially a T.11 adapted to RN standards, but was not carrier-capable. It was flown exclusively from terrestrial air stations. RN aircraft were refitted with ejection seats in 1956 and 1957.

It served in the FAA as an advanced trainer.
\subsubsection{Vampire F.30}

The Vampire F.30 was derived from the Vampire F.2, which had trialed the Nene engine in the F.1 airframe. It was equipped with a Nene-2VH engine with 5,000 lb of thrust, significantly more than the Goblin 3 with 3,500 lb. This greater thrust required greater airflow than could be provided by the standard wing-root intakes, so the F.30 was delivered with additional intakes on the upper side of the fuselage behind the cockpit. To improve handling close to the critical Mach number, these were later moved to the lower side of the fuselage. The F.30 was license-built by de Havilland (Australia).

The F.30 served in the RAAF from 1949 until 1960, when it was replaced by the CAC Sabre.

\subsubsection{Vampire FB.31}

The Vampire FB.31 was derived from the F.30 with the air-to-ground improvements of the FB.5. It was license-built by de Havilland (Australia).

The FB.31 served in the RAAF from 1952 until 1960, when it was replaced by the CAC Sabre.

\subsubsection{Vampire T.33 and T.33A}

The Vampire T.33 and was a trainer largely similar to the Vampire T.11 (and notably using the Goblin 35 engine rather than the Nene). The ejection seats and canopy of the T.35 were refitted to the T.33 to give the T.33A. It was license-built by de Havilland (Australia).

The T.33 served in the RAAF from 1952 until 1970, when it was replaced by the Aermacchi MB-326H (CAC CA-30). The T.33A conversions took place some time after 1957.

\subsubsection{Vampire T.34 and T.34A}

The Vampire T.34 and T.34A were similar to the T.33 and T.33A, but for the RAN rather than the RAAF. It was license-built by de Havilland (Australia).

The T.34 served in the RAN from 1954 until 1970, when it was replaced by the Aermacchi MB-326H (CAC CA-30). The T.34A conversions took place some time after 1957.

\subsubsection{Vampire T.35}

The Vampire T.35 was a development of the T.33 and added ejection seats, a new canopy, and increased fuel capacity. The ejection seats and canopy were refitted to the T.33 and T.34 to give the T.33A and T.34A versions. It was license-built by de Havilland (Australia).

The T.35 served in the RAAF from 1957 until 1970, when it was replaced by the Aermacchi MB-326H (CAC CA-30).

\subsubsection{Vampire FB.50 and FB.52}

The Vampire FB.50 and FB.52 were licensed or export versions of the FB.6.

The FB.50 and FB.52 were built by de Havilland and HAL license-built the FB.52 for the Indian IAF.

The FB.50 served in the Swedish Flygvapnet and the Dominican AMD/FARD. The FB.52 served in the Egyptian EAF, Finnish Ilmavoimat, Indian IAF (from 1952 to at least 1971), Iraqi Air Force, Jordanian RJAF, Lebanese LAF, RNZAF, Norwegian Luftförsvaret, SRAF/RRAF, Saudi Arabian Air Force, SAAF, Syrian Air Force, and Venezuelan FAV.

\subsubsection{Vampire FB.52A}

The Vampire FB.52A was a version of the FB.52 for the Italian AMI. Unusually, the FB.52A had the Goblin 2 engine rather than the more powerful Goblin 3 engine used by many of the other fighter-bomber Vampires.

It was built by de Havilland and also under-license by FIAT and Macchi.

It served in the Italian AMI, the Egyptian EAF, and the Syrian Air Force.

\subsubsection{Vampire NF.54}

The Vampire NF.54 was the export version of the NF.10.

It served in the Italian AMI and the Indian IAF.

\subsubsection{Vampire T.55}

The Vampire T.55 was the export version of the T.11, and again later variants were fitted with ejection seats.

It was built by de Havilland and also license-built by HAL for the Indian IAF and F+W for the Swiss Flugwaffe.

A few IAF aircraft were adapted in 1959 for photo-reconnaissance and designated PR.55.

It served in the Austrian Luftstreitkräfte, Burmese Air Force, Chilean FACh, Egyptian EAF, Finnish Ilmavoimat, Indian IAF and INAS (from 1952 to 1989), Indonesian Air Force, Iraqi Air Force, Irish IAC, RNZAF, Norwegian Luftforsvaret, Portuguese Air Force, SAAF, Swedish Flygvapnet, Swiss Flugwaffe (late version from 1955 to 1990), and Venezuelan FAV.

\subsubsection{Vampire T.55A}

The Vampire T.55A was a conversion of the FB.50 with a forward fuselage like that of the T.55.

It served in the Swedish Flygvapnet.

\subsection{Armament and Stores}

The gun armament of all versions was four Hispano 20 mm cannons with 150 rounds per gun.

A typical air-to-air load was two 100 gal (450L) fuel tanks to increase endurance.

A typical air-to-ground load was eight RP-3 or T-10 rockets and then either two 500 lb bombs or fuel tanks, depending on the mission radius. On short-range missions, two 1,000-lb bombs could be carried but without rockets.

\subsection{ADCs}

\begin{adclist}
    \adcitem{Vampire F.3}
    \adcitem{Vampire FB.5}
    \adcitem{Vampire FB.6}
    \adcitem{Vampire FB.6 (Late)}
    \adcitem{Vampire FB.9}
    \adcitem{Vampire FB.9 (Goblin 3)}
    \adcitem{Vampire NF.10}
    \adcitem{Vampire T.11}
    \adcitem{Vampire T.11 (Late)}
    \adcitem{Sea Vampire F.20}
    \adcitem{Sea Vampire T.22}
    \adcitem{Sea Vampire T.22 (Late)}
    \adcitem{Vampire F.30}
    \adcitem{Vampire FB.31}
    \adcitem{Vampire T.33}
    \adcitem{Vampire T.33A}
    \adcitem{Vampire T.34}
    \adcitem{Vampire T.34A}
    \adcitem{Vampire T.35}
    \adcitem{Vampire FB.50}
    \adcitem{Vampire FB.52}
    \adcitem{Vampire FB.52A}
    \adcitem{Vampire NF.54}
    \adcitem{Vampire T.55}
    \adcitem{Vampire T.55 (Late)}
    \adcitem{Vampire T.55A}
\end{adclist}

\subsection{See Also}

\begin{typelist}
    \typeitem{SNCASE Mistral}
\end{typelist}

\subsection{Photo Credit}
\begin{itemize}\raggedright
    \item \href{https://commons.wikimedia.org/wiki/File:De_Havilland_DH115_Vampire_banking_with_the_sun_reflecting_off_its_silver_wings_(cropped).jpg}{de Havilland Vampire}: Pseudopanax (Public domain)
\end{itemize}
