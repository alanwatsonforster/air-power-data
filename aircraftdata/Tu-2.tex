\section{Tupolev Tu-2}

\includegraphics[width=\linewidth]{Tu-2.jpg}

The Tupolev Tu-2 was a propeller-driven medium bomber that served in WW2 and after. Its NATO reporting name is Bat.

\subsection{Versions}

\subsubsection{Tu-2}

The Tu-2 was the initial version and entered service in small numbers with the Soviet VVS in 1942. Although well regarded, production was suspended in favor of fighters and the Pe-2 light bomber.

\subsubsection{Tu-2S}

In 1943, production restarted of the improved Tu-2S version. This version had improved and more powerful ASh-82FNs engines, a four-bladed propeller in place of the earlier three-bladed one, changes to the structure and systems, including elimination of the dive brakes to simplify production and improve robustness, removal of the fixed forward machine guns, and improved defensive armament in the form of three single 12.7~mm UBT machine guns.

The Tu-2S served in the VVS from early 1944 until 1955. It also served with the Bulgarian Air Force, Chinese PLAAF, Hungarian Air Force, North Korean KPAF, Polish Air Force and Navy, and Romanian Air Force. Indonesia later received a few ex-Chinese Tu-2S aircraft.

\subsubsection{Tu-2P}

The PLAAF developed the Tu-2P to intercept overflights by ROCAF aircraft. The RP-1 radar from the J-5A (MiG-17PF) was installed in the nose, and two 23 mm NR-23 cannons replaced the 20 mm cannons in the wings. Defensive armament was deleted. The radar was modified to eliminate the lower part of the scan, which reduced ground clutter at the cost of only being able to detect and track targets at the same or higher altitude.

Several PLAAF aircraft were converted in 1959.

\subsection{Armament and Stores}

The Tu-2S was armed with two fixed 20mm ShVaK cannons in the wing roots and three 12.7mm UBT defensive guns in single mounts operated by the navigator, the radio operator, and the ventral gunner. In the Tu-2P, the fixed guns were replaced by a pair of 23mm NR-23 cannons, and the defensive guns presumably removed.

A typical bomb load for the Tu-2S would be three 1,000~kg bombs (one carried internally and one under each wing), four 250~kg bombs (two carried internally and one under each wing), or nine 100~kg bombs (all carried internally).

\subsection{Combat}

The Tu-2S saw combat in WW2, with Chinese communist forces in the Chinese Civil War, with the PLAAF and KPAF in the Korean War, and with the PLAAF in the 1959 Tibetan Uprising and associated conflicts before and after.

The Tu-2P saw combat in November 1960 when three PLAAF aircraft attempted to intercept a ROCAF P2V. Two of the three flew into terrain, and the P2V escaped.

\subsection{ADCs}
\begin{adclist}
    \adcitem{Tu-2S}
    \adcitem{Tu-2P}
\end{adclist}

\subsection{Photo Credit}
\begin{itemize}\raggedright
    \item \href{https://commons.wikimedia.org/wiki/File:Tu-2_45937881.jpg}{Tu-2}: SDASM Archives (Public Domain)
\end{itemize}
