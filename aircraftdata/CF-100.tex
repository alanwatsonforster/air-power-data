\section{Avro Canada CF-100 Canuck}

\includegraphics[width=\linewidth]{CF-100.jpg}

The Avro Canada CF-100 was a twin-engined, straight-wing, all-weather interceptor designed specifically for the RCAF to intercept Soviet bombers at long range over Canada. It was similar in many respects to the F-89 Scorpion and F-94 Starfire, but was larger than both.

\subsection{Versions}

\subsubsection{Mk 3}

The Mk 3 was the first production version. It was equipped with the APG-33 radar and Hughes E-1 fire-control system (which were also used on the F-94A/B) and armed with eight .50~cal M3 machine guns in a ventral pack.

It served in the RCAF from 1953 but was relegated to training duties shortly after the introduction of the Mk 4A.

\subsubsection{Mk 4A}

The Mk 4A was a development of the Mk 3, and had more powerful Orenda 9 engines. The radar and fire-control system were upgraded to the APG-40 and Hughes MG-2 (which were also used on the F-89D). The main armament was wing-tip pods each with 29 FFAR rockets, although it retained the ventral gun pack.

It served in the RCAF in Canada from 1953 and in Europe from 1956, and was retired as a fighter in 1962.

\subsubsection{Mk 4B}

The Mk 4B was similar to the Mk 4A, but used more powerful Orenda 11 engines.

It served in the RCAF in Canada from sometime after 1953 and in Europe from 1956, and was retired as a fighter in 1962.

\subsubsection{Mk 5}

The Mk 5 was a further development of the Mk 4B, with extended wings and horizontal stabilizers for better performance at high altitude. The gun pack, considered ineffective for attacking bombers, was omitted to save weight.

It served in the RCAF from 1955 to 1962, and began to be replaced by the CF-101 from 1961. The Belgian Air Force also flew Mk 5s from 1957 to 1964.

\subsection{Armament and Stores}

The internal armament depended on the version and was a mixture of .50~cal machine guns and FFAR air-to-air rockets.

The wing-tip rocket pods could be swapped for 1200L fuel tanks for ferry flights.

\subsection{Combat}

The CF-100 was not used in combat.

\subsection{ADCs}
\begin{adclist}
    \adcitem{CF-100 Mk 4B}
    \adcitem{CF-100 Mk 5}
\end{adclist}

\subsection{Photo Credit}
\begin{itemize}\raggedright
    \item \href{https://commons.wikimedia.org/wiki/File:CF-100s_423_Sqn.jpg}{Avro Canada CF-100 Canuck}: Canadian Department of National Defence (Public Domain)
\end{itemize}
