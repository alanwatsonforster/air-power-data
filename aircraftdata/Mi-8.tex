%!TEX root = ./helicopterdata.tex
%LTeX: enabled=true

\subsection{Mil Mi-8}

The Mil Mi-8 (Hip) is a Soviet/Russian transport and attack helicopters.

The Mi-8T (Hip-C) is a transport version capable of carrying 24 passengers or 4,000 kg of cargo. It is crewed by a pilot, copilot, and flight engineer. Two external weapon racks allow it to be armed with four UB-16-57 \binarymultiply{16}{55 mm} RPs. A single 7.62 mm PK machine gun can be mounted in place of the central lower window and operated by the flight engineer, two more can be mounted on the weapon racks, and up to three more can be mounted on the side and rear doors. 

The Mi-8TV (Hip-E) version is a dedicated attack helicopter; when carrying a complete load of weapons it cannot carry a useful load of passengers or cargo. The forward firing PK machine gun is replaced by a 12.7 mm VK-4 heavy machine gun, the stores racks are extended to allow carriage of six UB-16-57 \binarymultiply{16}{55 mm} RPs, and four 9M17P Falanga-P (AT-2C Swatter-C) missiles can be carried above the racks. 

The Mi-8TVK (Hip-F) is the export version of the Mi-8TV and substitutes six 9M914M Malyutka (AT-3 Sagger) missiles for the Falangas.

As a result of experience in the Soviet-Afghan War, many Mi-8 helicopters were fitted with improved countermeasures, including an exhaust cooling system, an infrared jammer, flares, and a radar-warning system.

The Mi-8T entered service in the Soviet Union in 1967 and the Mi-8TV in 1974. They saw combat in the Soviet-Afghan War. After the breakup of the Soviet Union, they saw combat in the many wars between Russia and other successor states. The Mi-8 has been widely exported, both to allies of the Soviet Union or Russia and to many neutral countries.
