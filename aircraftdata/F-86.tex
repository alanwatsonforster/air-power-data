%!LW recipe=latexmk (xelatex)
%!TEX root = ../aircraftdatacards/sandbox.tex

\section{North American F-86 Sabre}

The North American F-86 Sabre is a single-engined fighter, interceptor, fighter-bomber, and tactical nuclear bomber, and was the first swept-wing aircraft to enter service in the USAF. There are many versions and variants of the F-86, and we discuss here first the fighters and fighter-bombers (F-86A/E/F/H) and then the interceptors (F-86D/K/L).%
\note{
    \noteparagraph{Sources}
    \begin{itemize}
        \item \href{https://www.af.mil/News/Photos.aspx?igphoto=2000593829}{Air Force photo of F-86F-35 and caption} (\href{https://web.archive.org/web/20260119033326/https://www.af.mil/News/Photos.aspx?igphoto=2000593829}{Wayback})
        \item \href{http://www.joebaugher.com/usaf_fighters/p86.html}{Baughter}
        \item Curtis, “North American F-86 Sabre”, 2000, Crowood.
        \item F-86A Flight Manual
        \item F-86L SAC
        \item Goebel, ”\href{http://airvectors.net/avf86.html}{North American F-86 Sabre}”, Air Vectors.
        \item Webster, “The F-86 Sabre Family Tree”, APJ 25. This has ADCs for A/E/D/F/F-25/F-40/H/K/L.
        \item \href{https://en.wikipedia.org/wiki/North_American_F-86_Sabre}{Wikipedia on F-86}
        \item \href{https://en.wikipedia.org/wiki/North_American_F-86D_Sabre}{Wikipedia on F-86D}
    \end{itemize}

    \noteparagraph{Summary of Versions} This table gives a summary of the major differences between versions. Note that many aircraft were later upgraded with the 6-3 or extended 6-3 wing.

    \begin{center}
        \scriptsize
        \begin{tabularx}{\linewidth}{llclL}
            \toprule
            Version&Engine&Thrust&Wing&Notes\\
            &&(lb)\\
            \midrule
            A&J47-GE-7 &5340&Original\\
            E&J47-GE-13 &5450&Original& All-flying tail. Radar ranging.\\
            F-1&J47-GE-27 &5970&Original\\
            F-5/10/15/20&J47-GE-27 &5970&Original&200-gallon FTs for combat\\
            F-5/10/15/20&J47-GE-27 &5970&Original&\\
            F-25/30&J47-GE-27 &5970&Original/6-3&Dual-store wing\\
            F-35&J47-GE-27 &5970&?&Dual-store wing with Mk 12 nuclear bomb\\
            F-40&J47-GE-27 &5970&Extended 6-3&Dual-store wing\\
            H&J73-GE &8900&6-3&Dual-store wing with FTs only on outer stations\\
            \midrule
            D&J47-GE-17 &5500/7650&Original\\
            K&&&Original\\
            L&J47-GE-33 &5500/7650&Extended 6-3\\
            \bottomrule
        \end{tabularx}
    \end{center}
}

\subsection{Fighter Versions}

\subsubsection{F-86A}

The F-86A has a swept wing and a swept tail with conventional control surfaces. It is powered by a J47-GE-7 engine with 5340 lb of dry thrust; only the interceptor versions of the F-86 had afterburners. It is armed with six .50 cal M3 machine guns and the view from the cockpit is excellent.%
\note{
    \noteparagraph{F-86A} ADC adapted from Webster. Curtis (pp.~22-23) states that radar ranging was also installed on the last 25 As, but I've not added this variant. Service from Curtis (p.~19).
}

Like many of its contemporaries, it suffers from a very short range on internal fuel, and relied on a pair of either 120-gallon or 206-gallon fuel tanks on under-wing stations to achieve a useful range. However, the 206-gallon tank limits maneuvers to 3G and so can only be used for ferry flights. Instead of fuel tanks, it can carry two bombs of up to 1000 lb or sixteen 5-inch HVAR rockets, but has a very restricted combat range while doing so, and therefore served essentially as a day fighter rather than a fighter bomber.

It entered service with the USAF in March 1949. It was rushed to Korea to counter Soviet MiG-15s, which were outclassing the straight-wing F-80C fighters already in theater, and saw combat starting in December 1950.

\subsubsection{F-86E}

The F-86E improves on the F-86A by having a slightly more powerful J47-GE-13 engine with 5450 lb of dry thrust, the A-1CM radar-ranging gunsight, and an all-flying tail. The new tail significantly improves the handling compared to the F-86A at high transonic speeds.%
\note{
    \noteparagraph{F-86E} ADC adapted from Webster. J47-GE-13 engine with 5450 lb thrust (Curtis, p. 62). All-flying tail (Curtis, p. 61). Service from Curtis (p.~63).

    \noteparagraph{Radar Ranging} Webster adds radar-ranging in the F, but Baughter and Curtis (p.~62) state the APG-30 was standard in the E. I have added radar-ranging to the ADC for the E.

}

The F-86E entered service in February 1951 and arrived in Korea in July 1951.

\subsubsection{F-86F}

The F-86F series introduces a series of significant improvements over earlier models. All F-86F variants are powered by the J47-GE-27 engine, producing 5970 lb of dry thrust.%
\note{
    \noteparagraph{F-86F} ADC adapted from Webster. J47-GE-27 with 5970 lb thrust (Curtis, p. 69).
}

The initial F-1 block otherwise closely resembles the F-86E. It entered service in April 1952 and arrived in Korea by June 1952.%
\note{
    \noteparagraph{F-86F-1} ADC adapted from Webster. Curtis (pp.~69--73).
}

The F-5 block adds the improved A-4 gunsight.%
\note{
    \noteparagraph{F-86F-5} ADC adapted from Webster. Goebel and Curtis (p.~72) state that the F-5 could carry 200-gallon FTs (each 1400~lb) on its wing stations. These are only rated for 1000 lb in the ADC from Webster. The station loads need increasing to 1400~lb each and the total load to 2800~lb.
}

The F-10, F-15, and F-20 blocks permit the use of 200 US gallon fuel tanks without maneuver restrictions, greatly enhancing operational range for day-fighter missions. These blocks entered service starting in June 1952.%
\note{
    \noteparagraph{F-86F-10/15/20} ADC originally from Webster. The F-5/F-10/F-15/F-20 are similar (Curtis, p.~71). Many early aircraft converted to 6-3 wing starting in September 1952 (Curtis p.~73). Since the F-25/30 was not produced until October 1952, these must have been F-20s and earlier.
}

The F-25 and F-30 blocks hugely improve the capability of the F-86 as a fighter bomber. They have a second pair of weapon pylons on the inner wing, allowing the aircraft to carry both a pair of bombs or eight HVAR rockets in addition to a pair of fuel tanks. This configuration provides a much improved range when carrying air-to-ground weapons. The F-25 models entered service in Korea in October 1952.%
\note{
    \noteparagraph{F-86F-25/30} ADC adapted from Webster. Early F-25 had the original wing, but later production had 6-3 wing (Curtis p.~73). Many early aircraft were converted to 6-3 wing starting in September 1952 (Curtis p.~73). The F-30 was identical to the F-25, but built at Inglewood rather than Columbus (Curtis, p.~72).
}

The F-35 block is specialized as a tactical nuclear bomber, with a strengthened the left inner pylon to permit carriage of the second-generation Mk~12 nuclear bomb and the LABS (Low Altitude Bombing System) to allow toss bombing. It entered service in January 1954 and was based mainly in Europe.%
\note{
    \noteparagraph{F-86F-35} ADC adapted from Webster. The F-35 could carry a Mk 12 nuclear bomb on its port wing and drop tanks on its starboard wing (Goebel, Air Force). I assumed the Mk 12 has a weight of 1200~lb and 3.0 load points.
}

All the versions mentioned up to this point were built with the basic wing, which had automatic leading-edge slats that increased lift at low speed. In the summer of 1952, North American developed the “6-3 wing” or the “solid 6-3 wing”, which features a 6-inch extension to the chord at the wing root and a 3-inch extension at the wing tip. The resulting larger area gives much better maneuverability at high altitude. The loss of the automatic slats increased the landing speed and the landing roll, and led to accidents when used by inadequately-trained pilots. Starting in September 1952, field-modification kits were provided to add the 6-3 wing to existing F models and later-production F-25/30/35 had them installed in the factory.%
\note{
    \noteparagraph{6-3 Wing} Many early aircraft were converted to 6-3 wing starting in September 1952 (Curtis p.~73). Since the F-25/30 was not produced until October 1952, these must have been F-20s and earlier.

    Webster writes about the 6-3 wing on the F-25:
    \begin{quote}
        “On the data card, this is reflected by reducing the minimum allowed speed at higher altitudes rather than reducing the decel for turning, since the next-lowest gameable decel amount is 0.5, which would skew the data card's power-to-decel relationship too far from reality compared to the other cards. One last thing about the 6-3 wing is that the Sabre's low-speed handling characteristics were degraded, but this applied to the landing regime and is not reflected in game terms. Adequately trained pilots had no problems with it.”
    \end{quote}

    The field conversions were modifications of the wing leading edge, and so would have maintained the same configuration of load stations.

}

The F-86F-40 was initially a version to be manufactured for the JASDF by Mitsubishi under license, but was also built by North American for the USAF and other US allies. It is similar to the F-25, but features a new wing with 12-inch span extensions at the wing-tips and once more automatic leading-edge slats. As such, it combined the superior high-altitude maneuverability of the 6-3 wing with the lower landing speeds of the basic wing; it was known as the “extended 6-3 wing” or just the “F-40 wing”. Deliveries to the USAF and US allies under the MAP (Military Assistance Program) started in 1955. Many earlier F-series aircraft were upgraded to the F-40 standard.%
\note{
    \noteparagraph{F-86F-40} ADC adapted from Webster. The F-40 had the 6-3 wing with automatic slats and 12-inch wing-tip extensions (Curtis, p.~83). Deliveries started in 1955 (Curtis, p.~83). Many earlier F models brought up to F-40 standard before being supplied to other countries through MAP (Curtis., p.~83).
}

\subsubsection{F-86H}

The F-86H further refines the F-86 as a fighter-bomber. It has a more powerful, albeit heavier, J73-GE-3 engine producing 8900 lb of dry thrust and giving a much improved rate of climb. The initial H-1 block maintains the armament of six 0.50 cal machine guns, but the later H-5/10 blocks have four 20~mm M39 guns each with 150 rounds, finally giving the Sabre a potent air-to-air armament, albeit in a model not intended as a fighter. Many were later upgraded with the extended 6-3 wing.%
\note{
    \noteparagraph{F-86H} ADC originally from Webster. This variant has the 6-3 wing.  Used GE J73, which was heavier but more powerful (8900~lb) (Curtis, p.~92). Entered service in 1954 (Curtis, p.~94). Versions with the extended 6-3 wing were produced in 1955, and many earlier H models also converted (Curtis, p.~95).

}

The F-86H entered service in 1954.

\subsection{Interceptor Versions}

The interceptor F-86D/K/L versions of the F-86 are significantly different to the F-86A/E/F/H versions, sharing wings and tail but little else. Informally, they were known as the “Sabre Dog”.

\subsubsection{F-86D}

The F-86D has the J47-GE-17 engine with 5500 lb of dry thrust and an afterburner giving 7650 lb; the afterburner was vital to allow the aircraft to quickly climb to the high altitudes at which it was expected to intercept intruding bombers. The larger fuselage allows more internal fuel, but despite this the F-86D and subsequent interceptors almost always flew with two 120-gallon fuel tanks on wing stations. The F-86D has the basic wing and all-flying tail of the F-86E.%
\note{
    \noteparagraph{F-86D} ADC adapted from Webster.
}

Its weapon system consisted of an APG-7 radar, twenty-four 70 mm FFAR rockets in an extending ventral tray, and a Hughes E-4 fire-control system linking the two. The combination permitted collision-course rocket attacks from the beam, but presented a heavy workload for the single pilot.

Early variants of the aircraft began to be delivered in March 1951, but versions with the E-4 fire control system were only delivered starting in July 1952.

\subsubsection{F-86K}

The F-86K is a simplified interceptor developed from F-86D for export under the Military Assistance Program (MAP). The weapon system was reworked, with four 20 mm M24 guns replacing the FFARs and a Hughes MG-4 fire-control system replacing the more advanced E-4 system. The simpler fire-control system no longer had collision-course attacks was capable of tail-chase attacks. Some F-86Ks have the extended 6-3 wing of the F-40, either as late-production builds or as upgrades.%
\note{
    \noteparagraph{F-86K} ADC originally from Webster. Similar to the F-86D (Curtis, p.~103--104), so presumably also with the original wing. Delivered to Italian AF in 1955 (Curtis, p.~105). Deliveries to Italy, Netherlands, and Germany completed in 1958. Last 45 has the extended 6-3 wing (Curtis, p.~105).

}

Many F-86Ks for European air forces were assembled by Fiat from kits provided by North American; others were built by North American itself.

The F-86K served with the air forces of Italy, Netherlands, and Germany, with deliveries between 1955 and  1958.

\subsubsection{F-86L}

The F-86L is a rebuild of the F-86D. It has the extended 6-3 wing and updated electronics, including a data link to allow the  intercepts to be directly guided by the computerized SAGE system.%
\note{
    \noteparagraph{F-86L} ADC originally from Webster. Has extended 6-3 wing (Curtis, p.~56).
}

The F-86L entered service in late 1957.

\subsection{Armament and Stores}

The F-86A/E/F-1 could use 120-gallon (450L) fuel tanks without maneuver restrictions, but 200-gallon (760L) fuel tanks limited maneuverability to 3G and were only used for ferry flights. The F-86F-5/10/15/20/25/35/40 and F-86H could carry 200-gallon fuel tanks without maneuver restrictions. The F-86D/K/L could only use 120-gallon (450L) fuel tanks, and almost always did.%
\note{
    \noteparagraph{Fuel Tanks}
    \begin{itemize}
        \item F-86A could use 120-gallon (450L) FTs (Curtis, p.~36)
        \item F-86A could use 206-gallon (772L) FTs for ferry missions (Curtis, p. 18-19). These tanks could be dropped, but limited the aircraft to 3G (F-86A Flight Manual, p. 126).
        \item F-86D could use two 120-gallon FTs (Curtis, p.~41)
        \item F-86D-25 first model with ``jettisonable drop tanks'' (Curtis, p. 49). I presume this means earlier models could use external FTs but not jettison them. The ``definitive'' version was the D-45 (Curtis, p.~50).
        \item F-86F-5 was the first Sabre to be able to use the 200-gallon (760L and 1380 lb) FT (Curtis, p.~71).
        \item The new pylon on the F-86F were inboard of the existing pylons for 120-gallon FT or 1000-lb BB (Curtis, p.~72).
        \item F-86F-35 could carry Mk 12 20-kt nuclear bomb on inner left pylon and a FT on the other side (Curtis, p.~82).
        \item F-86H could also carry 1200-lb Mk 12 nuclear bomb (Curtis, p.~93)
        \item F-86L can carry only 120-gallon FTs (SAC from 1958)
    \end{itemize}
}

Starting in 1958, some F-86D/K/L aircraft were retrofitted with two missile stations inboard of the fuel tanks for the Sidewinder missile. This upgrade enhanced their air-to-air combat capability, particularly against more maneuverable or heavily armed adversaries.

\subsection{ADCs}

\begin{adclist}
    \adcitem{F-86A}
    \adcitem{F-86D}
    \adcitem{F-86E}
    \adcitem{F-86F-1}
    \adcitem{F-86F-1 (6-3 Wing)}
    \adcitem{F-86F-5}
    \adcitem{F-86F-5 (6-3 Wing)}
    \adcitem{F-86F-10}
    \adcitem{F-86F-10 (6-3 Wing)}
    \adcitem{F-86F-25}
    \adcitem{F-86F-25 (6-3 Wing)}
    \adcitem{F-86F-35}
    \adcitem{F-86F-35 (6-3 Wing)}
    \adcitem{F-86F-40}
    \adcitem{F-86H}
    \adcitem{F-86H (Extended 6-3 Wing)}
    \adcitem{F-86K}
    \adcitem{F-86K (Extended 6-3 Wing)}
    \adcitem{F-86L}
\end{adclist}

\subsection{See Also}
\begin{typelist}
    \typeitem{CAC Sabre}
    \typeitem{Canadair Sabre}
\end{typelist}

\noteslist
