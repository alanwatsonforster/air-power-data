%!TEX root = ./helicopterdata.tex
%LTeX: enabled=true

\subsection{Bell UH-1 Iroquois}

The Bell UH-1 Iroquois is a medium transport and attack helicopter. It was introduced in 1959 with the US Army as the HU-1 and subsequently was almost universally known by its nickname “Huey” rather than its official name “Iroquois”. In 1962, it was redesignated as UH-1. The D and H versions had stretched cabins to accommodate more passengers.

It is often equipped with two 7.62 mm M60 door guns. During the Vietnam War, some B and C versions were also used as gunships or “airborne rocket artillery”. 

The principal gunship variants use the M6/M16 armament subsystem with four 7.62 mm M60 machine guns or two 7.62 mm M134 miniguns on two forward-facing articulated mounts either side of the fuselage. The gunships retain their door guns and the M16/M21 variants could also carry two fixed M158 \binarymultiply{7}{70 mm} RPs, one on each side of the fuselage. 

The principal rocket variant uses two fixed M3 \binarymultiply{24}{70 mm} RPs, one on each side of the fuselage. Again, it retains the door guns.

The M11/M22 armament subsystems allow the UH-1 to be armed with six the AGM-22A/AGM-22B ATGMs (licensed versions of the French SS.11 ATGM) on racks on either side of the fuselage. The experimental XM26 armament subsystem likewise allows it to use six TOW missiles.

It saw extensive service in the US Army, USMC, and USAF from 1961 until being finally withdrawn in the 21st century, and combat in almost all the wars in which the US was involved during that period. Notably, versions armed with the AGM-22B saw service in South Vietnam in 1966 to 1967 and 1972 and those with TOW in 1972. It also served with many US allies, especially the ARVN in the Vietnam War.
