\section{Ilyushin Il-28}

\includegraphics[width=\linewidth]{Il-28.jpg}

The Ilyushin Il-28 was a conventional and nuclear tactical bomber and a long-range reconnaissance aircraft. The NATO reporting name for the aircraft is “Beagle”.

It was powered by two Klimov VK-1 engines in underwing pods. This engine was a development on the Klimov RD-45, an unauthorized copy of the Rolls-Royce Nene engine, and was also used in the MiG-15bis.

\subsection{Versions}

\subsubsection{Il-28}

The Il-28 was a conventional tactical bomber. It was armed with two fixed 23 mm NR-23 cannons and two more in a defensive tail mount. It could carry up to 3,000 kg (6,600 lb) of bombs in its internal bomb bay, but a normal load was 1,000 kg (2,200 lb).

The Il-28 entered service with the Soviet VVS in 1950 and AV MF (Naval Aviation) in 1951 and with the PLAAF in 1952. They were exported to many other countries, including Afghanistan, Algeria, Bulgaria, Cambodia, Czechoslovakia, East Germany, Egypt, Finland, Hungary, Indonesia, Iraq, Morocco, Nigeria, North Korea, North Vietnam, Romania, Somalia, Syria, and Yemen. They were built under license in Czechoslovakia. China later built a modified version as the Harbin H-5.

\subsubsection{Il-28T}

The Il-28T was a torpedo bomber. Its bomb bay was lengthened to allow the internal carriage of torpedoes and its wing moved slightly to compensate for the resulting change in the center of gravity. It was equipped with two 350L wing-tip fuel tanks to compensate for the reduced capacity of the in fuselage fuel tanks.

The IL-28T entered service with the Soviet AV MF in 1951.

\subsubsection{Il-28R}

The Il-28R was a long-range photo-reconnaissance aircraft. Its bomb bay was given over to cameras, flares, and additional fuel. It also was equipped with two 350L wing-tip fuel tanks. One of the forward-firing guns was replaced by reconnaissance equipment.

The IL-28R entered service with the VVS and AV MF in 1952.

\subsubsection{Il-28N}

The Il-28N was a nuclear tactical bomber. It could carry the RDS-4 nuclear bomb in its internal bomb bay. Like the Il-28R, it was equipped with two 350L wing-tip fuel tanks.

The IL-28N entered service with the VVS in about 1954.

\subsubsection{Il-28REB}

The Il-28REB was an electronic-countermeasures aircraft. Its primary role was to protect Il-28s on conventional and nuclear bombing missions. Like the Il-28R, it was equipped with two 350L wing-tip fuel tanks.

The IL-28REB entered service with the VVS in about 1954.

\subsection{Armament and Stores}

The Il-28 could carry up to 3,000 kg (6,600 lb) of bombs in its internal bay. Alternatively, the Il-28T could carry one Type 45 or two RAT-52 torpedoes. The Il-28N could carry a single RDS-4 nuclear bomb.

\subsection{Combat}

PLAAF Il-28s did not see combat in the Korean War, but were an implicit threat during the last year of the Korean War. They saw action in 1956 against Taiwan, bombing the Tachen Islands, and suffered losses to RoCAF F-84 and F-86s. They were also used in the 1959 Tibetan Uprising.

Egyptian Air Force Il-28s fought in the 1967 War, the War of Attrition, and the 1973 War.

Soviet VVS Il-28s saw combat in the Soviet-Afghan War.

\subsection{ADCs}
\begin{adclist}
    \adcitem{Il-28}
    \adcitem{Il-28T}
    \adcitem{Il-28R}
    \adcitem{Il-28N}
\end{adclist}

\subsection{Photo Credit}
\begin{itemize}\raggedright
    \item \href{https://commons.wikimedia.org/wiki/File:%22Frontbomber%22_Iljuschin_Il-28_%22Beagle%22.jpg}{Il-28}: Bjoern Schwarz (CC BY 2.0)
    \end{itemize}
