\subsection{Mil Mi-24}

The Mil Mi-24 is a Soviet/Russian attack and assault helicopter. Uniquely, it can carry eight passengers in addition to its air-to-ground weapons.

The first Mi-24 (Hind-A) version has a flight crew of three, pilot, a navigator, and a gunner. While it is fast and very well protected by armor, it is not very agile. It can carry four 9M17 Falanga-M (AT-2 Swatter-2) RGs on the wing-tip stations and four UB-32 \binarymultiply{32}{55 mm} or B-8V20A \binarymultiply{20}{80 mm} RPs on its under-wing stations. It also has a single-barrelled 12.7 mm A-12.7 gun in a turret under the nose. The 

The Mi-24D (Hind-D) has a reduced crew of a pilot and gunner and accommodates them in a completely redesigned cockpit. The 12.7 mm A-12.7 gun was replaced by a multi-barrel 12.7 mm YakB-12.7 gun and the missiles upgraded to the much more effective 9M17P Falanga-P (AT-2C Swatter). The Mi-24V (Hind-E) uses the 9M114 Kokon (AT-6 Spiral) missile, but is otherwise very similar.

The Mi-24P (Hind-F) replaces the 12.7 mm gun with a fixed two-barrelled 30 mm GShK-30-2K cannon, but is otherwise similar to the Mi-24V.

As a result of experience in the Soviet-Afghan War, many Mi-24D/V helicopters were fitted with improved countermeasures, including an exhaust cooling system, flares, and a radar warning system. These were fitted as standard on the Mi-24P.

The Mi-24 and Mi-24D entered service with Soviet forces in 1973, the Mi-24V in 1976, and the Mi-24P in 1981. In Soviet service, they saw combat in the Soviet-Afghan War. They also served in the armed forces of the Soviet Union's successor states and in many allies of the Soviet Union, and saw combat notably in the Iran-Iraq War, the Gulf War, and the Invasion of Ukraine.
