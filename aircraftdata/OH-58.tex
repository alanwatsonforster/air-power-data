%!TEX root = ./helicopterdata.tex
%LTeX: enabled=true

\subsection{Bell OH-58 Kiowa}

The Bell OH-58 Kiowa is a scout and light attack helicopter. It initially lost the {\itshape Light Observation Helicopter} competition, but was later ordered by the US Army when the rising cost of the Hughes OH-6 caused bidding to be repeated.

The OH-58A is the initial version. Like the OH-6, it can be fitted with the M27 armament subsystem that mounts a 7.62 mm M134 minigun on the right side of the fuselage and provides it with 2000 rounds of ammunition. The mount can elevate and depress, but not traverse.

The OH-58C version is a conversion of the OH-58A and adds an IR suppression system and RWR. It may be armed with the M27 subsystem or two AIM-92 Stinger IRMs.

The OH-58D version in turn is a modification of the OH-58C and adds a more powerful engine, a four-bladed main rotor, and a mast-mounted sight. The {\itshape Armed} and {\itshape Kiowa Warrior} variants have two mounts on the fuselage sides each of which can carry two AGM-114 Hellfire RGs, two AIM-92 Stinger IRMs, or one M260 \binarymultiply{7}{70 mm} RP. The left pylon can alternatively carry an M296 0.50 cal GP with 500 rounds. The {\itshape Kiowa Warrior} variant also has improved defenses against IRMs.

The OH-58A served with the US Army from 1969 and saw combat in the Vietnam War. The C and D versions entered service in 1976 and 1986, and saw combat in both Gulf Wars. The {\itshape Armed} was developed in 1987, and these variants saw combat in the Persian Gulf protecting oil tankers from attacks by Iranian forces. There are suggestions that the aircraft flew in pairs, one armed with two Hellfire missiles and a .50 cal machine gun and the other with two Stingers and one rocket pod. The {\itshape Kiowa Warrior} variant entered service in 1989 and saw combat in the Second Gulf War, in the Occupation of Iraq, and in Afghanistan.