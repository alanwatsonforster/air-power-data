\section{Shenyang J-6 and F-6}

\includegraphics[width=\linewidth]{Shenyang J-6.jpg}

The Shenyang J-6 was a day fighter, interceptor, and trainer. It was derived from the MiG-19.

\subsection{Versions}

\subsubsection{J-6 and F-6}

The J-6 was a day fighter similar to the MiG-19S. It was manufactured starting in 1961 and continued at least into the 1980s. The F-6 was the export version.

The J-6 was a mainstay of the PLAAF and PLAN until replaced by more modern types in the late 1990s. The F-6 was exported to Albania, Bangladesh, Cambodia, Egypt, Iran, Iraq, Myanmar, North Korea, Pakistan, Sudan, Somalia, Tanzania, Vietnam, and Zambia.

The F-6 was used by the Pakistan Air Force from 1965 to 2002, and was modified with a more modern ejector seat, two additional weapon stations wired to carry AIM-9 IRMs, and the ability to carry a special ventral fuel tank.

\subsubsection{J-6A}

The J-6A was an interceptor similar to the MiG-19P. It was manufactured starting in 1958. From 1975, it could be equipped with PL-2 IRMs.

The J-6A served with the PLAAF.

\subsubsection{J-6B}

The J-6B was a missile-armed interceptor similar to the MiG-19PM and equipped with PL-1 BRMs.

The J-6B served with the PLAAF.

\subsubsection{J-6C and F-6C}

The J-6C was a later version of the J-6, with the brake parachute moved to a fairing at the base of the rudder and other minor changes. The F-6C was the export version.

The J-6C served with the PLAAF. The F-6C was exported to Somalia, Sudan, and Tanzania.

\subsubsection{JJ-6 and FT-6}

The Shenyang JJ-6 was a trainer. It was a modified version of the Shenyang J-6, with a second cockpit and only one cannon. The FT-6 was the export version.

The JJ-6 was used by the PLAAF. The FT-6 was used by Albania, Egypt, Pakistan, Somalia, Sudan, Tanzania, and Zambia.

\subsection{Armament and Stores}

The two internal 30 mm cannons were the primary air-to-air weapons in the early years. In later years, these were supplemented by PL-2 IRMs in the J-6A and PL-1 BRMs in the J-6B. F-6s in the Pakistan Air Force also carried AIM-9 IRMs.

For air-to-ground missions, typical loads would be 500 lb or 250 kg bombs and ORO-57 rocket pods.

Given the short endurance on internal fuel, 760L fuel tanks were carried on almost all missions.

\subsection{Combat}

Chinese PLAAF J-6s shot down a USAF F-104C that had strayed over Hainan Island in 1965. PLAAF J-6s engaged ROCAF F-104Gs during the 1967 Taiwan Strait Conflict

The F-6 saw combat with the Pakistan Air Force in the 1971 Indo-Pakistan War, with the VPAF from 1969 until the fall of South Vietnam in 1975, with the Tanzanian Air Force in the 1978-1979 Uganda-Tanzania War, with both sides in the Iran-Iraq War, with the Somali Air Force in border skirmishes with Ethiopia in 1981 and later during the Somali Rebellion, and with the Sudanese Air Force in the Second Sudanese Civil War in the 1980s and early 1990s.

\subsection{ADCs}

\begin{adclist}
    \adcitem{Shenyang J-6}
    \adcitem{Shenyang F-6}
    \adcitem{Shenyang F-6 (PAF)}
    \adcitem{Shenyang J-6A}
    \adcitem{Shenyang J-6B}
    \adcitem{Shenyang J-6C}
    \adcitem{Shenyang F-6C}
    \adcitem{Shenyang JJ-6}
    \adcitem{Shenyang FT-6}
\end{adclist}

\subsection{See Also}
\begin{typelist}
    \typeitem{Mikoyan-Gurevich MiG-19}
\end{typelist}

\subsection{Photo Credit}
\begin{itemize}\raggedright
    \item \href{https://commons.wikimedia.org/wiki/File:Shenyang_J-6_(cropped).jpg}{J-6}: Alert5 (CC BY-SA 4.0)
\end{itemize}
