\section{Hawker Sea Fury and Fury}

\includegraphics[width=\linewidth]{Sea Fury.jpg}

The Hawker Sea Fury was a post-WW2 propeller-driven fighter bomber.

The Sea Fury was a descendant of the WW2 Hawker Tempest fighter bomber, which was designed to be a long-range fighter for use by the RAF in the war against Japan. After WW2 ended, the RAF no longer had an interest, but the RN acquired a version adapted for carrier operations as a replacement for its Seafires, which were not well suited to carrier operations, and Corsairs, which had to be returned to the US as the Lend-Lease program ended.

The Sea Fury was powered by a Centaur radial engine, armed with four 20 mm guns, and had a bubble canopy with an excellent view except under the nose.

\subsection{Versions}

\subsubsection{Sea Fury F.10}

The initial version of the Sea Fury was the F.10 day fighter.

It entered service in the RN in 1947, but seems to have been quickly replaced by the FB.11 version.

\subsubsection{Sea Fury F.11}

The FB.11 was derived from the F.10 and had added armor and weapon stations to perform better as a fighter-bomber.

It served in the RN from 1948 to 1956, RAN from 1948 to 1959, and RCN from 1948 to 1956. It was replaced in the RN by the Sea Hawk and Attacker starting in 1953, in the RAN by the Sea Venom in 1956, and in the RCN by the F2H Banshees starting in 1956. New and ex-RN FB.11s also served in the Burmese Air Force from 1958 to 1968, the Cuban Air Force from 1957, the Egyptian Air Force, and the Pakistan Air Force.

\subsubsection{Sea Fury T.20}

The T.20 was a two-seater trainer. To compensate for the weight of the additional cockpit, the armament was reduced to two 20 mm cannons.

It served with the RN. Some ex-RN aircraft were later used by the Burmese Air Force from 1958 to 1968 and by the Cuban Air Force from 1957.

\subsubsection{Sea Fury F.50 and FB.50}

The F.50 and FB.50 were export versions of the F.10 and FB.11 with minor changes. Some were license-built by Fokker.

The F.50 and FB.50 were used by the Royal Netherlands Navy from 1947 and were eventually replaced by Sea Hawks.

\subsubsection{Fury and Fury FB.60}

The Fury (with no version designation but sometimes referred to as the “Baghdad Fury”) and the FB.60 were export versions of the FB.11 with carrier-specific equipment removed.

The Fury was used by the Iraqi Air Force from 1946 until 1969, being replaced by the Su-7 starting in 1967. The FB.60 was used by the Pakistan Air Force from 1950 until 1960, being replaced by Sabres starting in 1955.

\subsubsection{Fury Trainer and Fury T.61}

The Fury Trainer (with no version designation) and the T.61 were export versions of the T.20 with carrier-specific equipment removed.

The Fury Trainer was used by the Iraqi Air Force. The T.61 was used by the Pakistan Air Force.

\subsection{Armament and Stores}

A typical air-to-ground load was two 500-lb or 1000-lb bombs or twelve RP-3 rockets. It could also carry two 90-gallon fuel tanks to extend its range.

\subsection{Combat}

The RN and RAN used the Sea Fury as a fighter-bomber in the Korean War. They also saw combat with the Netherlands Royal Navy in the Dutch East Indies, with the Cuban Air Force against Fidel Castro's revolutionaries, and with the Cuban Revolutionary Air Force during the Bay of Pigs invasion.

\subsection{ADCs}

\begin{adclist}
    \adcitem{Sea Fury F.10}
    \adcitem{Sea Fury FB.11}
    \adcitem{Sea Fury T.20}
    \adcitem{Sea Fury F.50}
    \adcitem{Sea Fury FB.50}
    \adcitem{Sea Fury FB.60}
    \adcitem{Sea Fury T.61}
    \adcitem{Fury}
    \adcitem{Fury Trainer}
\end{adclist}

\subsection{Photo Credit}
\begin{itemize}\raggedright
    \item \href{https://commons.wikimedia.org/wiki/File:Hawker_Fury_FB.11_‘SR661’_(G-CBEL)_(35902999510).jpg}{Hawker Sea Fury}: Alan Wilson (CC BY-SA 2.0)
\end{itemize}
