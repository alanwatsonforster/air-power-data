\section{Mikoyan-Gurevich MiG-17}

\includegraphics[width=\linewidth]{MiG-17 alternative.jpg}

The Mikoyan-Gurevich MiG-17 was an early jet interceptor and day fighter, with limited capability as a fighter-bomber.

\subsection{Versions}

\subsubsection{MiG-17}

The MiG-17 day fighter was based on the MiG-15bis, but had a thinner but stiffer wing and tail surfaces for better control at higher Mach number and a longer fuselage, but retained conventional elevators. It maintained the Klimov VK-1A non-afterburning engine and the heavy armament of the MiG-15bis. The NATO reporting name is Fresco-A.

The MiG-17 entered service with the Soviet VVS in 1952. It was subsequently widely exported to dozens of Soviet allies, including notably the Chinese PLAAF and PLAN, the North Korean KPAF (after the Korean War), the North Vietnamese VPAF, the Egyptian and Syrian Air Forces.

\subsubsection{MiG-17F}

The MiG-17F had a VK-1F afterburning engine but was otherwise almost identical to the MiG-17. The afterburner gave a dramatic improvement in climb rate, which was important in an interceptor. The MiG-17F was the first Soviet operational aircraft to be equipped with an afterburner. The NATO reporting name is Fresco-C.

The Polish PZL-Mielec Lim-5 and Chinese Shenyang J-5/F-5 were licensed versions of the MiG-17F.

The MiG-17F entered service with the Soviet VVS in 1953. Like the preceding MiG-17, it was subsequently widely exported to dozens of Soviet allies, including notably the Chinese PLAAF and PLAN, the North Korean KPAF (after the Korean War), the North Vietnamese VPAF, the Egyptian and Syrian Air Forces.

\subsubsection{MiG-17P}

The MiG-17P all-weather interceptor was derived from the original MiG-17 and the prototype MiG-15P. It had an RP-1 Izumrud (Scan Odd) radar mounted in front of the air intake and kept the non-afterburning engine. The NATO reporting name is Fresco-B.

The MiG-17P entered service with the Soviet PVO and AV MF in 1953.

\subsubsection{MiG-17PF}

The MiG-17PF all-weather interceptor was a development of the MiG-17P. It gained the VK-1F afterburning engine and had an armament of three NR-23 guns. The NATO reporting name is Fresco-D.

The Polish PZL-Mielec Lim-5P and Chinese Shenyang J-5A/F-5A were licensed versions of the MiG-17PF.

The MiG-17PF served with the Soviet PVO from 1953. It also served with the Bulgarian Air Force, the Chinese PLAAF (both as original versions and as J-5As), the Hungarian Air Force, and the Polish Air Force (from 1955).

\subsubsection{MiG-17PFU}

The MiG-17PFU all-weather interceptor was a further development of the MiG-17PF. The major change was the replacement of the gun armament with four Kalingrad K-5 (AA-1 Alkali) BRMs guided by the RP-2 Izumrud radar. The NATO reporting name was Fresco-E.

The MiG-17PFU served with the Soviet PVO from 1956.

\subsection{Armament and Stores}

The primary air-to-air armament of the MiG-17 was its cannon. The K-13 (AA-2 Atoll) IRM became available in 1961, but it was not widely used as by then the MiG-17 was typically transitioning to a ground-attack role. The MiG-17PFU was armed with Kalingrad K-5 (AA-1 Alkali) BRMs.

To counter its short endurance, the MiG-17 could use 250, 300, 400, or 600 liter fuel tanks on its under-wing stations. The 600 liter tanks reduced maneuverability. It could also be armed with 250~kg bombs or, from 1955, with ORO-57 rocket pods each with eight 57~mm rockets.

\subsection{Combat}

A number of US reconnaissance flights were shot down by PVO MiG-17P/PFs between 1953 and 1970.

Chinese PLAAF MiG-17s clashed with ROCAF F-86s in the Second Taiwan Strait Crisis in 1958. They also intercepted ROCAF reconnaissance flights over mainland China.

From 1965 to 1972, the MiG-17/17F saw combat with the VPAF in the Vietnam War, initially embarrassing much more modern adversaries with its small size, maneuverability, and cannon armament. The MiGs were largely used as interceptors, but in 1972 two attacked USS Higbee and USS Oklahoma City as they carried out a naval gunfire support mission.

During the 1967 War, the War of Attrition, and the 1973 War, Egyptian and Syrian MiG-17/17F aircraft served in a ground-attack role.

In the late 1960s and the 1970s, MiG-17/17Fs were widely used in ground-attack roles as governments in Africa and Asia countered insurgencies.

\subsection{ADCs}

\begin{adclist}
    \adcitem{MiG-17}
    \adcitem{MiG-17F}
    \adcitem{MiG-17P}
    \adcitem{MiG-17PF}
    \adcitem{MiG-17PFU}
\end{adclist}

\subsection{See Also}
\begin{typelist}
    \typeitem{Shenyang J-5 and F-5}
\end{typelist}

\subsection{Photo Credit}
\begin{itemize}\raggedright
    \item \href{https://commons.wikimedia.org/wiki/File:MiG-17_Takes_to_the_Sky_(cropped).jpg}{MiG-17}: Balon Greyjoy (Public Domain)
\end{itemize}
