\section{Lockheed P2V Neptune}

The Lockheed P2V Neptune is a multiple-engined maritime patrol and anti-submarine aircraft. Early versions had two rotary engines, but later versions added two auxiliary jet engines. In 1962, it was redesignated as the P-2.

In 1954, Lockheed converted five P2V-7s into P2V-7U electronic intelligence (ELINT) aircraft for the CIA. Armament was removed and in its place a wide and variable range of ELINT equipment was installed.

There were nominally assigned to the USAF and designated RB-69A. From 1957 to 1959, they were used by CIA crews for missions in Europe. From 1957 to 1966, they were flown by ROCAF crews from the 34th “Black Bats” Squadron on ELINT missions over mainland China. Five were lost, with at least three being shot down.

The normal load for an RB-69A would be two 200 gal (760L) FTs on the wing tips. From 1964, AIM-9B IRMs could be carried, but the low speed of the aircraft reduced the chance of a successful launch.

An ADC is provided for the:
\begin{adclist}
    \adcitem{P2V-7U}
\end{adclist}
