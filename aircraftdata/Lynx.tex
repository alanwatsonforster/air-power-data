%!TEX root = ./helicopterdata.tex
%LTeX: enabled=true

\subsection{Westland Lynx}

The Westland Lynx is an attack, anti-submarine, anti-ship, and utility helicopter. It was developed by Westland and produced in collaboration with Sud Aviation (which later became part of Aérospatiale). An adapted Lynx holds the world speed record for helicopters.

There are many attack (AH), anti-submarine (HAS), and maritime attack (HMA) versions. The attack versions are armed with four TOW RGs. The naval versions can be armed with torpedos, depth charges, four Sea Skua RGs (in the RN) or four AS-12 missiles (in the French Navy). The naval versions also incorporate a search radar. All can use a door-mounted machine gun, typically a 7.62 mm gun in earlier versions but a .50 cal in the later AH.9A variant.

Important upgrades in the 1990s were the addition of TV/IR Optics to the naval versions and countermeasures in the attack versions.

The Lynx AH entered service in the British Army in 1979. After the introduction of the Apache in 2024, the Lynx was relegated to a utility role and finally retired in 2018. The attack version saw combat in the Gulf War, the invasion and occupation of Iraq, and Afghanistan. 

The Lynx HAS/HMA entered service with the Royal Navy in 1981 and was retired in 2017. It saw combat in the South Atlantic War (on both sides). Naval versions also served with the French Navy and many other navies.