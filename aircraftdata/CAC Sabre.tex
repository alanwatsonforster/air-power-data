%!TEX root = ../aircraftdatacards/australian.tex

\section{CAC Sabre}

\includegraphics[width=\linewidth]{CAC Sabre.jpg}

The Commonwealth Aircraft Corporation (CAC) Sabre is a day fighter and fighter-bomber derived from the North American F-86F Sabre with a lighter and more powerful Rolls-Royce Avon 20 engine replacing the General Electric J47 and two 30 mm ADEN cannons replacing the six .50 cal machine guns. Because of its engine, it is also known informally as the Avon Sabre.%
\note{
    \noteparagraph{Sources}
    \begin{itemize}[nosep]
        \item Curtis, “North American F-86 Sabre”, 2000, Crowood.
        \item Farquhar, “\href{https://adf-serials.com.au/2a94intro.htm}{A Brief History of the CAC Avon Sabre}” (\href{https://web.archive.org/web/20260117031456/https://adf-serials.com.au/2a94intro.htm}{Wayback}). Development and RAAF service.
        \item Webster, “The F-86 Sabre Family Tree”, APJ 25. This has ADCs for the Mk.31 and Mk.32.
        \item \href{https://en.wikipedia.org/wiki/ADEN_cannon}{Wikipedia on the ADEN cannon}.
        \item \href{https://en.wikipedia.org/wiki/CAC_Sabre}{Wikipedia on the CAC Sabre}.
    \end{itemize}
}

\subsection{Versions}

\subsubsection{CAC Sabre Mk.30}

The first version was the Mk.30. It had the early slatted wing from the A, E, and early F versions of the F-86.%
\note{
    \noteparagraph{Mk.30} The ADC for the Mk.30 is derived from that for the Mk.31.

    \noteparagraph{Wing} Essentially, the Mk.30 is a Mk.31 with the earlier slatted wing (Curtis, Wikipedia). The Mk.31 has the unslatted 6-3 wing. This situation is similar to the F-86F, which also had versions with the early slatted and unslatted 6-3 wings. The minimum speeds and turn drags of the Mk.31 match those of the F-86F-25 with the unslatted 6-3 wing. Therefore, I have created an ADC for the Mk.30 by modifying that of the Mk.31 with the minimum speeds and turn drag of the F-86F-25 with the slatted wing. As the F-86A/E/F with the early slatted wing are HTD, I have made the Mk.30 HTD too.
}

It served in the RAAF starting in 1954, replacing the Meteor F.8.

\subsubsection{CAC Sabre Mk.31}

The Mk.31 used the unslatted 6-3 wing from later F versions of the F-86, which gave better performance at higher speeds and altitudes, but was otherwise similar to the Mk.30. The Mk.31 was upgraded in 1960 with two additional weapon stations to allow it to carry missiles or bombs and fuel tanks simultaneously, like the Mk.32.%
\note{
    \noteparagraph{Mk.31} The ADC for the Mk.31 is adapted from Webster's in APJ 25.
}

The Mk.31 served in the RAAF from 1955 to 1971, when they were replaced by the Mirage III. All surviving Mk.30 aircraft were upgraded to the Mk.31 standard. The Mk.31 later served in the Malaysian and Indonesian air forces.

\subsubsection{CAC Sabre Mk.32}

The Mk.32 is a development of the Mk.31 with an Avon 26 engine, modifications to prevent surges when the guns were fired, and two additional weapon stations that allowed it to carry air-to-ground weapons and fuel tanks simultaneously, giving it a longer range when employed as a fighter-bomber. The Mk.32 competes with the Canadair Sabre 6 for the honor of being the very best day-fighter Sabre.%
\note{
    \noteparagraph{Mk.32} The ADC for the Mk.32 is adapted from Webster's in APJ 25.

    \noteparagraph{Fuel} Curtis states that the “wet wing” Mk.32s had an internal fuel capacity of 422 gallons (1899L) and with an additional 60 gallons (270L) in the wing leading edges. The additional fuel corresponds to about 23 fuel points.
}

The Mk.32 served in the RAAF from 1956 to 1971, when they were replaced by the Mirage III, and later in the Malaysian and Indonesian air forces.

\subsection{Service}

CAC Sabres served with the RAAF in the Malayan emergency, the Malaysian-Indonesian confrontation, and the Vietnam War in Thailand.

\subsection{Armament and Stores}

The CAC Sabres were armed with two 30 mm ADEN cannons each with 162 rounds.%
\note{
    \noteparagraph{Guns} Two 30 mm ADEN guns with 162 rounds per gun (Wikipedia). At 1200 rounds per minute, this is 4.0 shots.
}

For additional endurance, all versions can carry 167 gallon (760L) fuel tanks on the outer station, and those with inner station can also carry 100 gallon (450L) on these.%
\note{
    \noteparagraph{FTs} Curtis (p.~114) states that the Mk.32 could carry 100 gallon (450L) and 167 gallon (760L) FTs (perhaps on the inner pylons). Farquhar states that the Mk.32 could carry 100 gallon FTs on the inner pylons and 167 gallon FTs on the outer ones. Wikipedia states they could carry 200 gallon (900L) FTs on the outer pylons, but this seems to be a confusion between US and Imperial gallons, since 167 Imperial gallons are about 200 US gallons. I am assuming the capacity of the outer station tanks is unchanged in these different versions.
}

A typical air-to-air load would be two 167 gallon (760L) fuel tanks on the outer stations and, from 1960, two AIM-9Bs on the inner stations.

A typical air-to-ground load might be two 1000 lb bombs on the inner stations along with two 167 gallon tanks on the outer stations. Alternatively, the Mk.32 could carry twenty-four HVARs or Hispano Sura 80R rockets, with three carried one below the other on each of the eight rocket stations, but at the price of not being able to use external fuel tanks.%
\note{
    \noteparagraph{Load Limits}
    Webster gives weight limits of 1000~lb for both the inner and outer stations. However, Farquhar states that the Mk.32 could carry a 200 US gallon (167 gallon) fuel tanks on the outer pylons, and these weigh about 1400~lb. Curtis also states that the Mk.32 could carry these tanks, but does not specify on which station.

    The early F-86s (A/E/F-1/F-10/F-15/F-20) have load limits of 1400~lb on their outer stations; the later F-86s have the same load limits on their outer stations and 1000~lb limits on the inner stations. I have adopted these limits.

    Wikipedia gives the load limits of the F-86F and Mk.32 as 5300~lb and states they can both carry two 1000~lb bombs and two 200 US gallon FTs, which would be a total of about 4800~lb.  I have increased the total loads to 3000~lb for the Mk.30/31 and 5000~lb for the Mk.32.

    Farquhar states that eight or twenty-four HVARs could be carried.

}

For ferry flights, two 100 gallon (450L) fuel tanks could be carried on the inner stations and two 167 gallon (760L) fuel tanks to the outer ones.

In 1960, the RAAF adopted the AIM-9B IRM for both the Mk.31 and Mk.32.

\subsection{ADCs}

\begin{adclist}
    \adcitem{CAC Sabre Mk.30}
    \adcitem{CAC Sabre Mk.31}
    \adcitem{CAC Sabre Mk.31 (1960 Upgrade)}
    \adcitem{CAC Sabre Mk.32}
\end{adclist}

\subsection{See Also}

\begin{typelist}
    \typeitem{Canadair Sabre}
    \typeitem{North American F-86 Sabre}
\end{typelist}

\noteslist

\subsection{Photo Credit}
\begin{itemize}\raggedright
    \item \href{https://commons.wikimedia.org/wiki/File:Royal_Australian_Air_Force,_on_loan_to_the_Temora_Aviation_Museum,_(VH-IPN,_former_military_registration_A94-983)_CAC_Sabre_Mk.32_landing_at_Avalon_during_the_2015_Australian_International_Airshow.jpg}{CAC Sabre}: Bidgee (CC BY-SA 3.0)
\end{itemize}
