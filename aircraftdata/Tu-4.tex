\section{Tupolev Tu-4}

\includegraphics[width=\linewidth]{Tu-4.jpg}

The Tupolev Tu-4 was a propeller-driven strategic bomber.  As it provided the Soviet Union for the first time with the ability to conduct a one-way strike on peripheral cities in the continental US, including Los Angeles and Chicago, it spurred the development and deployment of defensive interceptors and SAMs. This effort gained further urgency when the Soviet Union demonstrated its first nuclear bomb. The NATO reporting name for the Tu-4 is Bull.

\subsection{Versions}

\subsubsection{Tu-4}

The Tu-4 was largely reverse-engineered from Boeing B-29As that had made emergency landings in the USSR during WWII and were subsequently interned. It was, however, fitted with Soviet Shvetsov ASh-73 engines and auxiliary equipment. Furthermore, the .50 cal machine guns on the original B-29A were replaced by more powerful 23~mm NS-23 cannons, with two in each turret and two in the tail position. The RPB Kobal’t attack radar was a copy of the B-29's APQ-13 radar; its NATO reporting name is Mushroom.

The Tu-4 entered service with the Soviet DA VS (Long-Range Aviation) in large numbers in 1949. In addition to service in the DA VS, a small number were used by the Chinese PLAAF from 1953.

\subsubsection{Tu-4A}

The Tu-4A version was a nuclear bomber and the counterpart of the Silverplate and Saddletree variants of the B-29A. Armament and armor were sacrificed to give longer range.

The Tu-4A served only in the Soviet DA VS.

\subsubsection{Tu-4P}

The Tu-4P version was a conversion carried out by the Chinese PLAAF to create a night fighter specifically to counter ROCAF intruders. The navigation radar was moved from its normal ventral position to a radome in place of the forward dorsal turret, creating a basic all-round air-search radar. The bomb bay was used as an air-intercept command post, and the guns were equipped with a basic infrared sight.

A few Tu-4P aircraft served with the PLAAF in 1960.

\subsection{Armament and Stores}

A typical bomb load for the conventional Tu-4 would be six 1,000~kg bombs in the internal bays.

The nuclear Tu-4A could carry the RDS-1, -3, and -5 nuclear bombs.

\subsection{Combat}

Only the Tu-4P saw combat.

\subsection{ADCs}
\begin{adclist}
    \adcitem{Tu-4}
    \adcitem{Tu-4A}
    \adcitem{Tu-4P}
\end{adclist}

\subsection{See Also}
\begin{typelist}
    \typeitem{Boeing B-29 Superfortress}
\end{typelist}

\subsection{Photo Credit}
\begin{itemize}\raggedright
    \item \href{https://commons.wikimedia.org/wiki/File:TU-4-MONIN0.jpg}{Tu-4}: Pavel Adzhigildaev (CC BY-SA 3.0)
\end{itemize}
