\section{Dassault Mirage 5}

\includegraphics[width=\linewidth]{Mirage 5.jpg}

The Mirage 5 is a day fighter and strike aircraft. It was developed by Dassault at the suggestion of the IAF, and to a large degree is a Mirage IIIE with the radar and other all-weather avionics removed and replaced by additional fuel and with an additional pair of weapon stations. In some later versions, more compact electronics allowed all-weather capabilities to be restored.

\subsection{Versions}

\subsubsection{Mirage 5F}

The Mirage 5F was the original version built for Israel but eventually being used by France.

These aircraft were originally built for Israel as the 5J. However, when delivery to Israel was blocked by the 1967 embargo, they entered service with the Armée de l'air as the 5F around 1970.

\subsubsection{Mirage 5BA}

The 5BA was a strike version for the Belgian Air Force. Changes included US avionics. It entered service in 1970 and was retired at the end of the 1980s.

\subsection{Armament and Stores}

The internal armament of the Mirage 5 was two 30~mm DEFA cannons. The external options were similar to the Mirage III, although RHMs were obviously not used on models without radar.

\subsection{Combat}

\subsection{ADCs}
\begin{adclist}
    \adcitem{Mirage 5F}
    \adcitem{Mirage 5BA}
\end{adclist}

\subsection{See Also}
\begin{typelist}
    \typeitem{Dassault Mirage III}
\end{typelist}

\subsection{Photo Credit}
\begin{itemize}\raggedright
    \item \href{https://commons.wikimedia.org/wiki/File:Chile_Air_Force_Dassault_(SABCA)_Mirage_5MA_Elkan_Lofting-2.jpg}{Mirage 5}: Chris Lofting (GFDL 1.2)
\end{itemize}
