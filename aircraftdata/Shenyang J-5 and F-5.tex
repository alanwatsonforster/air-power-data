\section{Shenyang J-5 and F-5}

\includegraphics[width=\linewidth]{MiG-17.jpg}

The Shenyang J-5 was a day fighter, interceptor, and trainer. The initial versions were licensed copies of the MiG-17F and PF, but the JJ-5 trainer was a subsequent independent development.

A number of Soviet-built MiG-17s served in the PLAAF as the J.4, but this version was not manufactured in China.

\subsection{Versions}

\subsubsection{J-5 and F-5}

The J-5 (also known as the Type 56 and the Dongfeng-101) was a licensed copy of the MiG-17F day fighter. It was produced from about 1956 until about 1969. The F-5 was the export version of the J-5. The NATO reporting name for both is Fresco-C.

The J-5 served with the PLAAF and PLAN, alongside the original MiG-17PF, from 1956 until about 1990. It was also exported to the air forces of Albania, Bangladesh, Cambodia, Indonesia, North Korea, Pakistan, Sri Lanka, Sudan, Vietnam, and Zambia.

\subsubsection{J-5A}

The J-5 was a licensed copy of the MiG-17PF interceptor. It was produced from about 1964 until about 1969. The J-5A was apparently not exported. The NATO reporting name for both is Fresco-D.

The J-5A served with the PLAAF, alongside the original MiG-17PF, from 1964 until about 1990.

\subsubsection{JJ-5 and FT-5}

The JJ-5 was a trainer developed by Shenyang and based on the J-5. It has a similar cockpit configuration to the MiG-15UTI, the non-afterburning Klimov VK-1A engine from the early MiG-17, and retained only the belly 23 mm cannon. The FT-5 was the export version of the JJ-5.

The J-5 served with the PLAAF and PLAN from 1968. It was also exported to the air forces of Albania, Bangladesh, North Korea, Pakistan, Sudan, Tanzania, Zambia, and Zimbabwe.

\subsection{Armament and Stores}

Typical load for an air-to-air missile would be fuel tanks for endurance or, from 1961, two AA-2 IRMs. A typical load for air-to-ground missions was two 250 kg bombs.

\subsection{Combat}

Chinese J-5s and J-5As saw combat against ROCAF aircraft in the late 1950s and early 1960s. Vietnamese F-5s (alongside original MiG-17Fs) were used by the VNAF during and after the Vietnam War.

\subsection{ADCs}
\begin{adclist}
    \adcitem{Shenyang J-5}
    \adcitem{Shenyang F-5}
    \adcitem{Shenyang J-5A}
    \adcitem{Shenyang JJ-5}
    \adcitem{Shenyang FT-5}
\end{adclist}

\subsection{See Also}
\begin{typelist}
    \typeitem{Mikoyan-Gurevich MiG-17}
\end{typelist}

\subsection{Photo Credit}
\begin{itemize}\raggedright
        % I can't find a photo of a real J-5.
    \item \href{https://commons.wikimedia.org/wiki/File:MiG-17_Take_to_the_Skies_Airfest_Durant,_Oklahoma_4.jpg}{MiG-17}: Balon Greyjoy (Public Domain)
\end{itemize}
