%!TEX root = ./helicopterdata.tex
%LTeX: enabled=true

\subsection{Bell AH-1 Cobra}

The Bell AH-1 Cobra is an attack helicopter originally derived from the UH-1 Iroquois utility helicopter.

The first AH-1G version is an interim attack helicopter designed to escort UH-1s in the Vietnam War. It uses the drive train of the UH-1 with a new, thin, armored cockpit. It is armed with a turret mounting two 7.62 mm M134 miniguns (or one minigun and one rocket launcher). Its stub wings have four weapon stations for two \binarymultiply{19}{70 mm} and two \binarymultiply{7}{70 mm} RPs or alternatively two M18 minigun pods in place of the larger RPs. The 1969 M35 variant has a 20 mm M195 Vulcan cannon on the left inner pylon and provides 950 rounds of ammunition. The 1972 upgrade variant has an AN/ALQ-144 IRCM unit and modified exhaust system to defend against the SA-7 IRM.

After the Vietnam War, the US Army refocused on a possible war in Europe, and adapted AH-1s to carry the TOW missile. These were the AH-1Q/S/P/E/F versions, each of which could carry eight TOW RGs. There were many incremental improvements in this series, but some notable differences were that the S/P could not carry RPs and the E adopted the 20 mm M197 cannon from the AH-1T Sea Cobra.

The AH-1G entered service with the US Army in 1967 and the AH-1Q/S/P/E/F in the late 1970s. The AH-1 was retired by the US Army in 1999.

\subsection{Bell AH-1 Sea Cobra, Super Cobra, and Viper}

The Bell AH-1 Sea Cobra is a two-engined version of the AH-1 Cobra developed for the USMC.

The AH-1J is armed with a turret mounting the 20 mm M197 cannon instead of two miniguns, but its armament is otherwise largely similar to the AH-1G. Its stub wings have four weapon stations for two \binarymultiply{19}{70 mm} and two \binarymultiply{7}{70 mm} RPs. The 1972 upgrade variant has an AN/ALQ-144 IRCM unit and modified exhaust system to defend against the SA-7 IRM.

The AH-1J was developed into the AH-1T, which has the capacity to carry two launchers each with four TOW RGs in place of the larger RPs.

The AH-1W Super Cobra is based on the AH-1T, but has significantly updated systems. In addition to the usual RPs, it can also carry two launchers each for four AGM-114 Hellfire RGs (in place of the TOW missile), two AIM-9L IRM (one on each of the outer stations), and AGM-122 ARM (one on each of the outer stations).

The AH-1Z Viper is another iteration, with improved systems and two new wing-tip stations that can carry either AIM-9L IRMs or (on the right side) the Longbow attack radar. It can also carry up to sixteen AGM-114 Hellfire RGs (one launcher on each wing station each with four missiles) or eight AGM-179 JAGM RG missiles (one launcher on each inner wing station each with four missiles).

The AH-1J entered service in 1970, the AH-1T in the late 1970s, the AH-1W in 1986, and the AH-1Z in 2011.