\section{Rockwell B-1 Lancer}

\includegraphics[width=\linewidth]{B-1B.jpg}

The Rockwell B-1 is a long-range strategic bomber.

\subsection{Versions}

\subsubsection{B-1A}

The original B-1A was designed and developed as a long-range, high-speed strategic nuclear bomber to replace the B-52. Four prototype aircraft were built. However, the project was cancelled in 1977 to a large degree because it was thought that the B-52 with air-launched cruise missiles could provide an equivalent capability at a lower cost.

The B-1A did not progress past the prototype stage.

\subsubsection{B-1B}

The B-1 project was revived in the 1980s. A total of 100 B-1Bs were acquired by the USAF, with modifications that gave lower radar cross-section and higher speed at low altitude at the cost of much reduced speed at high altitude, an improved defensive countermeasures suite, and a secondary conventional role. Although the B-1B is formally named the “Lancer,” informally it is called the “Bone”.

The B-1B entered service with the USAF in 1986, although full operability of its TFR and ECM systems took several years.

\subsubsection{B-1R}

The B-1R is a proposed modernization of the B-1B presented in 2004. It would have had more powerful Pratt \& Whitney F119 engines, the air intakes reverted to a high-speed configuration (with a concomitant increase in radar cross-section), provision for weapons on the external stations, including AIM-120 AMRAAM AHMs for self-defense, a dual-mode radar able to combine air-to-air and air-to-ground modes, and improved defensive countermeasures.

The B-1R did not progress beyond a conceptual proposal.

\subsection{Armament and Stores}

\subsubsection{B-1A}

The B-1A is equipped with three internal bays. Each internal bay can carry a rotary launcher with either eight AGM-69A SRAM nuclear ASMs or eight B43 or B61 nuclear BBs or a 2,975 gal (11,200L) fuel tank.

Had in entered service, possible upgrades could have included the AGM-69B SRAM nuclear ASM and the B77 nuclear BB.

\subsubsection{B-1B}

The B-1B is equipped with three internal bays and eight external weapon stations on the fuselage and wing gloves. The six fuselage stations have higher capacity than the two wing-glove stations. Use of the external stations causes a significant increase in the radar cross-section.

Each internal bay can carry either a Multi-Purpose Rotary Launcher (MPRL), a Conventional Bomb Module (CBM), or a 2,975 gal (11,200L) fuel tank. Except for in the first few aircraft, the two forward bays can also be combined to carry a Common Strategic Rotary Launcher (CSRL) and a 1,500 gal (5,700L) fuel tank.

At its entry to service in 1986, the nuclear weapon options of the B-1B were eight B61/B63 nuclear BB or eight AGM-69A SRAM nuclear ASM in each MPRL. The SRAM was retired in 1990, the B-1B removed from the nuclear role in 1997, and the ability to carry nuclear weapons was disabled in 2011 to comply with the New START treaty.

The B-1B was tested for the internal carriage of eight AGM-86B ALCM or AGM-129A ACM nuclear ASMs in the CSRL and for the external carriage of fourteen ALCMs or ACMs on the external stations (two on each of the six fuselage stations and one on each of the two wing-glove stations). The vibration levels were too severe for the external carriage of the ALCM, but external carriage of the more robust ACM was feasible. Nevertheless, neither of these options became operational as the USAF deployed the ALCM and ACM exclusively on the B-52 in compliance with the (unratified) SALT II treaty.

In addition to cruise missiles, the fuselage external stations were designed to each carry six Mk~84 bombs or one 1,000~gal (3,800L) FTs, and the wing-glove stations were designed to each carry four Mk~82 bombs, but these options were never deployed.

The conventional weapons load was initially restricted to twenty-eight Mk~82 500~lb bombs in a CBM. The Mk~82 bombs could be high-drag or low-drag, and the similar Mk~36 or Mk~62 mines could be used as well.

The conventional weapons options were expanded considerably during the aircraft's service with a series of upgrades:
\begin{itemize}\sloppy

    \item The 1996 upgrade allowed each CBM to carry ten CBU-87/89/97 cluster BBs.

    \item The 1998 upgrade allowed each MPRL to carry eight Mk~84 2,000~lb BBs or eight GBU-31 2,000~lb JDAM BSes, specifically the GBU-31(V)1/B with the standard Mk~84 2000~lb bomb as a warhead and the GBU-31(V)3/B with the BLU-109/B 2000~lb penetration bomb.

    \item The 2003 upgrade allowed the MPRL to carry eight AGM-154A JSOW BSes, but this weapon was not deployed operationally. It also allowed each CBM to carry ten CBU-103/104/105 wind-corrected cluster BBs or five GBU-38 JDAM BSes with a Mk~82 500~lb bomb as a warhead.

    \item The 2005 upgrade allowed each MPRL to carry eight AGM-158A JASSM ASMs. It also allowed each weapon bay to carry a different conventional weapon option. Prior to this, all weapon bays had to carry the same type of launcher and each launcher had to carry the same type of weapon.

    \item The 2008 upgrade recommissioned external station 1 for the AAQ-33 Sniper DP/LP and allowed the use of the laser-guided GBU-54 JDAM BS/BG in place of the GBU-38 BS.

    \item The 2014 upgrade allowed each MPRL to carry eight AGM-158B JASSM-ER ASMs.

    \item The 2018 upgrade allowed each MPRL to carry eight AGM-158C LRASM ASMs.

    \item The 2022 upgrade allowed each MPRL to carry a mixture of GBU-31 and GBU-38 JDAM BSes, with two GBU-38s replacing one GBU-31.

\end{itemize}

\subsubsection{B-1R}

I have assumed that the internal weapon options for the B-1R are the same as for the B-1B. I have assumed the external weapon stations can each carry two AMRAAMs or any air-to-ground weapon that can be carried internally.

\subsection{Combat}

The B-1B saw combat in 1998 in the Operation Desert Fox bombing campaign against Iraq, in 1999 in the Operation Allied Force armed intervention in Serbia and Kosovo, in the 2001-2021 War in Afghanistan, in the 2003 Invasion of Iraq, and in the 2011-2024 Syrian Civil War.

\subsection{ADCs}
\begin{adclist}
    \adcitem{B-1A}
    \adcitem{B-1B}
    \adcitem{B-1R}
\end{adclist}

\subsection{Photo Credit}
\begin{itemize}\raggedright
    \item \href{https://commons.wikimedia.org/wiki/File:B-1B.jpg}{B-1B}: Andy Dunaway (Public Domain)
\end{itemize}
