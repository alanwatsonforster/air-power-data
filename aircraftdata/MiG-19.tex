\section{Mikoyan-Gurevich MiG-19}

\includegraphics[width=\linewidth]{MiG-19.jpg}

The Mikoyan-Gurevich MiG-19 was a day fighter and interceptor. Its NATO reporting name is Farmer.

\subsection{Versions}

\subsubsection{MiG-19}

The initial MiG-19 day fighter was significantly different in design from its predecessor, the MiG-17, as it had twin Mikulin RD-9 engines. It was the first Soviet production aircraft capable of supersonic speed in level flight. The initial versions were the only versions with a conventional (not an all-flying) tail. The armament was three 23mm NR-23 cannons, two in the wings and one under the nose. Its NATO reporting name is Farmer-A.

The MiG-19 entered service with the Soviet VVS in 1955.

\subsubsection{MiG-19S and MiG-19SF}

The MiG-19S was an improved version of the MiG-19. It incorporated an all-flying tail and updated avionics.

The early versions of the MiG-19S kept the armament of three 23~mm NR-23 guns used in the initial MiG-19, but later versions switched to three 30~mm NR-30 guns. The NR-30 has slightly better rate of fire and muzzle velocity compared to the NR-23 and fires a much heavier shell. To a large degree, it combines the dogfight characteristics of the NR-23 with the anti-bomber characteristics of the N-37.

The MiG-19SF has slightly more powerful engines, but is otherwise very similar to the late MiG-19S.

The NATO reporting name of the MiG-19S/SF is Farmer-C.

The MiG-19S was built under license in Czechoslovakia as the Aero S-105 and in China as the Shenyang J-6/F-6.

The MiG-19S entered service with the Soviet VVS in 1956. It later served with the Afghan Air Force, the Albanian Air Force, the Bulgarian Air Force, the Chinese PLAAF, the Czechoslovak Air Force, the East German Air Force, the Egyptian Air Force, the Indonesian Air Force, the Iraqi Air Force, and the Pakistan Air Force (in addition to Shenyang F-6s).

\subsubsection{MiG-19P}

The MiG-19P was an all-weather interceptor. It was based on the MiG-19S, but had a modified nose and cockpit incorporating the RP-1 Izumrud (Scan Odd) radar and had only the guns in the wing roots. Again, the early versions had two 23~mm NR-23 guns, but later versions had two 30~mm NR-30 guns. In addition to underwing fuel tanks, it often carried two underwing rocket pods for use against bombers. Its NATO reporting name is Farmer-B.

The MiG-19P was built under license in China as the Shenyang J-6A.

The MiG-19P entered service with the Soviet PVO, VVS, and AV-MF in 1956. It later served with the Bulgarian Air Force, the Chinese PLAAF, the Czechoslovak Air Force, the Iraqi Air Force, the Polish Air Force, and the Romanian Air Force.

\subsubsection{MiG-19PM}

The MiG-19PM was also an all-weather interceptor. It was essentially a MiG-19P with the guns removed, provision for four Kaliningrad K-5 (AA-1 Alkali) beam-riding missiles, and the radar modified to support the missiles. Its NATO reporting name is Farmer-E.

The MiG-19PM was built under license and in China as the Shenyang J-6B.

The MiG-19PM entered service with the Soviet PVO in 1957. It later served with the Albanian Air Force, the Bulgarian Air Force, the Chinese PLAAF, the Czechoslovak Air Force, the East German Air Force, the Hungarian Air Force, the Iraqi Air Force, the Polish Air Force, and the Romanian Air Force.

\subsection{Armament and Stores}

The cannon-armed MiG-19s were equipped with either all 23~mm NR-23 or all 30~mm NR-30 cannons. Using homogeneous weapons was considered an improvement over the mixture of two 23~mm NR-23 and one 37~mm N-37 cannons in the MiG-15 and MiG-17, as the different calibers of these earlier guns had very different ballistic characteristics and so the shell streams diverged at long range.

Like most of its contemporaries, the MiG-19 suffered from short range on internal fuel, and this was mitigated by the use of under-wing 760L FTs.

The MiG-19 was potentially better suited to ground attack than the earlier MiG-15 or MiG-17, as it had four under-wing stations and so could carry both FTs and air-to-ground ordnance. However, this does not seem to have been exploited heavily except by the Pakistan Air Force. A typical air-to-ground load would be four ORO-57K RPs (on DRs) with a total of thirty-two 57~mm rockets along with two FTs.

\subsection{Combat}

A US reconnaissance flight was shot down by PVO MiG-19s in 1960. However, Soviet MiG-19s were not able to intercept U-2 overflights.

Egyptian MiG-19s were used for ground-attack missions in 1962 during the North Yemen Civil War. Egyptian and Syrian MiG-19s saw combat in the lead-up to and during the 1967 War. In Egyptian service, it also saw combat in the War of Attrition.

\subsection{ADCs}
\begin{adclist}
    \adcitem{MiG-19S (Early)}
    \adcitem{MiG-19S (Late)}
    \adcitem{MiG-19SF}
    \adcitem{MiG-19P (Early)}
    \adcitem{MiG-19P (Late)}
    \adcitem{MiG-19PM}
\end{adclist}

\subsection{See Also}
\begin{typelist}
    \typeitem{Shenyang J-6 and F-6}
\end{typelist}

\subsection{Photo Credit}
\begin{itemize}\raggedright
    \item \href{https://commons.wikimedia.org/wiki/File:MiG-19_Romania.png}{MiG-19}: Revista Cer Senin Nr\@. 4/2016 (Public Domain)
\end{itemize}
