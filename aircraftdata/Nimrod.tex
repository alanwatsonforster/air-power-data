\section{Hawker Siddeley Nimrod}

\includegraphics[width=\linewidth]{Nimrod.jpg}

The Hawker Siddeley Nimrod is a maritime patrol and signals intelligence aircraft. It was developed from the Comet jet airliner, and this line thus holds the distinction of being both the first jet airliner and the first jet maritime patrol aircraft.

\subsection{Versions}

\subsubsection{Nimrod MR.1}

The initial version was the MR.1 maritime patrol aircraft. The airframe of the MR.1 was developed from the Comet 4 jet airliner, with the addition of a large unpressurized pannier under the fuselage for sensors and weapons and the replacement of the original Rolls-Royce Avon turbojets with Rolls-Royce Spey turbofans to give longer endurance. The mission components, and in particular the ASV Mk~21 radar, were largely recycled from the Shackleton MR.3. The MR.1 entered service with the RAF in 1969.

\subsubsection{Nimrod R.1}

The Nimrod R.1 electronic and signals intelligence (ELINT and SIGINT) aircraft was developed from the MR.1. Signal detection equipment replaced the mission systems of the MR.1 and filled the weapons bays. The R.1 entered service with the RAF in 1974 and served until 2011. It was replaced by Boeing RC-135W Rivet Joint.

\subsubsection{Nimrod MR.2 and MR.2P}

Many of the MR.1 aircraft were upgraded to the MR.2 standard starting in 1975, gaining the much improved Searchwater radar and Yellow Gate ESM system. The remaining MR.1 aircraft were retired.

During the South Atlantic War, air-to-air refueling probes from Avro Vulcans were installed on several MR.2s to give the MR.2P version and the underwing stations were equipped with AIM-9G/L IRMs.

In 1990, some Nimrods were further equipped with decoy dispensers. In 2022, some gained TV/IR optics The MR.2 was retired in 2010.

\subsubsection{Nimrod AEW.3}

The Nimrod AEW.3 was a prototype airborne early-warning aircraft for the RAF. Development started in the 1970s and the project was cancelled in 1986s after significant technical problems, delays, and cost increases. The RAF acquired the Boeing E-3D Sentry for this role.

\subsubsection{Nimrod MR.4}

The Nimrod MR.4 was an advanced maritime patrol aircraft. It was based on existing MR.2 aircraft, but with new engines, wings, and systems. It was cancelled in 2010, when it was on the point of entering service. The RAF eventually acquired the Boeing P-8 Poseidon for this role.

\subsection{Armament and Stores}

The maritime patrol versions have three internal weapons bays for Mk.44, Mk.46, or Stingray torpedoes, Mk.11 conventional depth charges, Mk.57 nuclear depth charges, or auxiliary fuel tanks. During the South Atlantic War, they were also qualified to drop 1,000 lb bombs. During peacetime, one or two bays routinely carried air-droppable SAR equipment.

The two under-wing stations were originally intended to carry AS.12 or Martel missiles, but apparently they were not deployed. During the South Atlantic War, these stations were modified to each carry two AIM-9 Sidewinder missiles.

\subsection{Combat}

The Nimrod MR.2/2P and R.1 saw combat in the South Atlantic War, the Gulf War, the Invasion of Afghanistan, and the Invasion of Iraq. The R.1 further saw combat in the military intervention in Libya Civil War.

\subsection{ADCs}

ADCs are provided for:
\begin{adclist}
    \adcitem{Nimrod MR.1}
    \adcitem{Nimrod MR.2}
    \adcitem{Nimrod MR.2P}
\end{adclist}

\subsection{Photo Credit}
\begin{itemize}\raggedright
    \item \href{https://commons.wikimedia.org/wiki/File:British_Aerospace_Nimrod_MR.2,_United_Kingdom_-_Royal_Air_Force_(RAF)_JP506967.jpg}{Hawker Siddeley Nimrod}: Dale Coleman (GFDL 1.2)
\end{itemize}
