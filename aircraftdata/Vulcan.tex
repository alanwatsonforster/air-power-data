\section{Avro Vulcan}

\includegraphics[width=\linewidth]{Vulcan.jpg}

The Avro Vulcan was a strategic bomber. It featured a tailless delta-wing configuration designed for high-speed subsonic flight at high altitude, and was the last and most advanced of the V-bombers.

\subsection{Versions}

\subsubsection{Vulcan B.1}

The initial B.1 version had the Rolls-Royce Olympus 101 engine (12,000 lbf) and lacked ECM systems, tail-warning radar, and in-flight refueling capability. It entered service with the RAF in 1956. Many were converted to B.1As in 1959 to 1963 and the remainder retired in 1966.

\subsubsection{Vulcan B.1A}

The B.1A version was an upgrade of the B.1 with a more powerful Rolls-Royce Olympus 104 engines (13,500 lbf), ECM systems and a tail-warning radar similar to those in the B.2, and in-flight refueling capability. The first modified aircraft became available in 1960 and all were retired in 1968.

\subsubsection{Vulcan B.2}

The B.2 version was built with a larger wing to accommodate the Olympus 200/300-series engine (17,000 to 20,000 lbf), ECM systems and a tail-warning radar in an extended tail cone, and in-flight refueling capability. It served with the RAF from 1960 to 1982.

Some B.2s were modified to permit the carriage of the Blue Steel nuclear stand-off missile semi-recessed in the bomb bay and the cancelled Skybolt air-launched nuclear ballistic missile on under-wing station. TFR was fitted starting in 1966 and an improved RWR the mid 1970s.

In 1982, for the {\itshape Black Buck} missions in the South Atlantic War, several B.2s were modified to permit the carriage of Shrike ARMs or an ALQ-101D ECM pod on the underwing pylons originally installed for Skybolt.

\subsubsection{Vulcan B.2(MRR)}

Several B.2s were converted to maritime radar-reconnaissance aircraft (MRR) in 1973 and served until 1982.

\subsubsection{Vulcan K.2}

The South Atlantic War consumed much of the remaining fatigue life of the RAF's fleet of Victor tankers. To compensate, several Vulcan B.2s were converted to single-point tankers and designated K.2. They served from 1982 to 1984.

\subsubsection{Vulcan F.3}

The F.3 is a hypothetical long-range interceptor. At least two options were considered in the 1970s: one with twelve AIM-54 Phoenix missiles and another with ten air-launched variants of the Sea Dart missile.

\subsection{Armament and Stores}

The bomb bay could accommodate twenty-one 1,000 lb bombs, one nuclear bomb (Blue Danube, Violet Club, Mark 5, Yellow Sun, Red Beard, and WE.177B), or (in modified B.2s from 1960), or one Blue Steel nuclear stand-off missile. The B.2s modified for the {\itshape Black Buck} missions could also carry AGM-45 Shrike ARMs on under-wing stations.

\subsection{Combat}

Vulcans only saw combat in the {\itshape Black Buck} missions in the South Atlantic War.

\subsection{ADCs}

ADCs are provided for:
\begin{itemize}
        \adcitem{Vulcan B.2}
        \adcitem{Vulcan B.2(MRR)}
        \adcitem{Vulcan F.3 (Phoenix)}
        \adcitem{Vulcan F.3 (Sea Dart)}
\end{itemize}

\subsection{Photo Credit}
\begin{itemize}\raggedright
    \item \href{https://commons.wikimedia.org/wiki/File:Avro_Vulcan_XH558_Duxford_Airshow_2012_(7977149648).jpg}{Avro Vulcan}: John5199 (CC BY 2.0)
\end{itemize}
