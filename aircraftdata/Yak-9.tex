\section{Yakovlev Yak-9}

\includegraphics[width=\linewidth]{Yak-9D.jpg}

The Yakovlev Yak-9 was a propeller-driven fighter developed during WWII and which first entered service in 1942. After WWII, later versions were supplied to many Soviet allies. The NATO reporting name for the aircraft is “Frank.”

\subsection{Versions}

There were many versions of the Yak-9; these are the most relevant.

\subsubsection{Yak-9}

The first Yak-9 version was developed from the Yak-7B fighter. It was armed with one 20 mm ShVAK cannon firing through the spinner and one 12.7 mm UBS machine gun in the left wing. It entered service with the Soviet VVS in 1942.

\subsubsection{Yak-9T}

The Yak-9T was armed with a 37~mm NS-37 cannon in place of the 20~mm. The cockpit was moved backwards slightly to accommodate the cannon.

\subsubsection{Yak-9D}

The Yak-9D is a long-ranged escort fighter derived from the original Yak-9. It had additional fuel tanks in the wings, but otherwise was very similar.

\subsubsection{Yak-9M and Yak-9U}

The Yak-9M was also derived from the original Yak-9, but had the cockpit move backwards like the Yak-9T for commonality and a stronger construction, with the fuselage being skinned by plywood rather than fabric. It added a second 12.7 mm UBS machine gun in the right wing. The Yak-9U was an iteration of the Yak-9M with modifications to reduce defects during construction.

The Yak-9M entered service in the spring of 1994 and the Yak-9U followed before the end of WW2.

\subsubsection{Yak-9P}

The Yak-9P was a post-war development of the Yak-9U with completely metal wings.

The Yak-9U entered service with the VVS in 1946. It also served with many Soviet allies, including the Albanian Air Force, Bulgarian Air Force, Chinese PLAAF, Hungarian Air Force, North Korean KPAF (from 1949), Polish Air Force and Polish Navy, and the Yugoslav Air Force.

\subsection{Armament and Stores}

Most versions of the Yak-9 were armed with one 20 mm ShVAK cannon firing through the spinner and one or two 12.7 mm UBS machine guns in the wings.

Two 100 kg bombs could be carried under the wings.

\subsection{Combat}

Early versions of the Yak-9 fought in WW2. The Yak-9U/-9P were also used by the North Korean KPAF and saw extensive service in the first weeks of the Korean War.

\subsection{ADCs}

\begin{adclist}
    \adcitem{Yak-9D}
    \adcitem{Yak-9U}
    \adcitem{Yak-9P}
\end{adclist}

\subsection{Photo Credit}
\begin{itemize}\raggedright
    \item \href{https://commons.wikimedia.org/wiki/File:Yak-9Ds_at_Poltava,_Operation_Frantic.jpg}{Yak-9D}: USAAF photographer (Public Domain)
\end{itemize}