\section{Canadair Sabre}

\includegraphics[width=\linewidth]{Canadair Sabre.jpg}

The Canadair Sabre was a day fighter derived from the North American F-86 Sabre. The initial versions were only lightly modified, but later versions incorporated the more powerful Orenda engine. All the production versions were fitted with the original F-86 armament of six .50~cal M3 guns.

\subsection{Versions}

\subsubsection{Canadair Sabre Mk.1}

The single Mk.1 prototype was very similar to the F-86A.

\subsubsection{Canadair Sabre Mk.2}

The Mk.2 was the first production version and was essentially a license-built F-86E with the original slatted wing.

They were used by the RCAF in Europe and the USAF. Later, they were passed on to the air forces of Greece and Turkey. In USAF service, they saw combat in the Korean War.

\subsubsection{Canadair Sabre Mk.3}

The Canadair Sabres began to diverge from the North American originals with the single Mk.3 prototype, which used an Avro Canada Orenda 3 engine with significantly more thrust than that of the Mk.2.

\subsubsection{Canadair Sabre Mk.4}

The Mk.4 was very similar to the Mk.2. Later Mk.4s had the slatted 6-3 wing.

They were used in small numbers by the RCAF and in larger numbers by the RAF, where they were known as the Sabre F.4 and served alongside the Meteor F.8. They were the first swept-wing fighter in British service. Starting in 1954, they began to be refitted with the 6-3 wing. As Hawker Hunters became available in 1956, the RAF Sabres were transferred to the Yugoslav and Italian air forces.

\subsubsection{Canadair Sabre Mk.5}

The Mk.5 was a development of the Mk.3 prototype with the improved Orenda 10 engine and the unslatted 6-3 wing.

They were initially used by the RCAF, again mainly in Europe, replacing the Mk.2s. A number were later transferred to the Luftwaffe.

\subsubsection{Canadair Sabre Mk.6}

The Mk.6 was a development of the Mk.5 with the even more powerful Orenda 14 engine. Later Mk.6s had the slatted 6-3 wing. The Mk.6 competes with the CAC Sabre Mk.32 for the honor of being the very best day-fighter Sabre.

They replaced Mk.5s in RCAF service and also were used in large numbers by the Luftwaffe. These were later sold on to the Columbia, South Africa, and Pakistan. In PAF service, the Mk.6 was known, confusingly, as the F-86E and fought in the 1971 war with India.

\subsection{Armament and Stores}

All the production versions were fitted with the original F-86 armament of six .50~cal M3 guns.

A typical air-to-air load was two 200-gallon (750L) FTs on the outer stations and, from 1960 on the Mk.6, two AIM-9B IRMs on the inner stations.

A typical air-to-ground load was two 1000-lb bombs carried on the inner stations along with two 200-gallon (750L) FTs on the outer stations. Alternatively, on the later versions, sixteen HVAR rockets might have been carried without fuel tanks.

For ferry flights, two 120-gallon (400L) FTs could be carried on the inner stations and two 200-gallon (750L) FTs to the outer ones.

\subsection{Combat}

The Mk.2 saw combat with the USAF in the Korean War. The Mk.6 saw combat with the Pakistan Air Force (as the F-86E) in the 1971 war with India.

\subsection{ADCs}

\begin{adclist}
    \adcitem{Canadair Sabre Mk.2}
    \adcitem{Canadair Sabre Mk.4}
    \adcitem{Canadair Sabre Mk.4 (6-3 Wing)}
    \adcitem{Canadair Sabre Mk.5}
    \adcitem{Canadair Sabre Mk.6}
    \adcitem{Canadair Sabre Mk.6 (Slatted 6-3 Wing)}
\end{adclist}

\subsection{See Also}

\begin{typelist}
    \typeitem{CAC Sabre}
    \typeitem{North American F-86 Sabre}
\end{typelist}

\subsection{Photo Credit}
\begin{itemize}\raggedright
    \item \href{https://commons.wikimedia.org/wiki/File:Canadair_Sabre_0258.jpg}{Canadair Sabre}: Canadian Department of National Defence (Public Domain)
\end{itemize}
